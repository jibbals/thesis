
% Background Chapter
%%%
% New organisation:
%
% Introduction and Literature Review
%   Tropospheric ozone and air quality . . . . .
%   Isoprene and other VOCs . . . . . . . . . . .
%   Formaldehyde (HCHO) . . . . . . . . . . . .
%   Models . . . . . . . . . . . . . . . . . . . . .
%   Aims?
%%%%%%
\chapter{Introduction and Literature Review} % Chapter Title
\label{LR}
  
\section{Tropospheric ozone and air quality}
  \label{LR:O3andAQ}
  \subsection{Air Quality}
    \label{LR:O3andAQ:AQ}
    %%% AUSTRALIA
    
    % Moved to intro    
    
    \subsubsection{Ozone}
      %%% OZONE
      
      
      
    
    \subsubsection{Particulate matter and SOA}
      
    
    
    \subsubsection{Factors influencing ozone and PM}
      %\label{LR:O3andAQ:AQfactors}
      
    \subsubsection{How do we measure air quality?}
      
      
      
      
      
  \subsection{Ozone transported from the stratosphere}
    
    
    
    
    
    
    
    
  

  \subsection{Ozone formed in the troposphere}
    \label{LR:O3andAQ:BiogenicOzonePrecursors}
    
    
    
\section{Hydroxyl (OH) and other radicals}
  \label{LR:Radicals}
  
  
  
  
  

\section{Isoprene and other VOCs}
  \label{LR:VOCs}
  \subsection{What are VOCs}
    

    
  \subsection{What do they Do?}
    
    
    
  \subsection{Isoprene Cascade}
  
    
  \subsection{How and where do we measure them?}
    

  
  
  \subsection{Emissions estimates}
    \label{LR:VOCs:EmissionsEstimates}
    
    
  
\section{Formaldehyde (HCHO)}
  \label{LR:HCHO}
  
  %Paragraphs moved to Intro
  \subsection{How HCHO is measured}
  
  \subsection{Satellite Inversion}
  
  \subsection{Satellite HCHO detection}
    \label{LR:HCHO:SatelliteDetection}
    TODO: Refactor this section so it's readable
     
    
      
    
    
    \subsubsection{Satellite uncertainties}
      
    
    
  
    
\section{Models}
  \label{LR:Models}
  \subsection{How can models help}
    
    
    
    
    
  \subsection{Relevant model frameworks}
  \label{LR:Models:frames}
    
    % Outline of ACM
    Atmospheric chemistry models (ACMs) require various inputs and can be sensitive to ozone and oxidative parameterisations. 
    TODO: read more Christian 2017,
    TODO: put some more generic ACM info here.
    
    \subsubsection{Box models} 
    \label{LR:Models:frames:box}
    
      
      
    \subsubsection{Chemical transport} %% eg. GEOS-Chem
      
      
      The Model of Emissions of Gases and Aerosols in Nature (MEGAN) is one of the more commonly used natural emission models \citep{Monks2015}. (TODO: more cites which say this/use MEGAN)
    
    \subsubsection{Land based emissions} %% EG MEGAN
      % TODO: Overview of land based emissions modelling?
      
      
      

    \subsubsection{Radiative transfer} %% EG LIDORT
      %TODO: Lidort example?
      TODO: Lidort example?
    
  \subsection{Factors affecting isoprene emissions estimates}

      

      
      
      
      
      %Temperature has a strong exponential relationship with isoprene emissions, and can be readily seen in comparisons to a major isoprene product HCHO. 
  
  \subsection{Uncertainties}
    \label{LR:Models:Unc}
    
    
    
    
    
      
    
    % Transport uncertainties?
    \subsubsection{Transport}
      \label{LR:Models:Unc:Transport}
      TODO: Literature showing transport uncertainties or lack thereof     
      %TODO: examples of transport uncertainties
    
    
    
    
      
    
