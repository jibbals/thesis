%% CHAPTER 2 (probably)
%% MODELLING and DATA

\chapter{Data and Modelling} %with GEOS-Chem} % Main chapter title
\label{Model} %better reference name?
  
\section{Introduction}
  %Why use models?
  % Models
  
  In this thesis the word model is used in lieu of chemical transport model (CTM), a class of model that simulates chemistry and chemical transport through the atmosphere.
  Models of the atmosphere can be used to interpret measurements, estimate chemical concentrations at any scale, and predict atmospheric composition in the future.
  Some advanced measurement techniques make use of modelled a priori information in order to produce useful outputs.
  Models of ozone in the atmosphere are used broadly for international assessments of ozone precursor emissions, and estimating effects from related processes (such as radiation) \parencite{Young2018}.
  %\textcite{Young2018} summarise current global ozone modelling standards and the metrics and processes used to evaluate these models.
  Models provide an estimate of many trace gas concentrations; however, verification is required, and is generally performed using results from measurement campaigns.
  In situ measurements from campaigns or measurement stations can be used to examine what is happening at a particular location.
  These data are used to determine how accurate models or estimates are - however the utility is limited to where and when the measurements took place.
  In this thesis data from several campaigns are compared against model outputs and satellite datasets.
  %Satellite usage and reduction of uncertainty
  Satellite datasets provide large amounts of data over most of the planet.
  However, they can have high uncertainty due to instrument limitations.
  %Many datapoints can be averaged in order to reduce uncertainty.
  In this chapter several satellite datasets are combined to estimate biogenic HCHO amounts over Australia.
  
  The first goal is to analyse Australia-specific HCHO concentrations measured by satellite.
  %and determine isoprene sensitivity and any model bias.
  The second aim uses these concentrations along with model data to estimate isoprene emissions, which takes place in Chapter \ref{BioIsop}.
  The third goal is to quantify ozone transported from the stratosphere down into the troposphere (Chapter \ref{Ozone}).
  The focus in this chapter is to describe and analyse model outputs and measurements along with how they are calculated and compared for use in achieving the aims of this thesis.
  Section \ref{Model:datasets} details satellite and campaign datasets and describes model outputs.
  Measurement techniques used to retrieve the most utilised satellite dataset are outlined in Section \ref{Model:omhcho}.
  Section \ref{Model:GC} describes the GEOS-Chem model, how it is run and what setup and outputs are used in this thesis.
  In Section \ref{Model:AMF} the process of using model outputs to recalculate satellite vertical columns is described and analysed.
  In order to compare satellite data with other datasets, some work must be undertaken to avoid introducing bias \parencite[e.g.,][]{Palmer2001, Eskes2003, Marais2012, Lamsal2014}.
  One key step is to recalculate the satellite information using modelled data, detailed in Section \ref{Model:omiRecalc}.
  The effects of these recalculations on satellite HCHO is also examined.
  The creation and effects of filters used to remove non-biogenic influences are described in Section \ref{Model:filter}.
  
  
% put OMHCHO relevant stuff all into 2.3
% ADD MORE DESCRIPTIVE PICTURES TO SATELLITE DESCRIPTIONS

\section{Datasets}
  \label{Model:datasets}
  
  This section describes the datasets used in this thesis, along with an overview of the measurement techniques used for each.
  This includes modelled output, satellite measurements, and measurement campaigns.
  These datasets serve four purposes: 
  \begin{enumerate}
    \item Model output validation in this chapter
    \item Calculation of biogenic HCHO distribution over Australia in this chapter
    \item Recalculated OMI formaldehyde columns are used as a basis for estimating isoprene emissions in Chapter \ref{BioIsop}
    \item Extrapolation of ozone transport in Chapter \ref{Ozone}
  \end{enumerate}
  Details on filtering and interpolations which are undertaken when reading data are also provided, as each dataset has its own resolution.
  While no measurements are made throughout this thesis, it is important to understand the techniques used to create utilised datasets in order to understand possible anomalous datapoints or trends.
  
  % Coordinate systems in use
  Horizontal geographical coordinates are always discussed in terms of degrees north and east, from -180\degr ~to 180\degr ~spanning west to east around the globe and -90\degr ~to 90\degr ~spanning the latitudes from the Antarctic to the Arctic.
  Vertical resolution is sometimes discussed in terms of metres, sometimes in terms of pressure (hPa), and sometimes in terms of sigma ($\sigma$) coordinates.
  Sigma coordinates represent the fraction of the atmosphere vertically above, with $\sigma = 1$ being the surface and $\sigma = 0$ being the top of the atmosphere (TOA).
  This can be useful when running global atmospheric models as the ground altitude is always at $\sigma=1$ and we need not worry about topography.
  Conversion between $\sigma$ and pressure (p) coordinates is as follows:
  \begin{equation}
    \label{Model:datasets:eqn_sigma_pressure}
    \sigma = \frac{p_{S} - p}{p_{S}-p_{T}}
  \end{equation}
  where $p_{S}$ and $p_{T}$ is surface pressure and pressure at the TOA respectively.
  
  Uncertainty (or error) is present in each dataset and where possible the causes are explained.
  There are two types of error: systematic and random.
  Arguably the worst of these is systematic error (or bias).
  Bias normally indicates a problem in calculation or instrumentation.
  If the systematic error is known, it can be corrected for by either offsetting data in the opposite direction, or else fixing the cause.
  A proper fix can only be performed if the sources of error are known and there is a way of correcting or bypassing it.
  Random error is often reported as some function of a datasets variance, or uncertainty.
  It can be reduced through averaging either spatially or temporally. 
  Temporal and/or spatial averaging decreases uncertainty by a factor of $1/\sqrt{N}$ where N is the number of observations being averaged.
  
  \subsection{Satellite}
  \label{Model:datasets:satellite}
    % Satellite data product levels
    Satellite data products are generally classed into several categories, level zero through to level three. level zero products are sensor counts and orbital swath data, level one-B data calibrates and geo-locates the level zero data. 
    level two products additionally have temporal, spatial, solar, and viewing geometry information, as well as quality flags.
    %To create level two data slant column density is determined and then translated into vertical column density. 
    level three data is a temporally and spatially aggregated subset of level two data, for instance monthly or yearly averages.
    
    Difficulties can arise when aerosols (e.g., clouds, smoke, dust) interfere with recorded spectra; however,some of these can be detected and filtered out.
    Instruments including MODIS onboard the Aqua and Terra satellites are able to determine aerosol optical depth (AOD), a measure of atmospheric scatter and absorbance. 
    An AOD under 0.05 indicates a clear sky, while values of 1 or greater indicate increasingly hazy conditions.
    This is important in order to determine where measurements from other instruments may be compromised by high interference.
    Cloud filtering is performed on several satellite products used in this thesis, which reduces uncertainty at the cost of measurement quantity. 
    This has been seen to introduce a clear-sky bias in monthly averages since measurements do not include cloudy days \parencite{Surl2018}.
    
    % Satellites use DOAS for trace gases which we are interested in
    Satellite instruments measuring atmospheric composition record spectra between around 250-700~nm split into spectral components at around $0.3$~nm in order to calculate the abundance of trace gases including O$_3$, NO$_2$, and HCHO \parencite[e.g.,][]{Leue2001}.
    Satellite measurements are generally performed using spectral fitting followed by conversion to vertical column densities.
    %The use of multiple satellites can even be used to detect intradiel concentrations in trace gas columns, as shown in \textcite{Stavrakou2015} using OMI and GOME-2 measurements, which have respective overpass times of 1330 and 0930 LT.
    Several public data servers are available which include products from satellites, including NASA's Earthdata portal (\url{https://earthdata.nasa.gov/}) and the Belgian Institute for Space Aeronomy (IASB-BIRA) Aeronomie site (\url{http://h2co.aeronomie.be/}).
    The following subsections describe some of the satellite products used in this thesis.
    
    % RAYLEIGH AND MIE SCATTERING...
    %Rayleigh and Mie scattering describe two kinds of particle effects on radiation passing through a medium.
    %Rayleigh scattering is heavily wavelength dependent, and is the dominant form of scattering from particles up to roughly one tenth of the wavelength of the scattered light.
    %Mie scattering mostly involves larger particles, and has less wavelength dependence.
    %The effects of scattering provide information about substances in the atmosphere, as different particles and gases in the air have measurable properties seen by remote sensing devices (such as a satellite). 
    %Instruments are more or less sensitive to these properties depending on altitude, radiation, and other parameters \parencite[e.g.,][]{Martin2002}.
    
    
    
    
    %Satellite measured AOD requires validation by more accurate ground based instruments like those of AERONET which uses more than 200 sun photometers scattered globally.
    %Soon much more satellite data will be available in the form of geostationary satellite measurements \parencite{Kwon2017}.
    %\textcite{Kwon2017} examine simulated geostationary measurements against GEOS-Chem column simulations to determine the most important instrument sensitivities.
    %Geostationary satellites can provide temporally rich measurements over an area, as they are not sweeping around the earth but fixed relative to one latitude and longitude.
    
    \subsubsection{Formaldehyde}
      
      HCHO products can be found in four satellite instruments: GOME on ERS-2, SCIAMACHY on ENVI-SAT, OMI on EOS Aura, and GOME2 on MetOp-A and MetOp-B.
      These satellites have slightly different spectral and spatial resolutions, as well as using varied processes to estimate HCHO from detected radiances.
      This leads to different measurements between instruments \parencite{Lorente2017}, and both validation and comparison become more important when using these remotely sensed data.
      The data set used in this thesis is from the Ozone Monitoring Instrument (OMI) onboard the Aura satellite, as it has data over the time period covered by GEOS5 modelled meteorological information used in GEOS-Chem (Section \ref{Model:GC}), and sufficiently covers the southern hemisphere.
      
      %OMI spectra are used in several products used in this thesis, including OMHCHO, OMNO2d, and OMAERUVd.
      Satellite based formaldehyde measurements from the OMI instrument onboard Aura are stored in the OMHCHO product.
      OMHCHO data is used and modified extensively throughout this thesis, and so is discussed in more detail in Section \ref{Model:omhcho}.
      Calculation of column density and AMF are discussed respectively in sections \ref{Model:omhcho:DOAS} and \ref{Model:omhcho:amf}.
      
      
    \subsubsection{Nitrogen dioxide}
      \label{Model:datasets:OMNO2d}
      OMNO2d is a gridded daily level three NO$_2$ product of satellite measurements averaged into 0.25\degr x 0.25\degr ~horizontally resolved bins.
      An example from Jan 29, 2005 is shown in Figure \ref{Model:datasets:OMNO2d:fig_eg_omno2d}, while an average for 2005 (global) is shown in Figure \ref{Model:datasets:OMNO2d:fig_omno2d_2005}.
      OMNO2 measurement resolution is 40~km by 130~km.
      NO$_2$ measured by OMI is used in this thesis to check whether NO$_2$ is well represented by GEOS-Chem (see Section \ref{Model:GC:NOx} for the comparison between this product and GEOS-Chem calculations).
      It is also used to form the anthropogenic influence filter for calculating biogenic-only HCHO columns (see Section \ref{Model:filter:NOx}).
      
      
      \mypic{Figures/OMNO2d_2005m0129.png}{%
        Example of NO$_2$ tropospheric columns for 29 Jan. 2005 from the OMNO2d product.
      }{\label{Model:datasets:OMNO2d:fig_eg_omno2d}}
      
      \mypic{Figures/OMNO2d_avg2005.png}{%
        Average 2005 tropospheric NO$_2$ from OMNO2d with pixels screened for $<30\%$ cloud cover.
      }{\label{Model:datasets:OMNO2d:fig_omno2d_2005}}
      
      Like other satellite products, OMNO2d is influenced by a priori model profiles, which are required to convert slant path radiance to vertical columns.
      These models are generally low resolution ($\sim$ 110~km $\times$ 110~km), which leads to column smearing and difficulty detecting point sources of high NO emissions \parencite{Goldberg2018}.
      Uncertainty in this product arises mostly from the calculation of the AMF (up to 50\% of total error) \parencite{Lorente2017}.
    
    \subsubsection{Aerosol optical depth}
      \label{Model:datasets:OMAERUVd}
      
      % smoke aaod outline
      Aerosols in the atmosphere can be seen through their interactions with visible radiation. 
      Smoke and dust can be seen as an increase in aerosol optical depth (AOD) (see Section \ref{Model:omhcho:DOAS}).
      This is due these particles scattering and absorbing UV radiation \parencite{Ahn2008}.
      A data product provided by Earthdata (\url{https://disc.gsfc.nasa.gov/datasets/OMAERUVd_V003/summary}) called OMAERUVd (DOI: 10.5067/Aura/OMI/DATA3003) is used in this thesis. 
      
      
      OMAERUVd allows detection of areas which may be smoke-affected.
      The product contains AOD and aerosol absorption optical depths (AAOD) at three wavelengths (354, 388, and 500~nm), along with UV aerosol index.
      The OMAERUVd product is level three, based on quality filtered level two swath pixels which are then gridded by daily averaging.
      These products are most sensitive to error in the form of sub-pixel scale cloud interference, however AAOD is the less sensitive of the two \parencite{Ahn2008}.
      In this thesis AAOD is used to filter out potential smoke-affected areas.
      
      % How I read the AAOD
      In this work AAOD is mapped from 1\degr x1\degr ~horizontal resolution to \highhr ~using nearest value mapping.
      The AAOD at 500~nm wavelength is used to determine smoke influence, although any of the provided wavelengths would be affected by smoke plumes and could also be used.
      %This daily AAOD is compared to a threshold to create a daily smoke filter, 
      Any areas with daily AAOD$>0.03$ are considered to be potentially smoke plume affected (see Section \ref{Model:filter:fire}).
    
    
    \subsubsection{Active fires}
      \label{Model:datasets:MOD14A1}
      MOD14A1 is a gridded daily satellite dataset of fire counts at 1x1~km$^2$ horizontal resolution.
      Fire counts are observed four times daily from Terra (10:30~LT, 22:30~LT) and Aqua (01:30~LT, 13:30~LT).
      The fire pixels are detected based on parameters including apparent pixel temperature and the nearby background temperature.
      The dataset is obtained from NASA Earth Observations that is part of the EOS Project Science Office at the NASA Goddard Space Flight Center \url{https://neo.sci.gsfc.nasa.gov/view.php?datasetId=MOD14A1_M_FIRE}.
      This product is downloaded and binned into the lower \highhr ~resolution (using the sum of fire pixels) to create an active fire influence mask (see Section \ref{Model:filter:fire}).
      
    \subsubsection{Carbon monoxide}
      \label{Model:datasets:AIRS}
      
      In Chapter \ref{Ozone}, potential biomass burning plumes are identified using satellite observations of carbon monoxide (CO) from the AIRS (Atmospheric Infra-red Sounder) instrument aboard the Aqua satellite \parencite{AIRS3STD}.
      The Aqua satellite overpasses at approximately 01:30~LT and 13:30~LT, with NASA producing level three gridded output at 1\degr x1\degr .
      CO is used as a proxy for biomass burning plumes, and used to qualitatively attribute ozone intrusion events (Section \ref{Ozone:WeatherClassifications}).
      %This is a method of detecting fire influence near specific sites through visual analysis.
      

    \subsubsection{Uncertainties}
      
      %Satellite Errors
      While satellite data is effective at covering vast areas (the entire earth) it only provides information at a particular time of day, is affected by cloud cover, and generally does not have fine vertical resolution.
      Measurements retrieved by satellites have large uncertainties, which arise from the process of transforming spectra into total column measurements, as well as from instrument degradation (satellite instruments are hard to repair once they are launched).
      Uncertainty in transforming satellite spectra comes from a range of issues, including measurement difficulties introduced by clouds, and instrument sensitivity to particular aerosols \parencite{Millet2006}.
      Many products require analysis of cloud and aerosol properties in order to estimate total column amounts \parencite{Palmer2001,Palmer2003, Marais2012, Vasilkov2017}.
      
      %% Detection limit 
      %A common way of reducing satellite uncertainty is through oversampling or temporal averaging.
      %This is done frequently for trace gases (which are often near to the detection limit over much of the globe).
      %For example: \textcite{Vigouroux2009} reduce the measurement uncertainty (in SCIAMACHY HCHO columns) by at least a factor of 4 through averaging daily over roughly 500km around Saint-Denis, and only using days with at least 20 good measurements.
      %Another example of this can be seen in \textcite{Dufour2009}, where monthly averaging is used to decrease the measurements uncertainty at the cost of temporal resolution.
      %They examine HCHO in Europe, which is low; near the detection limit of satellite measurements.
      
      %% SURFACE CONDITIONS AND CLOUDS
      %In cloudy, hazy or polluted areas measurements are more difficult to analyse \parencite[e.g.,][]{Palmer2003,Marais2014}.
      %Recent work by \textcite{Vasilkov2017} showed that updating how the surface reflectivity is incorporated into satellite measurements can change the retrievals by $50\%$ in polluted areas.
      
  \subsection{Model datasets}
    
    \subsubsection{GEOS-Chem output}
      
      The GEOS-Chem model is used extensively in this thesis and is discussed in more detail in Section \ref{Model:GC}.
      GEOS-Chem model outputs are described in Section \ref{Model:GC:simulations:outputs}.
      These are generally resolved to 47 vertical levels from the ground up to 0.01~hPa, at \lowhr ~horizontal resolution.
    
    \subsubsection{Meteorological reanalysis}
      \label{Model:datasets:ERAI}
      
      Synoptic scale weather patterns are taken from the European Centre for Medium-range Weather Forecasts (ECMWF) Interim Reanalysis (ERA-I) \parencite{Dee2011}.
      These are used in Chapter \ref{Ozone} to determine typical weather systems for stratospheric ozone intrusions.
      The ERA-I output was the most up to date at the time (2016) but has since been superseded by ERA5.
    
    \subsubsection{Surface temperatures}
    \label{Model:datasets:model:CPC}
      The Climate Prediction Center (CPC) provides a product with maximum daily land-surface temperature at 0.5\degr x0.5\degr ~ horizontal resolution. 
      These data are used to check the correlation between HCHO and temperature at a higher resolution than is provided by GEOS-Chem output.
      A full description of the data can be found at \url{https://www.esrl.noaa.gov/psd/data/gridded/data.cpc.globaltemp.html}.
      CPC Global Temperature data is provided by National Oceanic and Atmospheric Administration/Office of Oceanic and Atmospheric Research/Earth System Research Laboratory Physical Sciences Division, Boulder, Colorado, USA, from their web site at \url{https://www.esrl.noaa.gov/psd/}.
      An example of one day of land temperature output is shown in Figure \ref{Model:datasets:fig_CPC_EG}.
      
      \mypicw{.5\textwidth}{%
        Figures/CPC_Temperature_EG.png}{CPC daily maximum temperature data for 1 Jan. 2005.
      }{\label{Model:datasets:fig_CPC_EG}}
  
  \subsection{In-situ and ground based datasets}
  
    Data from several in-situ measurement campaigns and long-term measurement programs are used to examine the accuracy of modelled data at specific sites.
    Figure \ref{Model:datasets:fig_locations} shows the locations of ground-based measurement sites in the top panel, and release sites for ozonesondes in the bottom panel.
    
    % Figure from tests/test_campaigns.py
    \mypic{Figures/campaigns/campaign_locations.png}{%
      Locations of ground based in-situ measurements (top panel) and ozonesonde release sites (bottom panel). 
    }{\label{Model:datasets:fig_locations}}
    
    \subsubsection{Short term measurement campaigns}
      Some campaign datasets (described in this section) provide separate time series for brief periods of both isoprene and formaldehyde.
      Figure \ref{Model:datasets:fig_campaigns_compared} shows these along with the detection limits and also shows isoprene measurements superimposed over a single year.
      It is apparent that more measurements are required to see more than the daily cycles.
      
      \mypic{Figures/OMI_link/campaigns_compared.png}{ %
        Top: MUMBA, SPS1, and SPS2 time-series for HCHO (orange) and isoprene (green), along with detection limits (dashed).
        Bottom: isoprene measurements superimposed onto a single year.
        }{\label{Model:datasets:fig_campaigns_compared}}
    
    %\subsubsection{Measurements of Urban, Marine and Biogenic Air (MUMBA)}
    \label{Model:datasets:MUMBA}
    
      The MUMBA campaign \parencite{PatonWalsh2017} measured various compound abundances including isoprene, formaldehyde, and ozone from 21 December 2012 to 15 February 2013.
      These measurements took place in Wollongong, 10~m above ground level (40~m above sea level).
      Ozone was measured by Thermo UV absorption with 1-minute time resolution averaged into hourly outputs.
      Isoprene and HCHO were measured by Ionicon Proton-Transfer-Reaction Mass spectrometer (PTR-MS), with a time resolution of 3-minutes, averaged each hour.
      Detection limits varied due to instrument conditions, and are listed in Table \ref{Model:datasets:MUMBA:tab_detectionlimits}.
      The full dataset has been published on PANGAEA (DOI:10.1594/PANGAEA.871982) \parencite{Guerette2018}.
      
      
      \begin{table}
        \caption{Detection limits for MUMBA}
        \begin{tabular}{  l |  l  l  l }
          
          \textbf{Dates} & \textbf{HCHO (ppb)} & \textbf{Isoprene (ppb)} & \textbf{Ozone (ppb)}
          \\ \hline
          21/Dec/2012 - 29/Dec/2012 & 0.205 & 0.003 & 0.5 \\
          29/Dec/2012 - 18/Jan/2013 & 0.105 & 0.005 & 0.5 \\
          19/Jan/2013 - 15/Feb/2013 & 0.186 & 0.003 & 0.5 \\
        \end{tabular}
        \label{Model:datasets:MUMBA:tab_detectionlimits}
      \end{table}
      
      Isoprene concentration measurement uncertainty in this product is estimated by \textcite{Dunne2018} to be $50\%$.
      Although the uncertainty determined through calibration measurements was only 15\% \parencite{Guerette2018}, this does not account for competing trace gas interference (such as furan).
      When used in this thesis, readings are re-sampled to hourly averages, and measurements below the detection limit are set to half of the detection limit.
      % Measurement difficulties
      %Measurements contain errors, and depending on the device used and compound being measured this error can be significant.
      The major sources of uncertainty include interference from non-target compounds and under-reporting \parencite[e.g.,][]{Dunne2018,Guerette2018}.
      %Overall isoprene uncertainty in measurements analysed by \textcite{Dunne2018} was $50 - 100\%$.
      %This can feed into uncertainties in modelling and satellite retrievals, as verification and correlations are affected.
      
      
    %\subsubsection{Sydney Particle Studies (SPS1, SPS2)}
    \label{Model:datasets:SPS}
      Two trace gas measurement campaigns took place at the Westmead air quality station: %with scientists from CSIRO, OEH, and ANSTO 
      stage 1 (SPS1) from 5 February to 7 March, 2011 and stage 2 (SPS2) from 16 April to 14 May, 2012.
      Two instruments measured VOC concentrations: a PTR-MS, and a gas chromatograph with a flame ionisation detector (GC-FID).
      The PTR-MS uses chemical ionisation mass spectrometry and can quantify VOCs at high temporal resolution ($< 1$~s).
      It was calibrated several times per day against HCHO, isoprene, $\alpha$-pinene, and several other VOCs.
      Further measurement specifics can be found in \textcite{Dunne2018}.
      % Dunne 2018 looks like an analysis of VOC readings from the sites
      
      The output provides hourly averaged concentrations of trace gases based on the mass to charge ratio (m/z), which for isoprene is 69.
      % Sensitivities
      It is possible that other chemicals (such as furan, with the same m/z) interfered with this value, especially at low ambient isoprene concentrations and towards the end of autumn (SPS2) when wood fires start to become frequent \parencite{Guerette2018}.
      The GC-FID analysed samples collected in multi-absorbent tubes, with lower temporal resolution but no interference. 
      Further details for this method can be found in \textcite{Cheng2016}.
      GC-FID data are averaged from 0500-1000~LT, and 1100-1900~LT, while PTR-MS data are averaged hourly.
      Significant differences occur between measurement devices when detecting isoprene, potentially due to interfering compounds in the PTR-MS \parencite{Dunne2018}.
      
      
    \subsubsection{Ozonesondes}
    \label{Model:datasets:ozonesondes}
    
      Ozonesonde data come from the World Ozone and Ultraviolet Data Centre (WOUDC).
      Ozonesondes are weather balloons that measure ozone concentrations using an electrochemical concentration cell (\url{http://www.ndsc.ncep.noaa.gov/organize/protocols/appendix5/}).
      Ozonesondes provide a high vertical resolution profile of ozone, temperature, pressure, and humidity.
      Generally the instrument will perform 150-300 measurements from the surface to around 35km.
      Ozonesondes are launched approximately weekly from Melbourne (38\degr ~S, 145\degr ~E), Macquarie Island (55\degr ~S, 159\degr ~E) and Davis (69\degr ~S, 78\degr ~E). 
      % This stuff is in chapter 4 I think
      %Melbourne, a major city with more than 4 million residents \parencite{ABS2016}, may be affected by anthropogenic pollution in the lower troposphere.
      %%Actual releases are north of the central business district in the Broadmeadows suburb.
      %Macquarie Island is in the remote Southern Ocean and unlikely to be affected by any local pollution events.
      %Davis (on the coast of Antarctica) is also unlikely to experience the effects of anthropogenic pollution.
      More information on this dataset is given in Section \ref{Ozone:DataMethods}.
    
    \subsubsection{Wollongong FTIR}
    \label{Model:datasets:wollongong_ftir}
    
      Upon the roof of the Chemistry building at the University of Wollongong lies a solar Fourier transform infra-red spectrometer (FTIR) which measures HCHO (amongst other gases) in the path of light between the instrument and the sun.
      %%HCHO is measured using mid infra spectrum measurements transformed 
      The instrument is part of the Network for the Detection of Atmospheric Composition Change (NDACC) and data can be retrieved from the NDACC database (\url{http://www.ndaccdemo.org/stations/wollongong-australia}).
      The current principle investigator producing and quality assuring the data set is \href{mailto:njones@uow.edu.au}{Dr. Nicholas Jones}.
      
      Measurements of vertical profiles are affected by the instrument's sensitivity to HCHO at all vertical levels.
      Vertical mixing ratios resolved to 48 vertical levels are retrieved. %with the help of a priori assumed vertical profiles ($x_{apri}$).
      The averaging kernel ($A$) essentially represents the sensitivity of the retrieval technique to HCHO concentrations.
      Figure \ref{Model:datasets:wollongong_ftir:fig_apriori} shows averaged retrievals $x_{ret}$ and the a priori side by side.
      Most of the HCHO is close to the surface, and generally the a priori lies within the inter-quartile range of the retrievals, just below the mean.
      Figure \ref{Model:datasets:wollongong_ftir:fig_MiddayAK} shows how the retrievals of total vertical column HCHO ($\Omega$) are sensitive to HCHO between 0~km and 60~km altitude.
      The panel on the right shows how each retrieved vertical level is sensitive to HCHO at all levels.
      This shows there is not much sensitivity at higher altitudes and HCHO at lower altitudes can affect retrieved values for higher altitudes.
      When comparing FTIR measurements to modelled data, vertical columns are used rather than profiles as the degrees of freedom in each measurement is low ($\sim 1.3$ in summer to $\sim 1.7$ in winter).
      Degrees of freedom signify how many pieces of information can be retrieved from a measurement, and profiles resolved to \textit{N} vertical levels would require at least \textit{N}-1 degrees of freedom.
      
      \mypicw{0.5\textwidth}{Figures/OMI_link/FTIR_apriori.png}{%
        Mean midday FTIR profile ($x_{ret}$) and a priori profile ($x_{apri}$).
        Shaded area shows the inter-quartile range over the same time period.
        Profile is averaged between November 2007 and April 2013.
      }{\label{Model:datasets:wollongong_ftir:fig_apriori}}
      
      \mypic{Figures/OMI_link/FTIR_midday_AK.png}{%
        Left panel shows the mean midday (13:00 - 14:00 local time) total column averaging kernel, along with the inter-quartile range (IQR) between November 2007 and April 2013. 
        Right panel shows the mean averaging kernel for the vertical profile over the same time period, coloured by vertical level.
        One in six vertical levels are labelled and plotted with a solid line, and the rest are shown with dashed lines.
        The x-axes are unitless in the two plots shown here.
      }{\label{Model:datasets:wollongong_ftir:fig_MiddayAK}}
      
      
      
      The retrieved vertical profile ($x_{ret}$) can be expressed as a linear function of the true vertical profile ($x_{true}$) and the a priori vertical profile ($x_{apri}$):
      \begin{equation}
        \label{Model:datasets:wollongong_ftir:eqn_retrieval}
        x_{ret} = x_{apri} + A \left( x_{true} - x_{apri} \right)
      \end{equation}
      One way to understand Equation \ref{Model:datasets:wollongong_ftir:eqn_retrieval} is to consider that if the device was perfectly sensitive to HCHO concentrations at all vertical levels (making $A=I$ the identity matrix), then the retrieval would be equal to the true profile.
      If the instrument was perfectly insensitive (making $A=0$), then the retrieval would completely ignore the true profile and simply equal the a priori.
      To compare modelled vertical profiles ($x_{GC}$) with the FTIR, the modelled profiles are \textit{convolved} with the instrument averaging kernel and a priori to create a pseudo retrieval ($x'_{GC}$) which shows what the instrument would retrieve if the model were the truth:
      \begin{equation}
        x'_{GC} = x_{a} + A \left( x_{GC} - x_{a} \right)
      \end{equation}
      $x'_{GC}$ can be compared against $x_{ret}$ to determine if measurements and modelled vertical profiles or columns show significant differences.
      This convolution is required so as to avoid introducing a bias between the instrument and the model which is due to the instrument lack of sensitivity.
      Retrieved vertical profiles can be transformed into vertical columns ($\Omega$) using the hydrostatic relationship and an assumed dry air profile (see \textcite{Deeter2002}):
      %\url{https://www.acom.ucar.edu/mopitt/avg_krnls_app.pdf})
      \begin{equation}
        \Omega = x \times 2.12 \times 10^{13} \times \Delta p
      \end{equation}
      where $\Omega$ is in \moleccm, $x$ is in ppbv, and $\Delta p$ is the pressure in hPa between pressure midpoints for each vertical level in $x$.

\section{OMI satellite formaldehyde}
\label{Model:omhcho}
  
  % Aura satellite trace gas measurements
  One satellite product used extensively in this thesis is OMHCHO from NASA's Earth Observing System's Aura satellite. %, which provides several other useful datasets (products).
  Aura orbits the earth in a polar sun-synchronous pattern, circling the earth on a (geometric) plane coincident with the sun and the poles.
  % omi instrument onboard aura
  Aura houses the Ozone Monitoring Instrument (OMI), a near-UV/Visible Charged Coupled Device (CCD) spectrometer.
  The OMI instrument onboard Aura has been active since July 2005.
  It records spectra from 264-504~nm using an array of 60 detectors with moderate resolution (0.4-0.6~nm).
  This band of wavelengths allows measurements of trace gases (among other quantities) and the formaldehyde product is detailed here.
  
  
  From here onwards the word pixel is used to describe one data point retrieved by OMI.
  Each pixel includes a latitude and longitude within OMI's data product.
  Figure \ref{LR:HCHO:Sat:fig_Shenkeveld_OMI_summary} shows the details of OMI's detector array and measurement resolutions.
  %OMI measures atmospheric trace gases including NO$_2$, SO$_2$, BrO, HCHO, O$_3$, and aerosols.
  OMI measurements occur from right to left on a band covering a viewing angle of 115$^{\circ}$, resulting in swaths of around 2600~km, with pixel sizes from 13x24~km$^2$ at nadir to 26x135~km$^2$ at the swath edges \parencite{Abad2015}.
  The swaths cover Earth daily, both on the light and dark side of the planet, but only daytime measurements provide useful near-UV/Visible information.
  While satellite measurements can only be used during daytime hours, HCHO lifetimes are sufficiently short that any night-time chemistry will not affect midday observations \parencite{Wolfe2016}.
  
  \begin{figure}
    \includegraphics[width=\textwidth]{Figures/Shenkeveld_OMI_summary.png}
    \caption{ %
      %``An impression of OMI flying over the Earth.
      %The spectrum of a ground pixel is projected on the wavelength dimension of the charge-coupled device (CCD; the columns). 
      %The cross-track ground pixels are projected on the swath dimension of the CCD (the rows).
      %The forward speed of 7~kms$^{-1}$ and an exposure time of 2~s lead to a ground pixel size of 13~km in the flight direction.
      %The viewing angle of 114\degr ~leads to a swath width on the ground of 2600~km.''
      Here is an impression of the OMI scanning swathes as the housing satellite sweeps around the Earth.
      The charge-coupled device (CCD) has two dimensions: wavelengths and cross-tracks.
      The table shows the optical properties for OMI's three channels.
      Reproduced from \textcite{Schenkeveld2017}
    }\label{LR:HCHO:Sat:fig_Shenkeveld_OMI_summary}
  \end{figure}
  
  % How omi gets its AMF
  The latest OMHCHO algorithm uses a shape factor determined from GEOS-Chem (V9) using 47 vertical levels at monthly temporal resolution and \lowhr ~latitude by longitude horizontal resolution \parencite{Abad2015}.
  The GEOS-Chem model has been substantially updated since then.
  In this thesis a more recent version $V10.01$ is used to recalculate the vertical column HCHO (details are shown in Section \ref{Model:omiRecalc}).
  
  OMI uses a Differential Optical Absorption Spectroscopy (DOAS) based technique to retrieve HCHO along the path of light that reaches the satellite instrument.
  The first step is to determine how much HCHO is in the path of light between the sun and detector, which is done by applying a forward radiative transfer model (RTM) (see Section \ref{Model:Meas:sat:LIDORT}) in order to determine the radiative properties of a trace gas at various altitudes.
  The forward RTM used for satellite data products also involve functions representing extinction from Mie and Rayleigh scattering, and the effect of these on spectra.
  These RTM are also required to account for (often estimated) atmospheric parameters such as albedo.
  The next step is to transform the calculated amounts along the non-vertical light path into vertical column amounts.
  This is done by applying an AMF.
  In the absence of atmospheric scattering a simple geometric AMF can be defined as a function of the solar zenith angle. 
  The solar zenith angle ($\theta_s$) and the satellite viewing angle ($\theta_v$) are shown in image \ref{ch_HCHO:fig:zenithangle}.
  However, in the UV-VIS region of the spectrum, Rayleigh and Mie scattering (see Section \ref{Model:omhcho:DOAS}) must also be accounted for.
  
  
  \begin{figure}\begin{center}
      \includegraphics[width=0.6\textwidth]{Figures/ZenithAngles.png}
      \caption{Solar and viewing zenith angles, reproduced from \url{http://sacs.aeronomie.be/info/sza.php}. %\textcite{SZA_Image}, originally from a NASA website.
        }
      \label{ch_HCHO:fig:zenithangle}
    \end{center}\end{figure}
  
  % How omhcho is produced
  %A DOAS fit determines the total column amount of a trace gas along the path that the instrument views.
  %This uses the Beer-Lambert law where radiance is reduced as light travels through a medium.
  In this thesis I use the NASA OMHCHOv003 data product \parencite{Abad2015}, with HCHO determined using the spectral window $328.5$~nm$ - 356.5$~nm. 
  The algorithm used is based on direct fitting of radiances, and accounts for competing absorbers, under-sampling, and Ring effects.
  %An OMI radiance measurement over the remote Pacific ocean is used instead of an irradiance measurement.
  %This means that the 
  Slant columns ($\Os$) are determined from the spectra differential with respect to radiance reference column over the remote Pacific.
  The full method details for slant column retrieval by OMI are outlined 
  %in supplemental Section \ref{SuppNotes:Satellite:OMI_BOAS}, or 
  in the technical document (\url{https://docserver.gesdisc.eosdis.nasa.gov/repository/Mission/OMI/3.3_ScienceDataProductDocumentation/3.3.4_ProductGenerationAlgorithm/ATBD-OMI-04.pdf}).
  Slant columns range from $\sim 4\times 10^{15} $ to $\sim 6 \times 10^{16}$~molec cm$^{-2}$, with uncertainties from 30\% (larger columns) to over 100\% (smaller columns) \parencite{Abad2015}.
  %Atmospheric HCHO detected by satellite requires that other trace gases with similar features near the HCHO affected wavelengths are accounted for.
  

  \subsection{Pixel filtering}
  \label{Model:omhcho:pixel_filtering}
  
    % What the swathes look like
    This thesis uses the level two OMHCHO product swath output from the NASA earth data web portal.
    OMHCHO level two data includes 14-15 daily swaths of measurements.
    Each swath contains roughly $9 \times 10^4$ pixels, each of which includes latitude, longitude, vertical column HCHO, along with all the ancillary data required to calculate the vertical column and several data quality metrics.
    The OMHCHO dataset has a quality flag that can be used to remove unlikely or poor satellite measurements.
    The states represented by this quality flag are shown in Table \ref{Model:datasets:OMHCHO:tab_qflag} that is reproduced here from \textcite{Kurosu2014}.
    First all \textit{good} pixels (those with QA flag equal to 0) are read into a long list (roughly 1 million per day).
    Filtering bad or missing measurement pixels is performed prior to any other filtering.
    This includes the datapoints affected by the row anomaly (see \url{https://www.cfa.harvard.edu/atmosphere/Instruments/OMI/PGEReleases/READMEs/OMHCHO_README_v3.0.pdf}).
    This anomaly affects radiance data at particular viewing angles, corresponding to a row on the CCD detectors, and is dynamic over time \parencite{Huang2018}.
    The slant columns affected are flagged and removed before any further processing.
    
    \begin{table}
      \caption{OMI quality flag values from \textcite{Kurosu2014}}
      \begin{tabular}{  l  l  p{10cm} }
        \hline
        \textbf{Value} & \textbf{Classification} & \textbf{Rational} 
        \\ \hline
        0 & Good & Column value present and passes all quality checks; data may be used with confidence. 
        \\ \hline
        1 & Suspect & Caution advised because one or more of the following conditions are present: 
        \begin{itemize}
          \item Fit convergence flag is $<$ 300 but $>$ 0: Convergence at noise level
          \item Column $+ 2 \sigma$ uncertainty $<$ 0 $<$ Column $ + 3 \sigma $ uncertainty
          \item Absolute column value $>$ Maximum column amount (1e19 molec cm$^{-2}$)
        \end{itemize}
        \\ \hline
        2 & Bad & Avoid using as one of the following conditions are present: 
        \begin{itemize}
          \item Fit convergence flag is $<$ 0 : No convergence, abnormal termination
          \item Column $+ 3 \sigma$ uncertainty $<$ 0
        \end{itemize}
        \\ \hline
        $<0$ & Missing & No column values have been computed; entries are missing
        \\ \hline
      \end{tabular}
      \label{Model:datasets:OMHCHO:tab_qflag}
    \end{table}
    
    Approximately every $90$ minutes, the Aura satellite sweeps over the sun lit side of the planet and makes around 90000 measurements, of which 50000 - 80000 pixels are classified as good.
    Each pixel contains several important pieces of data that are needed for recalculation of the HCHO vertical column: the total column of HCHO ($\Omega$; \moleccm), cloud fraction, associated shape factor, AMF, geometric AMF, scattering weights and their vertical altitudes (hPa), viewing zenith angle, solar zenith angle, latitude, longitude, OMI sensor track, main data quality flag, cross track flag, and total column uncertainty.
    All of these data are needed in order to reconstruct the total vertical column using a new a priori estimate of the vertical profile in lieu of that provided by NASA.
    This reconstruction is required so that perceived bias between satellite measurements and model outputs is not due to a priori information, but rather what the satellite is producing.
    %Each pixel includes an estimate of the cloud fraction created using the OMI cloud product OMCLDO2.
    
    %Filtering
    After being filtered for data quality, pixels are filtered by solar zenith angle (SZA) and latitude, similarly to other works \parencite[e.g.,][]{Marais2012, Barkley2013, Zhu2014, Bauwens2016, Zhu2016}.
    Satellite measurements polewards of 60\degr ~north or south are removed as well as measurements with SZA greater than 60\degr.
    % UP TO HERE JENNY
    Measurements with high SZA are unlikely to have much information near the surface, and often have high uncertainty in total column retrievals \parencite[e.g.,][]{Stone2015}.
    Pixels with cloud fraction greater than 40\% are removed after determining the reference sector correction (Section \ref{Model:omiRecalc:RSC}), as in \textcite{Abad2015, DeSmedt2015}.
    This removes around 30\% of the pixels that remain after filtering out the bad or missing data.

    % Negative columns
    Due to noise and the differential measurement technique, small negative columns are present in the satellite product that are not removed so as not to introduce a bias.
    However, some very large negative values are removed, as these represent an unflagged error in the retrieval process.
    Unlikely high positive measurements are also removed, leaving only measurements within the range $-0.5 \times 10^{16}$ to $1 \times 10^{17} $\moleccm, as is performed by \textcite{Zhu2016}.
    This filter is required due to currently unexplained large negative values which occur in the OMI HCHO product increasingly over time.
    Figure \ref{Model:omhcho:pixel_filtering:fig_OMI_negative_hist} shows how unfiltered HCHO columns are affected by a small set of highly negative values that heavily affect the mean column amount over any region.
    The highly negative values can be seen around $\Omega = -10^{19}$~molecules cm$^{-2}$.
    
    % Figure from ?!?!
    \mypic{Figures/AusOMHCHO_Hist_20130318_final.png}
    {
      HCHO Column amount histograms for a subset of OMI swaths over Australia on the 18th of March 2013.
      Negative entries are shown in the left panel, positive in the right.
      Note the different scale between negative and positive panels.
    }{\label{Model:omhcho:pixel_filtering:fig_OMI_negative_hist}}
  
  
  
  \subsection{Air mass factor (AMF)}
  \label{Model:omhcho:amf}
    % what satellite amfs are, and what they do
    To convert the trace gas profile from a reflected solar radiance column (slanted along the light path) into a purely vertical column requires calculations of an air mass factor (AMF).
    In satellite data, the AMF is typically a scalar value for each horizontal grid point that gives the ratio of the total vertical column density to the total slant column density.
    An AMF characterises measurement sensitivity to a trace gas at various altitudes \parencite{Palmer2001}.
    This value %requires calculations to account for instrument sensitivities to various wavelengths over resolved altitudes, and 
    is unique for each trace gas under consideration \parencite{Palmer2001,Millet2006}.
    \textcite{Lorente2017} show that AMF calculations can be the largest source of uncertainty in satellite measurements.
    %Another way of describing AMFs are as measures of how radiance at the top of the atmosphere (TOA) changes with trace gas optical depths at specific altitudes \parencite{Lorente2017}.
    Calculation of the AMF is important as it is multiplied against the estimated slant columns in order to give vertical column amounts.
    
    %how do we get an AMF?
    DOAS column retrievals are an integration of a trace gas over the instrument's viewing path.
    In order to convert this total to a vertically distributed column a few assumptions and estimates are required. 
    The initial (a priori) vertical profile of a trace gas is assumed or estimated via a CTM, while its scattering and radiative properties are calculated at prescribed altitudes using an RTM. 
    These properties are combined to determine the AMF.
    Different models are used for different satellite AMFs.
    For example GOME-2 products %(\url{http://atmos.caf.dlr.de/gome/product_hcho.html}), 
    use the LIDORT RTM with IMAGESv2 CTM, and OMI products use LIDORT combined with the GEOS-Chem CTM \parencite{Chance2002, Abad2015}.
    \textcite{Lamsal2014} recommends that when comparing satellite data to models, the AMF should first be recalculated using the model as an a priori.
    This is in order to remove any a priori bias between model and satellite columns.
    For this reason the OMHCHO AMF are recalculated in this thesis using GEOS-Chem to create the a priori profiles and LIDORT to calculate scattering weights.
    
    \label{Model:Meas:sat:LIDORT}
    %http://www.rtslidort.com/about_publications.html
    LIDORT is a model of LInearized Discrete Ordinate Radiative Transfer, used to determine backscatter intensities and weighting functions at arbitrary elevation angles \parencite{Spurr2001}.
    The model solves radiative transfer equations and can be used to determine various atmospheric column measurement attributes such as optical depth, ring effects, and scattering.
    These radiative properties (or at least estimates thereof) are required when measuring trace gases in the atmosphere through a long path such as seen by satellites \parencite[e.g.,][]{Palmer2001,Martin2002a,DeSmedt2015,Abad2015}.
    
    % averaging kernel relation to amf, why amfs are important when comparing data
    Related to the AMF is the averaging kernel (AK), which is used to handle instrument measurements that are sensitive to gas concentrations at different altitudes through the atmosphere.
    DOAS methods can be heavily influenced by the initial estimates of a trace gas profile (the a priori) that is often produced by modelling, so when comparing models of these trace gases to satellite measurements extra care needs to be taken to avoid introducing bias from differing a priori assumptions.
    One way to remove these a priori influences is through the satellite AK (or AMF), which takes into account the vertical profile of the modelled trace gas and instrument sensitivity to the trace gas \parencite{Eskes2003, Palmer2001}.
    This process is called deconvolution ($\Omega = AK \times VC_{satellite} + (I - AK) \times  VC_{a priori}$) of the AK of the satellite instrument.
    The AK represents sensitivities to each species at multiple altitudes through the atmosphere and in the case of OMI, can be approximated from the scattering weights ($\omega(z)$) function as follows:
    \begin{equation} \label{ch_HCHO:eqn:AKfromw}
    AK(z) = \frac{\omega(z)}{AMF}
    \end{equation}
    \parencite{Abad2015}.
    %This is an approximation for the OMI product, which does not include the AK but does include the $\omega$ and AMF, as explained in \textcite{Abad2015}.
    
    
  
  \subsection{Uncertainty}
  
    The main sources of error in satellite retrievals of HCHO are instrument detection sensitivities and calculation of the air mass factor (AMF) that converts slanted light path concentrations into a vertical profile \parencite{Millet2006}.
    % amf agreements between groups, but sensitive to a priori
    Calculations of the AMF performed by different groups tend to agree fairly well, as long as the a priori and ancillary data are similar.
    Large differences can occur depending on the a priori vertical profile, trace gas concentrations, and cloud properties \parencite{Lorente2017}.
    Choice of radiative transfer model and interpolation operations have a relatively small affect compared to the assumed state of the atmosphere, with high structural uncertainty introduced at this stage of AMF calculation \parencite{Lorente2017}.
    
    Uncertainty in the OMI satellite instrument is calculated by the Smithsonian Astrophysical Observatory (SAO) group using the uncertainty in backscattered radiation retrievals \parencite{Abad2015, Abad2016}.
    Uncertainty in a single pixel for OMHCHO is roughly the same magnitude as HCHO background levels.
    Each pixel has $\sim 2 \times 10^{14}$\moleccm ~uncertainty, which is five times higher than GOME.
    However, there are $\sim 100-200 $ times as many measurements allowing a greater reduction of uncertainty with averaging.
    This is due to the smaller footprint and better temporal resolution of OMI \parencite{Chance2002,Millet2008}.
    % Uncertainty in OMI pixels measurements
    The finer nadir resolution of OMI (13 by 24~km${^2}$) compared to other satellites also reduces cloud influence \parencite{Millet2006, Millet2008}.
    The top row in Figure \ref{Model:Meas:sat:fig_OMI_uncertainty} shows OMI HCHO columns binned to at \highhr ~longitude by latitude averaged over one day and one month (with and without filtering).
    Row two shows uncertainty of the satellite data after averaging.
    It is clear that one day of satellite data is too uncertain when binned at \highhr ~horizontal resolution; however, after a month (with or without filtering) the uncertainties become manageable.
    If we assume the uncertainty is random error, and not bias introduced through calculation techniques, then we are able to reduce the uncertainty through averaging.
    %High resolution low detection limit estimates can be built up using ``oversampling'', which averages satellite measurements over time \parencite[e.g.,][]{Zhu2014}.
    Uncertainty in satellite recalculations related to estimating isoprene emissions is analysed in Section \ref{BioIsop:uncertainty:satellite}.
    %A good example can be seen in \textcite{Zhu2014} where 0.2\degr ~by 0.2\degr ~resolution with high enough sensitivity to see anthropogenic HCHO is acheived with three summers worth of satellite data.
    
    % Figure from tests.py
    \mypic{Figures/OMI_link/Uncertainty_OMI_200501_final.png}{
      Top row shows \highhr ~binned OMHCHO columns with one day, one month, and one month with non-biogenic masking applied from left to right respectively. 
      Bottom row shows the uncertainty for each grid square after averaging.
    }{\label{Model:Meas:sat:fig_OMI_uncertainty}}
    
    
    
    % Row anomaly
    Recently \textcite{Schenkeveld2017} analysed the instrument performance over the life time (approximately 2004 to 2015) of OMI and found irradiance degradation of 3-8\%, changed radiances of 1-2\%, and a stable wavelength calibration within 0.005-0.020~nm.
    These changes are measured excluding the row anomaly effect, which has been relatively stable since 2011, although it is still growing and remains the most serious concern.
    Their analysis concludes that the satellite data is still of high quality and will deliver useful information for 5-10 more years, with radiance only changing by 1-2\% outside of row anomaly impacted areas.
    An analysis of the row anomaly by \textcite{Huang2018} states that measurements remain suitable for scientific use, with recommendation for further evaluation.
    % they were talking about ozone columns...
    The row anomaly began in June 2007, with some cross-track rows seemingly blocked. 
    The most likely cause is some instrument insulation partially obscuring the radiance port \parencite{Schenkeveld2017}.
    In this thesis pixels potentially affected by the row anomaly are removed prior to calculation or analysis.
    
    %% BACKGROUND MEASUREMENTS
    In satellite HCHO products, concentrations over the remote Pacific ocean are sometimes used to analyse faulty instrument readings.
    This is due to the expected invariance of HCHO over this region.
    For instance GOME (an instrument that measures trace gases onboard the ERS-2) corrects for an instrument artefact using modelled HCHO over the remote Pacific \parencite{Shim2005}.
    OMI HCHO products use a similar technique to account for sensor plate drift and changing bromine sensitivity \parencite{Abad2015}.
    In this thesis a background uncertainty estimation is performed based on differences between satellite measurements and monthly averaged modelled HCHO over the remote Pacific ocean (see Section \ref{BioIsop:uncertainty:satellite}).
    %Another method of calculating the uncertainty is used by the Belgian Institute for Space Aeronomy (BIRA) group, who determine uncertainty from the standard deviation of HCHO over the remote Pacific ocean \parencite{DeSmedt2012, DeSmedt2015}.
    
    %% LOW BIAS IN SATELLITE HCHO EXAMPLES
    %Potential Bias 
    The OMI dataset may suffer from bias; however, in order to determine any bias an independent dataset is required, and this does not exist over Australia.
    In this thesis, HCHO columns (pixels) with cloud fractions over 40\% are filtered as done in \textcite{Palmer2001}, which may introduce a clear-sky bias to any monthly averages.
    This is due to HCHO being lower on unrecorded cloudy days.
    This bias has been measured as a 13\% positive monthly mean bias \parencite{Palmer2001, Surl2018}.
    %Generally OMI is seen to suffer from a low bias of up to $40\%$ depending on region, local concentrations, and other factors \parencite[e.g.,][]{Barkley2013,DeSmedt2015,Zhu2016}.
    For many places the tropospheric column HCHO measured by satellite is biased low.
    %TODO: JENNYNOTE: UPDATE BIAS RANGES TO JUST OMI
    \textcite{Zhu2016} examine six available datasets and show a bias of 20 - 51\% over southeast USA when compared against a campaign of aircraft observations (SEAC$^4$RS).
    \textcite{DeSmedt2015} also found OMI and GOME2 observations were 20 - 40\% lower than ground based vertical profiles, and \textcite{Barkley2013} determine OMI to be 37\% low compared with aircraft measurements over Guyana.
    %These bias can be corrected by improving the assumed a priori HCHO profiles that are used to calculate the AMFs of the satellite columns.
    \textcite{Millet2006} examine OMI HCHO columns over North America and determine overall uncertainty to be 40\%, with most of this coming from cloud interference.
    \textcite{Millet2008} shows that there also exists some latitude based bias, as well as a systematic offset between the OMI and GOME instruments.
    This does not appear to be due to the different overpass times of the two instruments.
    

    %A full analysis of the AMF uncertainty in OMI measurements, as well as the structural uncertainty (between different systems of calculations applied to the same data) is performed by \textcite{Lorente2017}.
    AMF calculation often dominates the total uncertainty in satellite retrievals, especially in polluted regions \parencite{Lorente2017}.
    In scenarios where HCHO is enhanced in the lower troposphere, AMF calculation is the largest uncertainty in satellite measurements.
    In polluted environments the structural uncertainty is estimated at 42~\%, or 31~\% over unpolluted environments \parencite{Lorente2017}.
    Another impact often not included in uncertainty calculations is the structural uncertainty of retrieval methods.
    The structural uncertainty of AMF calculation approaches used by different retrieval groups comes from how the AMF is calculated, rather than uncertainty in the calculation components.
    The importance of a priori and ancillary data (such as surface albedo and cloud top height) sharply affects the structural uncertainty \parencite{Lorente2017}.
    In this thesis the AMF uncertainty is assumed to be $30\%$, as concentrations of HCHO and pollution are relatively low over Australia.
    
    % UP TO HERE JENNY
    % JENNY TODO: Need some context for why this section is here. At first glance it seems odd because it is in the middle of the satellite stuff. But I think it is actually here because you need it for the AMF calculation, right? Explain that at the outset. Another option would be to rearrange so you discuss GEOS-Chem first, then move onto OMI HCHO and then the details of that...
\section{GEOS-Chem}
  \label{Model:GC}

  \subsection{Overview}
    % Geos chem is a box model with chemistry and meteorology
    GEOS-Chem is a global, Eulerian CTM (see Section \ref{LR:Models:ctm}) with a state of the science chemical mechanism, and transport driven by meteorological input from the Goddard Earth Observing System (GEOS) of the NASA Global Modeling and Assimilation Office (GMAO).
    Chemistry, transport, and meteorology are simulated at 15 minute time steps within a global set of 3-D boxes.
    Emissions are either prescribed by inventories (e.g., fire emissions are prescribed from the global fire emissions database GFED4) or modelled (e.g., biogenic emissions are modelled using MEGAN).
    % It uses GEOS, from GMAO
    
    % This thesis uses 10.01 at 2x2.5 model output
    GEOS-Chem simulates more than 100 chemical species within the atmosphere, from the earth's surface up to 0.01~hPa.
    Modelled concentrations can be used in combination with measurements to give a verifiable estimate of atmospheric gases and aerosols in places and times where no measurements exist.
    It was developed, and is maintained, by Harvard University staff as well as users and researchers worldwide.
    In this thesis I use version 10.01 of GEOS-Chem, which outputs up to 66 chemical species (tracers) in the standard run, at \lowhr ~horizontal resolution, with 47 levels up to the top of the atmosphere (TOA at 0.01~hPa).
    
    
    Global CTMs are often run using one or several emission models (or the output from them) to determine boundary conditions.
    Some of the inventories used by GEOS-Chem are described here.
    Meteorological fields are taken from NASA's GEOS-5 dataset (0.5\degr ~x 0.666\degr) \parencite{Chen2009}, which exists up to April 2013.
    GEOS-5 meteorological fields are used as the boundary conditions driving transport.
    % Emissions models used for boundary conditions
    Fire emissions come from the global fire emissions database (GFED4) product \parencite{Giglio2013}. 
    Anthropogenic VOC emissions come from the Emission Database for Global Atmospheric Research (EDGAR) inventory, while biogenic VOC emissions are simulated using the MEGAN model (see Section \ref{Model:GC:MEGAN}).
    MEGAN is used to determine biogenic emissions for our default GEOS-Chem simulation.
    %The estimated biogenic VOC emissions are important are used extensively within this thesis.
    
  
  \subsection{Installing and running GEOS-Chem}
    \label{Model:GC:running}
    GEOS-Chem instructions for download, compilation, and running can be found in the user guide provided by Harvard: \url{http://acmg.seas.harvard.edu/geos/doc/man/}.
    In order to build and run GEOS-Chem a high-speed computing system is optimal, as globally gridded chemical calculations can take a long time to perform (for us $\sim 70$ computation hours per month).
    I installed GEOS-Chem onto a suitably configured workspace on the Raijin supercomputer at the National Computational Infrastructure (NCI, \url{http://nci.org.au/}).
    This workspace included access to compilers and libraries that are needed to build the Fortran based GEOS-Chem source code, and IDL, Python, and various editors and scripting languages to read, run, edit, and analyse both GEOS-Chem and its output.
    %After downloading GEOS-Chem, the code can be compiled with different options for resolution and chemical mechanisms.
  
  \subsection{Chemical Mechanism}
    \label{Model:GC:mechanisms}
    A chemical mechanism is a closed system of chemical reactions and their associated rate constants.
    Chemical reactions are represented by systems of differential equations to be solved for each gridbox in GEOS-Chem.
    Simplifications are required due to the large number of reactions that occur in the atmosphere, and the coupled and stiff nature of these reactions that slow down computation of the solutions \parencite{BrasseurJacob2017}.
    Stiffness in chemical systems of differential equations is due to the massively differing reaction time scales - for instance hydroxyl radicals react within seconds while methane has an atmospheric lifetime of 8-10 years \parencite{Wuebbles2002}. 
    
    Some of the important reactions involving isoprene are reproduced here, including reaction rates ($k$) in the form $ k = A \exp{-E/RT}$, where T is temperature, E is activation energy, and R is the gas constant, $A$, $E$, and $R$ are predefined for each reaction.
    The following set of equations (\ref{Model:GC:mechanisms:eqn_mechanisms}) lists the main isoprene and sundry reactions, with terms defined earlier in Table \ref{Model:GC:mechanisms:tab_species}.
    LISOPOH is added in order to allow the model to keep track of how much isoprene is oxidised by OH.
    The mechanisms used in version 10.01 (and its history) are described online at \url{http://wiki.seas.harvard.edu/geos-chem/index.php/NOx-Ox-HC-Aer-Br_chemistry_mechanism}.
    
    %globchem.dat in the run directory.
    % species in http://wiki.seas.harvard.edu/geos-chem/index.php/Species_in_GEOS-Chem#Aerosol-only
    %Isoprene is ISOP. 
    %% LISOPOH is isop lost to OH reactions
    %% RIO2 is the isoprene peroxy radical ISOPOO
    %% INO2 is a class of products formed from Isop and NO3
    %% ISOPNB is ISOPN-beta
    %% ISOPNBO2 is ISOPNB+OH products
    %% ISNP is an ISOPN 
    %% PRPE are >C3 alkenes (C3H6, ...)
    % first line shows reaction rate:
    %     A nnn <A> xxxExx <-ER>  xxx  xx  xx
    %  k  =    <A>  exp {<-ER>/T}
    % T is Temperature
    
    \begin{subequations}
    \label{Model:GC:mechanisms:eqn_mechanisms}
    \begin{align} %\begin{split}
    \ce{
      % A 587 Isop oxidation by OH
      ISOP + OH ->[3.1*10^{-11} \exp{(350/T})] 
        & RIO2 ( + LISOPOH) \label{mech1}  \\ 
      % A606 Isoprene ozonolysis
      ISOP + O3 ->[1*10^{-14} \exp{(-1970/T})] 
        & 0.244MVK + .325MACR + \highlight{0.845HCHO} \notag \\
        & + .11H2O2 + .522CO + .204HCOOH \notag \\
        & + .199MCO3 +.026HO2 + .27OH \notag \\
        & + .128PRPE + .051MO2  \label{mech2} \\ 
      % A621 Isoprene NO3 oxidation
      ISOP + NO3 ->[3.3*10^{-12} \exp{(-450/T)}] 
        & INO2 \label{mech3} \\
      % products from ISOPOO + NO
      RIO2 ->[4.07*10^{8} \exp{(-7694/T)}] 
        & 2HO2 + \highlight{HCHO} + .5MGLY + .5GLYC \notag \\
        & + .5GLYX + .5HAC + OH \label{mech4} \\
      RIO2 + NO ->[2.7*10^{-12} \exp{(350/T)}] 
        & .883NO2 + .783HO2 + \highlight{.66HCHO} \notag \\
        & + .4MVK + .26MACR + .07ISOPND \notag \\
        & + .123HC5 + .1DIBOO + .047ISOPNB \label{mech5} \\
      RIO2 + HO2 ->[2.91*10^{-13} \exp{(1300/T)}] 
        & .88RIP + .12OH + .047 MACR \notag \\
        & + .073MVK + .12HO2 + \highlight{.12HCHO} \label{mech6} \\
      RIO2 + MO2 ->[8.37*10^{-14}] 
        & 1.1HO2 + \highlight{1.22HCHO} + .28MVK  \notag\\
        & + .18MACR + .3HC5 + .24MOH + .24ROH \label{mech7} \\
      RIO2 + RIO2 ->[1.54*10^{-13}] 
        & 1.28HO2 + \highlight{.92HCHO} + .56MVK + \notag \\
        & .36MACR + .48ROH + .5HC5 \label{mech8} \\
      RIO2 + MCO3 ->[1.68*10^{-12} \exp{(500/T)}] 
        & .887HO2 + \highlight{.747HCHO} + .453MVK  \notag  \\
        & +.294MACR + .14HC5 + .113DIBOO  \notag  \\
        & + CO2 + MO2 \label{mech9}  \\
      RIO2 + MCO3 ->[1.87*10^{-13} \exp{(500/T)}] 
        & MEK + ACTA  \label{mech10} 
    }
    %     % k1=2.7*10^{-12} \exp{350/T}
    %     % k2=4.07*10^{8} \exp{-7694/T}
    %      
    %\end{split}
    \end{align} \end{subequations} 
    
    \begin{table}
      \caption{Species or classes from the GEOS-Chem mechanism.}
      \begin{tabular}{ c l }
        \textbf{Name} & \textbf{Definition} 
        \\ \hline
        ACTA    & Acetic acid: CH3C(O)OH \\
        DIBOO   & Dibble peroxy radical \\
        GLYC    & Glycoaldehyde: HOCH2CHO \\
        GLYX    & Glyoxal: CHOCHO \\
        HAC     & Hydroxyacetone: HOCH2C(O)CH3 \\
        INO2    &  RO2 from ISOP+NO3    \\  
        ISNP    & an isoprene nitrate \\
        ISOPNB  & $\beta$ isoprene nitrates     \\
        ISOPND  & $\delta$ isoprene nitrates    \\
        MACR    & Methacrolein: CH2=C(CH3)CHO \\
        MCO3    & Peroxyacetyl radical: CH3C(O)OO \\
        MEK     & Methyl ethyl ketone: RC(O)R \\
        MGLY    & Methylglyoxyal: CH3COCHO \\
        MO2     & Methylperoxy radical: CH3O2 \\
        MOH     & Methanol: CH3OH \\
        MVK     & Methylvinylketone: CH2=CHC(=O)CH3 \\
        PRPE    & >C3 alkenes: C3H6, ... \\
        RIO2    & isoprene peroxy radical: \roo   \\
        ROH     & >C2 alcohols \\
      \end{tabular}
      \label{Model:GC:mechanisms:tab_species}
    \end{table}
    
  \subsection{GEOS-Chem isoprene}
    \label{Model:GC:Isop}
    
    % outline of papers making up isoprene mechanisms
    The isoprene reactions simulated by GEOS-Chem were originally based on \textcite{Horowitz1998}.
    %This involved simulating NO$_X$, O$_3$, and NMHC chemistry in the troposphere at continental scale in three dimensions, with detailed (for 1998) NMHC chemistry with isoprene reactions and products.
    The mechanism was subsequently updated by \textcite{Mao2013}, who changed the isoprene nitrate yields and added products based on \textcite{Paulot2009a, Paulot2009b}.
    %They used anion chemical ionization mass spectrometry (CIMS) to determine products of isoprene photooxidation.
    They used the yields and reactions of various positional isomers of isoprene nitrates, and their oxidation products. %, within GEOS-Chem's suite of chemical mechanisms.
    Further mechanistic properties, like isomerisation rates, are based on results from four publications: \textcite{Peeters2009, Peeters2010, Crounse2011, Crounse2012}.    
    % Peeters looks at OH adduction and subsequent pathways to reform OH with ~ 80% efficiency overall
    % hpald: hydroperoxyenal, hydroperoxyaldehyde, hydroperoxy-methyl-butenals
    
    In a chamber with clean air and high NO concentrations, isoprene photooxidation is initially driven by OH addition, followed by NO$_X$ chemistry (150~min - 600~min), and finally HO$_X$ dominated chemistry.
    Formation of isoprene nitrates (ISOPN) affect ozone levels through NO$_X$ sequestration, the yields and fate of these nitrates was analysed in \textcite{Paulot2009a}.
    %% ISOPN can be oxidised (by OH) to form nitrated organic products \parencite{Paulot2009a}.
    Prior to 2012, oxidation chamber studies were performed in high NO or HO$_2$ concentrations, giving peroxy lifetimes of less than 0.1~s \parencite{Crounse2012, Wolfe2012}.
    In most environments NO and HO$_2$ concentrations are not so high, GEOS-Chem uses production rates for different NO concentrations and peroxy radical lifetimes determined by \textcite{Crounse2012}.
    
    \subsubsection{Oxidation}
    
      % roo production, rate and cites
      \textcite{Crounse2011} examined the isomerisations associated with the oxidation of isoprene to six different isomers of \roo formed in the presence of oxygen.
      The primary oxidation pathway of isoprene is reaction with the OH radical ($ ISOP + OH \to RIO_2 $) shown in Equation \ref{mech1}.
      Isoprene undergoes OH addition at the 1 and 4 positions, becoming $\beta$ (71\%) or $\delta$ (29\%) \roo, although these are not distinguished in the GEOS-Chem mechanism. 
      Secondary oxidative pathways are from ozonolysis (Equation \ref{mech2} and reaction with NO$_3$ (Equation \ref{mech3}).
      These pathways are much slower but can be important in particular scenarios such as inside a pollution plume or at night when NO$_3$ radicals can build up.
    
      Following isoprene oxidation, seven potential \roo reactions (RIO2 in Equation \ref{Model:GC:mechanisms:eqn_mechanisms}) can occur depending on reactant concentrations and local temperature.
      \textcite{Crounse2011} determined rates and uncertainties involved in these reactions, and studied the rate of formation of C$_5$-hydroperoxyenals (C5-HPALD) by isomerisation.
      Reactions \ref{mech4} - \ref{mech10} compete to determine the fate of \roo, which is often grouped into high or low NO$_X$ concentration pathways.
      
    
    \subsubsection{Nitrogen oxide impacts}
    
      GEOS-Chem reactions account for high and low NO$_X$ scenarios as determined in \textcite{Mao2013}, based on \textcite{Paulot2009a}.
      High NO$_X$ has been defined as conditions where NO reactions are the main cause of losses for \roo (Reaction \ref{mech5}) \parencite{Palmer2003}.
      High NO$_X$ concentrations are roughly 1~ppb, while low NO$_X$ concentrations are around 0.1~ppb.
      
      % Low nox mechanism pathways
      In low NO$_X$ \roo losses occur from isomerisation or reaction with several potential compounds.
      NO$_X$ + HO$_2$ (Reaction \ref{mech6}) produces mostly hydroxy hydroperoxides (ISOPOOH), and some HCHO.
      ISOPOOH can be oxidised (by OH) to produce epoxydiols, recycling OH \parencite{Paulot2009b}. 
      %($70\%$ yield of hydroxy hydroperoxides, ISOPOOH), 
      Isomerisation of \roo (Reaction \ref{mech4}) largely produces HCHO, while recycling OH and producing HO$_2$.
      This isomerisation accounts for 1,5-H shifts producing MACR, MVK, HCHO, or 1,6-H shifts producing HPALDs.
      HPALDs can photolyse to regenerate OH and small VOCs \parencite{Crounse2011,Wolfe2012, Peeters2014}.
      Under low NO$_X$ conditions the expected production of HCHO, MVK, and MACR is 4.7\%, 7.3\%, and 12\% respectively.
      Refer to Section \ref{LR:VOCs:IsopCascade} for more information.
      Less frequently, \roo reacts with itself, MO$_2$, or MCO$_3$.
      
      % High nox chemistry
      Under high NO$_X$ conditions, the fate of \roo differs depending on how it was formed.
      The $\beta$-hydroxyl reacts with NO$_X$ and produces HCHO (66\%), methylvinylketone (40\%) (MVK), methacrolein (26\%), and $\beta$-hydroxyl nitrates (6.7\%) (ISOPNB).
      The $\delta$-hydroxyl reacts with NO to form $\delta$-hydroxyl nitrates (24\%) (ISOPND), and ISOPNB (6.7\%).
      These two pathways are combined in Reaction \ref{mech5}, which shows the production from \roo and NO.
      
      
    \subsubsection{OH}
    
      Until recently, models struggled to represent OH in the atmosphere, due to missing chemistry and issues in measurement techniques.
      Prior to \textcite{Mao2012}, measurements of OH in high VOC regions may have been up to double the real atmospheric OH levels, due to formation of OH inside measuring instruments.
      OH regeneration through photolysis of HPALDs in areas with high isoprene emissions are included from \textcite{Peeters2010}.
      Photolysis of photolabile peroxy-acid-aldehydes generates OH, improving model agreement with continental observations.
      OH and HPALD interactions are central to maintaining the OH levels in pristine and moderately polluted environments, which makes isoprene both a source and sink of OH \parencite{Peeters2010,Taraborrelli2012}.
      Isoprene chemistry in GEOS-Chem includes OH regeneration from oxidation of epoxydiols (not shown in Equation \ref{Model:GC:mechanisms:eqn_mechanisms}) and slow isomerisation of \roo (Equation \ref{mech4}) \parencite{Mao2013}.
      This regeneration dealt with the problem seen in older models where ISOPOOH production titrated OH, which was not backed up by measurements \parencite{Paulot2009b,Mao2013}.
      \textcite{Mao2013} showed that drastically lowering (by a factor of 50) the rate constant for \roo isomerisation lead to better organic nitrate agreements with measurements.% (ICARTT).
      These chemical updates led to more accurate modelling of OH concentrations, especially in low NO$_X$ conditions common in remote forests.
      The updates to isoprene chemistry by \textcite{Mao2013}, and those shown in \textcite{Crounse2011,Crounse2012} are the last before GEOS-Chem version 11.

  
  \subsection{Emissions from MEGAN}
    \label{Model:GC:MEGAN}
    
    % General description of MEGAN?
    
    GEOS-Chem runs the Model of Emissions of Gases and Aerosols from Nature (MEGAN) to determine biogenic emissions globally.
    GEOS-Chem V10.01 uses MEGAN V2.1 with biogenic emissions from \textcite{Guenther2012}.
    MEGAN is itself a global model with resolution of around 1~km, which uses globally estimated or measured %(including remote sensed satellite data) 
    inventories of parameters to estimate emissions.
    MEGAN uses leaf area index (LAI), global meteorological data, plant functional types (PFT), and photosynthetic photon flux density to simulate terrestrial isoprene emissions \parencite{Kefauver2014}.
    The schematic for MEGAN, from \url{http://lar.wsu.edu/megan/}, is shown in Figure \ref{Model:GC:Isop:MEGAN:fig_megan_schematic}.
    %\textcite{Megan_Website}
    
    \begin{figure}
      \includegraphics[width=\textwidth]{Figures/MEGANmodel_img.jpg}
      \caption{MEGAN schematic, from \url{http://lar.wsu.edu/megan/}}
      %\textcite{Megan_Website}}
      \label{Model:GC:Isop:MEGAN:fig_megan_schematic}
    \end{figure}
    
    MEGAN was developed as a replacement for two earlier canopy-environment emission models (the biogenic emission inventory system, and the global emissions initiative), initially including a simple canopy radiative transfer model that parameterised sun-lit and shaded conditions through a canopy.
    Early models did not account for abiotic stresses, such as drought, prior rainfall and land use changes. 
    These stresses influence species-specific emissions by more than an order of magnitude \parencite{Niinemets1999}.
    Isoprene emissions were based on temperature, leaf area, and light, but have since been updated to include leaf age activity \parencite{Guenther2000}, and a leaf energy balance model \parencite{Guenther2006} in MEGANv2.0.
    This update included a parameter for soil moisture, to account for drought conditions; however,this parameter is currently (as of version 2.1) not applied to isoprene \parencite{Sindelarova2014}.
    %Soil moisture effects on isoprene emission are important, and can affect estimates.
    Instructions to run version 2.1 are available at \url{http://lar.wsu.edu/megan/docs/MEGAN2.1_User_GuideWSU.pdf}.
    %, and a version using the Community Land Model (CLM) is available at \url{http://www.cesm.ucar.edu}.
    %It uses meteorological fields from the Weather Research and Forecasting (WRF) modelling system.
    Version 2.1 (updated from 2.0 \parencite{Guenther2006}) includes 147 species, in 19 BVOC classes, which can be lumped together to provide appropriate output for mechanisms in various chemical models.
    GEOS-Chem uses an embedded version of MEGAN 2.1.
    
    GEOS-Chem computes some emissions using predefined emission factor ($\epsilon$) maps from MEGAN source code, and others using PFT maps and associated emission factors.
    For isoprene (the focus in this thesis) MEGAN calculates emissions online, using local meteorological conditions.
    % I think this is the case as MEGAN extension is set to ON for isop in HEMCO_Config.rc
    Emissions $E$ of species $i$ are calculated using these EF for classes of plant types $j$ and associated grid-box coverage $\xi$ and an activity factor $\gamma$, which accounts for response to environmental conditions.
    The following equation is reproduced from \textcite{Guenther2012} showing how $E_i$ are determined in MEGAN: 
    %% PFT, LAI, and other parameters are used to generate emissions which represent quantities of a compound released to the atmosphere through an associated activity.
    \begin{eqnarray} 
      %\begin{split}
      \label{Model:GC:MEGAN:eqn_emission_factors}
      %% EF_{i,j} = 
      E_i = & \gamma_i \mathlarger{\sum}_j{\epsilon_{i,j} \chi_j}
      %\end{split} 
    \end{eqnarray}
    For example: isoprene emissions are tied to sunshine, temperature, plant emission strength ($\epsilon$), emitting species coverage, etc., for each grid box.
    
  
  \subsection{Nitrogen oxides}
  \label{Model:GC:NOx}
    % How NOX affects us and how GEOS-Chem incorporates it
    NO$_X$ concentrations affect atmospheric oxidative capacity, which changes many factors important in estimating isoprene emissions including isoprene to HCHO yield, isoprene lifetime, and isoprene oxidation pathways.
    In GEOS-Chem, NO$_X$ concentrations are regulated by O$_3$, VOC, HO$_X$, Bromine and aerosols reactions.
    NO$_X$ emissions from nearby agriculture, power generation, and combustion transport drive enhancements in populated areas.
    The largest source (70\%) of NO$_x$ emissions comes from agriculture where nitrogenous fertilisers are in use \parencite{WorldBankNitrogenEmissionsPage}. %(\url{https://data.worldbank.org/indicator/EN.ATM.NOXE.AG.ZS}).
    In GEOS-Chem, anthropogenic emissions are taken from the Emissions Database for Global Atmospheric Research (EDGAR), for more details visit \url{http://wiki.seas.harvard.edu/geos-chem/index.php/EDGAR_v4.2_anthropogenic_emissions}.
    % these are mostly from agriculture (70\%) and energy production (9\%) (\url{https://data.worldbank.org/indicator/EN.ATM.NOXE.AG.ZS}).
    % previous versions had $\sim 21$~Tg N a$^{-1}$ from fossil fuel combustion
    Other NO$_X$ emissions arise from sources including soil emissions ($\sim 10$~Tg N a$^{-1}$), and lightning.
    Soil emissions are layed out in \textcite{Hudson2012}, parameterised using biome specific emission factors, and an explicit fertiliser dataset.
    Lightning based NO$_X$ production is created based on inventories of lightning flash rates scaled within GEOS-Chem.
    Conversion to nitric acid (HNO$_3$) followed by deposition is the primary NO$_x$ removal mechanism \parencite{Delmas1997, Ayers2006}.
    
    % How GEOS-Chem compares to satellite NO
    If GEOS-Chem is misrepresenting NO$_X$ then yields and the effects of transport may be incorrectly accounted for.
    In order to determine the accuracy of GEOS-Chem simulated NO$_X$ over Australia, modelled NO$_2$ amounts are compared to satellite data (where available) for 2005.
    Figure \ref{Model:analysis:NOx:fig_NO_maps_2005} shows tropospheric NO$_2$ columns from GEOS-Chem, OMNO2d, and their differences over four seasons in 2005.
    % Geos chem vs tropno2 from omno2d
    Simulated GEOS-Chem tropospheric NO$_2$ columns averaged from 13:00-14:00 local time (LT) are compared against OMNO2d data that are averaged into seasonal \lowhr ~bins. %(Sec. \ref{Model:datasets:OMNO2d}). 
    Both data sets show Sydney and Melbourne as NO$_2$ hotspots; however, Sydney itself is underestimated by GEOS-Chem throughout the year, likely due to low or poorly aligned emission maps (shown in Figure \ref{Model:analysis:NOx:fig_NO_emiss_maps_2005}).
    Over much of the country GEOS-Chem overestimates NO$_2$ by 10-60\%, except in the northern areas where up to 50\% underestimation occurs in Summer.
    An underestimation in southern NSW can also be seen in the later half of the year.
    Notably GEOS-Chem anthropogenic NO does not have any seasonality (Figure \ref{Model:analysis:NOx:fig_NO_emiss_maps_2005}), while soil shows clear differences in emission hotspots between summer and winter.
      
    % Figure from GC_tests.py NO_comparisons
    \mypic{Figures/OMI_link/GC/NO_maps_2005.png}{%
      Left to right columns show tropospheric NO$_2$ columns ($\Omega_{NO_2}$; molec cm$^{-2}$) from GEOS-Chem (GC), OMNO2d (OMI), and the differences.
      Each row shows one season from 2005, the left two columns use the left colour scale, while the third column uses the right colour scale.
    }{\label{Model:analysis:NOx:fig_NO_maps_2005}}
    
    % second Figure from GC_tests.py NO_comparisons
    \mypic{Figures/OMI_link/GC/NO_emiss_maps_2005.png}{%
      Left and right columns show anthropogenic and soil emissions of NO respectively from GEOS-Chem in molec cm$^{-2}$ s$^{-1}$.
      Each row shows one season from 2005.
      Anthropogenic and soil emissions use a logarithmic and linear colour scale respectively.
    }{\label{Model:analysis:NOx:fig_NO_emiss_maps_2005}}
    
    Figure \ref{Model:analysis:NOx:fig_NO_regressions_2005} shows scatter plots with one data point for each land square over Australia, for each season, coloured by total NO emissions.
    The correlation between model and satellite NO$_2$ columns is reasonable throughout the year over Australia, with a general positive bias throughout the year in modelled amounts.
    A comparison between the bias (GEOS-Chem - OMNO2d) with anthropogenic or soil emissions (columns 2 and 3 respectively) shows no real correlation, suggesting the bias is not driven by either anthropogenic or soil NO emissions in any season.
    Without a clear link between emissions and biases, alterations would be too subjective, and a change to NO$_X$ chemistry is beyond the scope of this thesis.
    Additionally, high satellite NO$_2$ columns are filtered in later calculations that assume biogenic air masses as they suggest anthropogenic influence.
    The conclusion drawn is that modelled anthropogenic and soil NO emissions do not show sufficient evidence of biasing GEOS-Chem NO$_2$ columns away from satellite measurements over Australia.
    For this reason modelled NO emissions are not modified in model runs in this thesis.
    
    % Third figure from GC_tests.NO_comparisons
    \mypic{Figures/OMI_link/GC/NO_regressions_2005.png}{%
      The first column shows the scatter plots of tropospheric NO$_2$ columns between GEOS-Chem (y-axis) and OMNO2d (x-axis) at \lowhr.
      The reduced major axis linear regression is drawn in red, and the equation, regression coefficient (r), and number of grid squares (n) used is inlaid as a legend.
      The second and third columns show the scatter between emissions and the bias between GEOS-Chem and OMNO2d (GC-OMI), for anthropogenic and soil emissions respectively.
      These two columns share the far right axis; however,emissions are from anthropogenic and soil sources respectively.
      All scatter points are coloured by the sum of anthropogenic and soil NO emissions (from GEOS-chem), as per the colour bar shown at the bottom.
    }{\label{Model:analysis:NOx:fig_NO_regressions_2005}}
    
    %    This comparison is expanded, including against modelled emissions, and repeated for autumn (MAM), winter (JJA), and spring (SON) in figures \ref{Model:analysis:NOx:fig_GC_vs_OMI_anthro_Sum} to \ref{Model:analysis:NOx:fig_GC_vs_OMI_soil_Spr}.
    %    These show an analysis of GEOS-Chem NO emissions and their correlations with the bias between GEOS-Chem NO$_2$ mid-day columns and the OMNO2d product, averaged over each season in 2005.
    %    These figures show the visible biases are not driven by modelled emissions of NO.
    %    While the correlation between column NO$_2$ and emitted NO is clear, emissions do not appear to bias the model in either direction away from the satellite data.

    %    Figure \ref{Model:analysis:NOx:fig_GC_vs_OMNO2d_summer_2005} shows the direct comparison between these datasets averaged over the months of January and February, 2005.
    %    The top row shows (from left to right) GEOS-Chem NO$_2$, OMI NO$_2$ at \highhr, and OMI NO$_2$ at \lowhr.
    %    The bottom row shows the difference (absolute, and relative) between GEOS-Chem and OMI, as well as the reduced major axis (RMA) linear fit.
    %    The comparison is repeated for winter (JJA) of 2005 in Figure \ref{Model:analysis:NOx:fig_GC_vs_OMNO2d_winter_2005}.
    
    %    \begin{figure}
    %      % Figure from GC_tests.py GC_vs_OMNO2d, then modified in paint
    %      % Summer correlation
    %      \includegraphics[width=\textwidth]{Figures/OMI_link/GC/GC_vs_OMNO2_AUS_20050101-20050228.png}
    %      \caption{%
    %        Row 1 shows the tropospheric columns in molec cm$^{-2}$, GEOS-Chem, OMNO2d, and OMNO2d averaged onto the lower resolution of GEOS-Chem from left to right.
    %        Row 2 shows the correlations of GEOS-Chem (X axes) between daily anthropogenic emissions, and mid-day OMNO2d columns.
    %        Row 3 shows the differences with OMNO2d columns averaged into the lower resolution of GEOS-Chem.
    %      }
    %      \label{Model:analysis:NOx:fig_GC_vs_OMNO2d_summer_2005}
    %    \end{figure}
    %    \begin{figure}
    %      % Figure from GC_tests.py GC_vs_OMNO2d, then modified in paint
    %      % Winter correlation
    %      \includegraphics[width=\textwidth]{Figures/OMI_link/GC/GC_vs_OMNO2_AUS_20050601-20050831.png}
    %      \caption{%
    %        As Figure \ref{Model:analysis:NOx:fig_GC_vs_OMNO2d_summer_2005}, for winter 2005.
    %      }
    %      \label{Model:analysis:NOx:fig_GC_vs_OMNO2d_winter_2005}
    %    \end{figure}
    
%    % Figures from GC_tests.py GCe_vs_OMNO2d
%    % TODO update figures re Jenny's comments
%    \mypic{Figures/OMI_link/GC/GCanthro_vs_OMNO2_AUS_20050101-20050228.png}
%    {
%      Top row (left to right): GEOS-Chem NO$_2$ mid-day tropospheric columns, OMNO2d NO$_2$ columns, modelled anthropogenic NO emissions. 
%      Second row: absolute and relative difference between GEOS-Chem and OMI NO$_2$ data, and the correlation.
%      Third row: correlation between GEOS-Chem tropospheric column NO$_2$ and emitted NO, then between the model-satellite bias and the emissions.
%      All correlation plots are coloured by NO emission rates.
%    }
%    {\label{Model:analysis:NOx:fig_GC_vs_OMI_anthro_Sum}}
%    
%    \mypic{Figures/OMI_link/GC/GCanthro_vs_OMNO2_AUS_20050301-20050531.png}
%    {As Figure \ref{Model:analysis:NOx:fig_GC_vs_OMI_anthro_Sum}, for Autumn 2005.}
%    {\label{Model:analysis:NOx:fig_GC_vs_OMI_anthro_Aut}}
%    
%    \mypic{Figures/OMI_link/GC/GCanthro_vs_OMNO2_AUS_20050601-20050831.png}
%    {As Figure \ref{Model:analysis:NOx:fig_GC_vs_OMI_anthro_Sum}, for Winter 2005.}
%    {\label{Model:analysis:NOx:fig_GC_vs_OMI_anthro_Win}}
%    
%    \mypic{Figures/OMI_link/GC/GCanthro_vs_OMNO2_AUS_20050901-20051130.png}
%    {As Figure \ref{Model:analysis:NOx:fig_GC_vs_OMI_anthro_Sum}, for Spring 2005.}
%    {\label{Model:analysis:NOx:fig_GC_vs_OMI_anthro_Spr}}
%    
%    % Soil pictures
%    \mypic{Figures/OMI_link/GC/GCsoil_vs_OMNO2_AUS_20050101-20050228.png}
%    {As Figure \ref{Model:analysis:NOx:fig_GC_vs_OMI_anthro_Sum}, except anthropogenic NO emissions are replaced by soil NO emissions.}
%    {\label{Model:analysis:NOx:fig_GC_vs_OMI_soil_Sum}}
%    
%    \mypic{Figures/OMI_link/GC/GCsoil_vs_OMNO2_AUS_20050301-20050531.png}
%    {As Figure \ref{Model:analysis:NOx:fig_GC_vs_OMI_anthro_Sum}, for Autumn 2005, with soil NO emissions replacing anthropogenic NO emissions.}
%    {\label{Model:analysis:NOx:fig_GC_vs_OMI_soil_Aut}}
%    
%    \mypic{Figures/OMI_link/GC/GCanthro_vs_OMNO2_AUS_20050601-20050831.png}
%    {As Figure \ref{Model:analysis:NOx:fig_GC_vs_OMI_anthro_Sum}, for Winter 2005, with soil NO emissions replacing anthropogenic NO emissions.}
%    {\label{Model:analysis:NOx:fig_GC_vs_OMI_soil_Win}}
%    
%    \mypic{Figures/OMI_link/GC/GCanthro_vs_OMNO2_AUS_20050901-20051130.png}
%    {As Figure \ref{Model:analysis:NOx:fig_GC_vs_OMI_anthro_Sum}, for Spring 2005, with soil NO emissions replacing anthropogenic NO emissions.}
%    {\label{Model:analysis:NOx:fig_GC_vs_OMI_soil_Spr}}
        
  
  \subsection{GEOS-Chem simulations}
    \label{Model:GC:simulations}
    
    GEOS-Chem is run five times independently in this thesis, with different outputs from each simulation used to determine specific information. 
    The different output types are first described in Section \ref{Model:GC:simulations:outputs}.
    Following this is the list of model runs, including a summary of the run, outputs created, and a summary of how they are used (Section \ref{Model:GC:simulations:runs}).
    Finally a brief comparison between a subset of the runs is performed (Section \ref{Model:GC:simulations:comparison}).
    
    \subsubsection{GEOS-Chem outputs}
      \label{Model:GC:simulations:outputs}
      
      % Default time step and resolution of GC outputs
      GEOS-Chem in this thesis is run with a 15 minute time step for transport and 30 minutes for chemistry, at \lowhr ~horizontal resolution over 47 vertical levels.
      Output is the average of these time steps.
      Optionally one to many columns can be output at high temporally resolution.
      This has been used here to compare modelled ozone with ozonesonde profiles at three sonde release sites discussed in Chapter \ref{Ozone}.
      %For example: in this thesis, estimation of model isoprene to HCHO yields uses daily averaged HCHO columns compared against co-located isoprene emissions.
      

      \begin{description}
        \item[Midday output]%
          is output from averaging over a window of local time for each gridbox. 
          Output averaged between 13:00-14:00 local time is saved to match with Aura satellite measurements, as Aura overpasses at $\sim$1330 local time each day.
          Midday output is saved both for comparison with, and recalculation of, satellite measurements.
          %This has been performed by others \parencite[e.g.,][]{Jin2017}.
        \item[HEMCO diagnostics]:%
          the Harvard-NASA Emissions Component (HEMCO) deals with emissions inventories used in GEOS-Chem.
          % Local time offsets
          When working with globally gridded data, handling local time offsets becomes more important.
          The hourly averaged emissions of isoprene are saved using GMT, and an offset based on the longitude is used to retrieve values at local time.
          %This is done by setting up a latitude by longitude array that matches the horizontal resolution of the data, filling each grid box with its local time offset.
          This offset is determined as one hour per 15\degr ~(as 360\degr ~is 24 hours).
          %The retrieval of a daily local time global array is done by index matching the GMT+local time (modulo 24) with the desired hour on this grid over the 24 GMT hours.
          % Emissions are averaged over the hour for hemco diag output in biogenic run at least
        \item[Tracer averages]%
          are daily or monthly averaged gridbox concentrations. 
        \item[Time series]%
          are vertical profiles of ozone and other species and meteorological diagnostics saved at a temporal resolution of up to 15 minutes.
      \end{description}
    
    One issue with modelled outputs over Australia is that they are difficult to compare to in-situ measurements as they are averaged over a large horizontal space and vertical volume.
    
    \subsubsection{GEOS-Chem runs}
      \label{Model:GC:simulations:runs}
      
      The following list summarises each model run as well as enumerating the outputs (described above), and how the run is used in the thesis.
      \begin{description}
        \item[UCX]%
          The model was run using the Universal tropospheric-stratospheric Chemistry eXtension (UCX) mechanism with 72 vertical levels from the surface to the top of the atmosphere (TOA $\sim$0.1~hPa).
          UCX runs a chemistry mechanism with combined tropospheric and stratospheric reactions, with an increased number of stratospheric calculations performed online \parencite{Eastham2014}.
          \begin{enumerate}
            \item Outputs: Midday output, daily tracer averages, and time series over three stations.
            %\item The Midday output is used to check how shape factors for AMF recalculation are affected by vertical resolution and stratospheric chemistry. 
            %%TODO: Should I add a plot of how the AMF$_{GC}$ changes with/without the UCX? I'll remove this item if it's not worth it.
            \item This run shows what influence the stratospheric chemistry additions have over tropospheric isoprene, HCHO, and ozone concentrations.
            \item The daily tracer averages are used to determine ozone intrusion quantification (Section \ref{Ozone:fluxcalc}), and ozone concentration seasonality (Section \ref{Ozone:ModelComparison}).
            \item Time series outputs are compared against ozonesonde releases (Section \ref{Ozone:ModelComparison}) both over time and vertically.
          \end{enumerate}
        
        \item [Tropchem (standard)]%
          default settings for GEOS-Chem 10.01, using 47 vertical levels at \lowhr ~horizontal resolution.
          Additional midday output is created to allow AMF recalculation code to run on OMI satellite measurements (Section \ref{Model:omiRecalc:ppcode}).
          \begin{enumerate}
            \item Outputs: Midday output, daily tracer averages, and HEMCO diagnostics
            \item These are used in recalculation of the satellite AMF (Section \ref{Model:AMF}), and the modelled background HCHO over the remote Pacific that is used in the reference sector correction for OMI column retrievals (Section \ref{Model:omiRecalc:RSC}).
            Additional diagnostic outputs are used hard coded to allow AMF recalculation.
            \item Midday output is combined from two different runs in order to determine smearing (Section \ref{Model:filter:smearing})
            \item Total yearly ozone concentrations are compared before and after scaling isoprene emissions using the top-down estimate.
          \end{enumerate}
        
        \item [Tropchem (isoprene emissions halved)]%
          identical to standard tropchem except isoprene emissions are halved.
          \begin{enumerate}
            \item Outputs: Midday output, and monthly tracer averages 
            \item Check modelled ozone sensitivity to isoprene emissions (following section Figure \ref{Model:GC:simulations:comparison:fig_UCXvsTrop_O3}).
            \item Combined with standard run to determine model sensitivity transport (Section \ref{Model:filter:smearing})
          \end{enumerate}
        
        \item [Tropchem (biogenic emissions only)]%
          identical to standard tropchem except all non-biogenic emissions inventories are disabled.
          \begin{enumerate}
            \item Outputs: Midday output, and hourly biogenic emissions from MEGAN
            \item Midday and HEMCO outputs are used to determine isoprene to HCHO yield, after removing days with high smearing (Section \ref{BioIsop:method:slope})
            \item HEMCO diagnostics are compared against top-down estimations of isoprene emissions (Section \ref{BioIsop:results:emissions})
          \end{enumerate}
        
        \item [Tropchem (altered MEGAN scaling factor)]%
          Identical to standard tropchem except isoprene emissions are scaled to match multi-year monthly averaged top-down estimates.
          \begin{enumerate}
            \item Midday output, time series, daily averaged tracers
            \item Compare to campaign datasets after altering isoprene emissions (see Chapter \ref{BioIsop})
            \item Compare against satellite column HCHO as a sanity check on improving isoprene emissions
          \end{enumerate}
      \end{description}
    
    \subsubsection{UCX vs tropchem}
      \label{Model:GC:simulations:comparison}
      
      % brief description of both runs resolution
      Here we compare the model output with and without enabling the Universal tropospheric-stratospheric Chemistry eXtension (UCX).
      %From version 11 of GEOS-Chem, the UCX mechanism is enabled by default.
      Both runs use \lowhr ~latitude by longitude; however,the UCX mechanism is run with 72 vertical levels from the surface to the top of the atmosphere (TOA$\sim$0.1~hPa), while the standard (tropchem) run uses 47 levels.
      The reduced vertical levels are created by lumping together sets of the higher resolved levels from around 70~hPa to the top of the atmosphere.
      For both runs the input parameters such as MEGAN emissions and GEOS-5 meteorological fields are identical.
      
      % Figure showing differences between runs HCHO described and analysed
      Figure \ref{Model:GC:simulations:comparison:fig_UCXvsTrop_HCHO_surf_fullday} shows an example of surface HCHO amounts, averaged over Jan and Feb, 2007, with and without the UCX mechanism enabled.
      Surface HCHO (first model level; up to $\sim$100~m) does not differ much between runs.
      The differences do not exceed 3\% over Australia, and absolute differences are minor (note the scale in A-B).
      The major notable difference occurs over northern Africa, where HCHO is around 20\% lower in the UCX run.
      Additionally a slight ($<8\%$) decrease in HCHO over the oceans can be seen.
      The comparison is repeated using total columns (instead of surface values) in Figure \ref{Model:GC:simulations:comparison:fig_UCXvsTrop_HCHO_totcol_fullday}, showing that differences affecting HCHO between the model run are spread over the entire vertical column.
      
      \begin{figure}
        % These figures created in GC_test.py -> UCX_vs_trp...:
        \includegraphics[width=\textwidth]{Figures/OMI_link/GC/UCX_vs_trp_avgsurf_20070101-20070228_hcho_final.png}
        \caption{%
          Surface HCHO simulated by GEOS-Chem with UCX (A: top left), and without UCX (B: top right), along with their absolute and relative differences (bottom left, right respectively).
          Amounts are the average of all times between 1 Jan and 28 Feb 2007.
        }
        \label{Model:GC:simulations:comparison:fig_UCXvsTrop_HCHO_surf_fullday}
      \end{figure}
      
      \begin{figure}
        % These figures created in GC_test.py -> UCX_vs_trp...:
        \includegraphics[width=\textwidth]{Figures/OMI_link/GC/UCX_vs_trp_avgtotcol_20070101-20070228_hcho.png}
        \caption{%
          As Figure \ref{Model:GC:simulations:comparison:fig_UCXvsTrop_HCHO_surf_fullday} using total column amounts instead of surface concentrations.
        }
        \label{Model:GC:simulations:comparison:fig_UCXvsTrop_HCHO_totcol_fullday}
      \end{figure}
      

      % Differences between run isoprene amounts described and analysed here
      Figure \ref{Model:GC:simulations:comparison:fig_UCXvsTrop_isop_surf_fullday} shows the differences in surface isoprene concentrations over Australia, averaged over 1, Jan to 28, Feb, 2007.
      Here we start to see a higher relative difference in concentrations, although this is generally over the areas with less absolute concentrations. 
      Very little isoprene is seen away from the continents (4-5 orders of magnitude less), due to its short lifetime and  lack of oceanic sources.
      Generally isoprene is 0-30\% higher over mid to western Australia when the UCX mechanism is turned on; however,this increase is lower in the regions with high isoprene emissions (north-east to south-east coastline).
      This enhancement can be seen throughout the entire tropospheric column as shown by Figure \ref{Model:GC:simulations:comparison:fig_UCXvsTrop_isop_surf_fullday}.
      There is a greater effect in Africa and South America in the tropics, with high relative differences in many regions with low absolute amounts.
      
      \begin{figure}
        \includegraphics[width=\textwidth]{Figures/OMI_link/GC/UCX_vs_trp_avgsurf_20070101-20070228_isop_final.png}
        \caption{ %
          As Figure \ref{Model:GC:simulations:comparison:fig_UCXvsTrop_HCHO_surf_fullday}, except showing isoprene surface concentrations. 
        }      
        \label{Model:GC:simulations:comparison:fig_UCXvsTrop_isop_surf_fullday}
      \end{figure}
      \begin{figure}
        \includegraphics[width=\textwidth]{Figures/OMI_link/GC/UCX_vs_trp_avgtotcol_20070101-20070228_isop_final.png}
        \caption{ %
          As Figure \ref{Model:GC:simulations:comparison:fig_UCXvsTrop_isop_surf_fullday}, except showing isoprene total column amounts. 
        }      
        \label{Model:GC:simulations:comparison:fig_UCXvsTrop_isop_totcol_fullday}
      \end{figure}
      
      
      The difference in isoprene between UCX and tropchem is likely caused by differences in the modelled radiation reaching the troposphere due to differences in simulated ozone in the stratosphere.
      Figure \ref{Model:GC:simulations:comparison:fig_UCXvsTrop_O3} shows the total column ozone between UCX and non-UCX run.
      This shows that UCX ozone is lower everywhere except for a thin band just north and south of the equator.
      With higher stratospheric ozone levels, less radiation would reach the troposphere.
      This would slow photolysis limited reactions, such as the splitting of ozone that leads to OH production in the troposphere, in turn slowing the isoprene loss to OH reactions.
      % hydroxyl radical plot used to be mentioned here
      % is this correct??
      %Decreased O$_3$ could lead to lower OH and the other differences between the runs over Australia.
      
      % Figure from GC_tests->UCX_vs_...
      \mypic{Figures/OMI_link/GC/UCX_vs_trp_avgtotcol_20070101-20070228_O3.png}{%
        Total column ozone simulated by GEOS-Chem with UCX (A: top left), and without UCX (B: top right), along with their absolute and relative differences (bottom left, right respectively).
        Amounts are the average of all times between 1 Jan and 28 Feb 2007.
      }{\label{Model:GC:simulations:comparison:fig_UCXvsTrop_O3}}
      
      % Wrap up of UCX vs Tropchem differences and explanation of tropchem usage
      Overall the UCX mechanism and increased vertical resolution lead to slightly higher isoprene and HCHO, and slightly lower ozone over Australia.
      These differences are on the order of 0-10\% and as shown later (see Section \ref{BioIsop:uncertainty}) are minimal compared to other uncertainties in both AMF calculation and emissions estimation.
      Running UCX requires roughly twice the computation (and real) hours, and outside of Chapter \ref{Ozone} the normal tropchem runs are used.
      
      
      %% Ozone surface plot, not useful to explain column 
      %\begin{figure}
      %  % Figure from GC_tests.py -> ucx vs tropchem function
      %  \includegraphics[width=\textwidth]{Figures/OMI_link/GC/UCX_vs_trp_avgsurf_20070101-20070228_O3.png}
      %  \caption{%
      %    As Figure \ref{Model:GC:simulations:comparison:fig_UCXvsTrop_HCHO_surf_fullday}, except showing ozone. 
      %  }
      %  \label{Model:GC:simulations:comparison:fig_UCXvsTrop_O3_surf_fullday}
      %\end{figure}
      
      %      Figure \ref{Model:GC:simulations:comparison:fig_UCXvsTrop_OH_surf_midday} shows how OH at midday is changed between runs.
      %      This figure shows a marked striping which is due to how OH is handled in GEOS-Chem.
      %      The most notable difference is again over northern Africa into eastern Europe, with some small decrease in the UCX OH over Australia everywhere except the east coast.
      %% Gc_tests.py -> compare_tc_ucx(...)
      %\mypic{Figures/OMI_link/GC/UCX_vs_trp_middaysurf_20070101-20070228_OH.png}{%
      %  Midday (1300-1400~LT) surface OH concentrations averaged over Jan-Feb, 2007. 
      %  Absolute and relative differences are shown on the bottom row.}
      %{\label{Model:GC:simulations:comparison:fig_UCXvsTrop_OH_surf_midday}}
  

\section{Calculating new AMF}
  \label{Model:AMF}
  
  % What is AMF again?
  One of the uses of GEOS-Chem in this thesis is to update the a priori HCHO column used in satellite measurements.
  The a priori is used when deriving the air mass factor ($AMF$), which is needed to transform satellite slant columns ($SC$) into vertical columns ($\Omega$ in molecules cm$^{-2}$):
  The AMF is the ratio of $SC$ to the vertical column
  \begin{equation} \label{Model:AMF:eqn_AMFFrac}
    AMF=\frac{SC}{\Omega} = \frac{\tau_s}{\tau_v}
  \end{equation}
  with $\tau$ being the optical depth or thickness of the absorber through the slant (\textit{s}) or vertical (\textit{v}) path of light.
  
  
  The OMI instrument records spectra of light that enter the viewing lens on the Aura satellite.
  The spectra provide backscattered intensity ($I_B$) at various wavelengths (see Section \ref{Model:omhcho}), with the light source ($I_{B_0}$) being the sun. 
  Using the log of Beers law (Equation \ref{Model:omhcho:DOAS:eqn_beerslaw}) we get 
  $$ \tau_s = \ln{I_{B_0}} - \ln{I_B} $$
  which can be subbed into Equation \ref{Model:AMF:eqn_AMFFrac} to give an expression for the AMF that includes scattering:
  \begin{equation} \label{Model:AMF:eqn_amfscattering}
    AMF = \frac{\ln{I_{B_0}}-\ln{I_B}}{\tau_v}
  \end{equation}
  %{Model:omhcho:DOAS:eqn_backscatterextinction}
  %$\tau_s$ is the optical thickness of the absorber along the measured path between source and instrument.
  
  We use $\nabla I = I_B - I_{B_0}$ to represent the change in intensity due to the absorber. 
  For optically thin absorption, $\nabla I / I_B << 1$, and we can use:
  \begin{equation} \label{Model:AMF:eqn_AMFthin}
    AMF = \frac{\ln{ \left( 1 - \frac{\nabla I}{I_B} \right)} }{\tau_v} \approx \frac{ - \frac{\nabla I}{I_B} }{\tau_v}
  \end{equation}
  This is due to the logarithmic property $\ln \left(1-x \right) \approx -x$ for $x<<1$.
  $\nabla I$ can also be expressed as the integral of the absorption slices over optical depth increments: 
  \begin{eqnarray*}
    \nabla I &= \int_0^{\tau_v}{\frac{\partial I_B}{\partial \tau} \mathrm{d}\tau}
    \frac{\nabla I}{I_B} & = \int_0^{\tau_v}{\frac{\partial \ln{I_B}}{\partial \tau} \mathrm{d}\tau}
  \end{eqnarray*}
  which can be placed into Equation \ref{Model:AMF:eqn_AMFthin} leading to
  \begin{equation*}
    AMF \approx \frac{-1}{\tau_v} \int_0^{\tau_v}{\frac{\partial \ln{I_B}}{\partial \tau} \mathrm{d}\tau}
  \end{equation*}
  We can then convert $\text{d}\tau$ to our path using Equation \ref{Model:omhcho:DOAS:eqn_tau} leading to
  \begin{equation} \label{Model:AMF:eqn_AMFcross}
    AMF = \frac{-1}{\tau_v} \int_0^\infty {\frac{\partial \ln{I_B}}{\partial \tau} \alpha(z)\eta(z)\mathrm{d}z}
  \end{equation}
  where $\alpha(z)$ and $\eta(z)$ represent absorption cross section in m$^2$ molecule$^{-1}$, and number density in molecules m$^{-3}$ respectively. 
  This uses the attenuation cross section relationship to optical depth (see Section \ref{Model:omhcho:DOAS}).
  
  To represent an average cross section weighted by the absorbing species' vertical distribution, the effective cross section ($\alpha_e$) is used.
  This is to account for temperature and pressure dependence of $\alpha(z)$, and is defined as:
  \begin{align*}
    \alpha_e &= \frac{1}{\Omega_v} \int_0^\infty \alpha(z) \eta(z) \mathrm{d}z \\
     &= \frac{\tau_v}{\Omega_v}
  \end{align*}
  Then replacing the $\tau_v$ in Equation \ref{Model:AMF:eqn_AMFcross} we obtain:
  \begin{equation} \label{Model:AMF:eqn_AMFpreomega}
    AMF=-\int_0^\infty{ \frac{\partial \ln{I_B}}{\partial \tau} \frac{\alpha(z)}{\alpha_e} \frac{\eta(z)}{\Omega_v} \mathrm{d}z }
  \end{equation}
  Often the integrand of this AMF formula (Equation \ref{Model:AMF:eqn_AMFpreomega}) is broken apart into two defining terms: the scattering weights $\omega(z)$ and the shape factor $S(z)$, described here:
  \begin{description}
    \item[$\omega$] The scattering weights describing sensitivity of the backscattered spectrum to the abundance of an absorber at altitude z:
    \begin{equation} \label{Model:AMF:eqn_omega}
      \omega(z) = -\frac{1}{AMF_G} \frac{\alpha(z)}{\alpha_e} \frac{\partial \ln{I_B}}{\partial \tau}
    \end{equation}
    It is worth noting that in the OMI satellite product, the provided $\omega(z)$ term does not include the $\frac{1}{AMF_G}$ term and the calculations that follow therefor do not include this term when utilising the provided $\omega$.
    This is not noted in any of the papers that recalculate the AMF from the OMI product, due to them recalculating the $\omega$ term themselves with a radiative transfer model such as LIDORT.
    \item[$S$] the shape factor describes the profile of an absorber ($\eta(z)$) normalised by its total vertical column amount ($\Omega_v$):
    \begin{equation} \label{Model:AMF:eqn_shapefactor}
      S(z) = \frac{\eta(z)}{\Omega_v}
    \end{equation}
  \end{description}
  
  Plugging Equations \ref{Model:AMF:eqn_omega} and \ref{Model:AMF:eqn_shapefactor} into Equation \ref{Model:AMF:eqn_AMFpreomega} gives us:
  \begin{equation} \label{Model:AMF:eqn_AMF_amfgintwSdz}
    AMF = AMF_G\int_0^\infty{ \omega(z) S(z) \mathrm{d}z}
  \end{equation}
  Since we are using the $\omega$ provided by OMI, the AMF$_G$ term is removed from this calculation as it is not part Equation \ref{Model:AMF:eqn_omega} leading to 
  \begin{equation} \label{Model:AMF:eqn_AMF_intwSdz}
    AMF = \int_0^\infty{ \omega(z) S(z) \mathrm{d}z}
  \end{equation}
  
  Additionally the AMF can be determined using the sigma ($\sigma$) coordinate system, which denotes altitude as going from 1 (ground level) to 0 (TOA) (see Section \ref{Model:datasets}).
  A conversion to the $\sigma$ vertical coordinate is performed using $p = \sigma (p_S - p_T) + p_T$, where $p_T$ is pressure at the top of the atmosphere and $p_S$ is surface pressure. 
  $S_\sigma$ is a dimensionless normalised shape factor on the $\sigma$ coordinate system.
  In the sigma coordinate system we calculated the shape factor as defined in \textcite{Palmer2001}:
  \begin{equation}
    \label{Model:AMF:eqn_ShapeFactorSigma}
    S_\sigma(\sigma) = \frac{\Omega_a}{\Omega_v}C_{HCHO}(\sigma)
  \end{equation}
  where $\Omega_a$ is the vertical column of air from the surface to the top of the atmosphere and C$_{HCHO}(\sigma)$ is the mixing ratio of HCHO.
  We can integrate over the sigma coordinates using the hydrostatic relation $p = - \rho_a g z$, with $\rho_a$ being air density, and $g$ the gravitational acceleration:
  \begin{align*}
    \rho_a g z & = \sigma \left( p_S - p_T \right) + p_T \\
    \mathrm{d}\sigma  & = - \frac{ \rho_a g }{ p_S - p_T } \mathrm{d}z
  \end{align*}
  Substitution into \ref{Model:AMF:eqn_AMF_intwSdz} gives AMF using the sigma coordinates:
  \begin{equation} \label{Model:AMF:eqn_AMFintwSdsigma}
    AMF = \int_0^1 w(\sigma) S_\sigma(\sigma) \mathrm{d}\sigma
  \end{equation}

\section{Recalculation of OMI HCHO}
  \label{Model:omiRecalc}
  
  
  % Reiteration of why we recalculate AMFs
  %from Equation \ref{Model:AMF:eqn_AMFFrac}:
  %\begin{equation*} %\label{Model:omiRecalc:eqn_AMFratio}
  %%  AMF = \frac{SC}{\Omega} %= \frac{\tau_s}{\tau_v}
  %\end{equation*}
  OMI HCHO vertical columns are calculated using modelled a priori HCHO profiles (see Section \ref{Model:omhcho}).
  When comparing satellite measurements against models it is important to recognise the impact of this a priori on the total column values.
  This is complicated by how OMI is differently sensitive to HCHO (and other trace gases) vertically throughout the atmosphere.
  When comparing OMI vertical columns ($\Oomi$) to GEOS-Chem ($\Ogc$), the satellite AMF needs to be recalculated using GEOS-Chem modelled vertical gas profiles as the a prioris.
  Without performing this step a bias between modelled and measured total column values may be due to the a priori rather than chemistry or measurements \parencite{Palmer2001, Lamsal2014}.
  
  
  % Very brief outline
  To recalculate vertical columns two components are calculated: and AMF and a reference sector correction.
  Here, two new AMFs are calculated, both using GEOS-Chem HCHO profiles as the new a priori. 
  The first (AMF$_{GC}$) recalculates the satellite shape factor, but uses the original satellite scattering weights.
  The second (AMF$_{PP}$) recalculates both the satellite shape factor and the scattering weights.
  AMF$_{PP}$ is created using code initially written by Professor P. Palmer. % (see section \ref{Model:omiRecalc:AMF} for more details).
  
  In addition to a new AMF, a reference sector correction is determined using the method described in \textcite{Abad2016}. %(see Section \ref{Model:omiRecalc:RSC}).
  This correction has also been applied to the original vertical column in the OMHCHO product, so the column is available with and without this correction.
  The reference sector correction is unique for each of the 60 measurement tracks used by OMI.
  The correction is calculated for each of the two new vertical columns, and applied to each pixel to create the corrected vertical columns. 
  The end product is three sets of corrected vertical columns: the original, one based on new shape factors, and one from with new shape factors and scattering weights.
  
  \subsection{Outline}
    \label{Model:omiRecalc:outline}
    
    % Summary of actual actions taken:
    An outline in computational order of what takes place when recalculating the $\Omega$ from OMI follows:
    \begin{enumerate}
      \item Section \ref{Model:omiRecalc:shape_factors} describes how GEOS-Chem satellite overpass output %(see Section \ref{Model:GC:simulations:outputs})
        is used to create new shape factors (S$_z$ and S$_\sigma$).
      \begin{enumerate}
        \item Vertical pressure edges and geometric midpoints are determined, along with altitudes (\textit{z}), and box heights $H$.
        \item Number density and mixing ratio of HCHO ($n_{HCHO}$, $C_{HCHO}$ respectively) are taken or created from model outputs HCHO (ppb), air density (molec cm$^{-3}$), and box heights (m).
        \item Total column HCHO from GEOS-Chem ($\Ogc$) is calculated 
        \begin{equation*}
          \Ogc = \Sigma_z \left(n_{HCHO}(z) \times H(z) \right)
        \end{equation*}
        along with total column air ($\Omega_{A}$, calculated similarly).
        \item The shape factor S$_z(z)$ is calculated for each altitude 
        \begin{equation*}
          S_z(z) = n_{HCHO} / \Omega_{HCHO}
        \end{equation*}
        \item Sigma coordinates are created from pressures (see Section \ref{Model:datasets}, Equation \ref{Model:datasets:eqn_sigma_pressure}).
        \item S$_\sigma(z)$ is calculated on each altitude: 
        \begin{equation*}
          S_\sigma(z) = C_{HCHO}(z) \times \frac{\Omega_A}{\Omega_{HCHO}}
        \end{equation*}
      \end{enumerate}
      \item For each pixel, new AMF (AMF$_{GC}$ and AMF$_{PP}$) is created using the GEOS-Chem shape factors and satellite scattering weights in Section \ref{Model:omiRecalc:AMF}:
      \begin{enumerate}
        \item Satellite pixels (SC, scattering weights ($\omega(z)$), pressure levels, latitude and longitude) are read from the OMHCHO dataset.
        \item scattering weights ($\omega$) are interpolated onto the same vertical dimensions (\textit{z} and $\sigma$) as the shape factors.
        \item Integration (approximated using rectangular method) is performed along the vertical dimension to calculate the new AMF on both coordinate systems:
        \begin{align}
          AMF_z &= \Sigma_z \left(\omega(z) \times S_z(z) \times H(z)\right) \\
          AMF_{\sigma} &= \Sigma_{\sigma} \left(\omega(\sigma) \times S_{\sigma}(\sigma) \times d\sigma \right)
        \end{align}
        These two AMFs represent the same thing using different coordinates, and either one can be used as the AMF$_{GC}$.
        \item Code from Dr. Surl and Professor Palmer is used to recalculate both the shape factor and scattering weights to produce AMF$_{PP}$.
      \end{enumerate}
      \item Section \ref{Model:omiRecalc:vertical_columns} describes how the original AMF, AMF$_{PP}$, and AMF$_{GC}$ are used to determine the new vertical columns: $\Omega = SC/AMF$.
      \item A reference sector correction (RSC) is defined each day using these AMFs along with modelled HCHO over the remote Pacific in Section \ref{Model:omiRecalc:RSC}:
      \begin{enumerate}
        \item GEOS-Chem midday output ($\Omega_{GEOS-Chem}$ from 140{\degr}W to 160{\degr}W) are averaged monthly and longitudinally to provide modelled vertical columns over the reference sector.
        \item These vertical columns ($\Omega_{GEOS-Chem}$) are turned into modelled slant columns by multiplication with the AMF over the reference sector:
        \begin{equation*}
          SC_{GEOS-Chem} = \Omega_{GEOS-Chem} \times AMF
        \end{equation*}
        \item For each satellite pixel in the reference sector, a correction ($corr$) is calculated as the measured $SC$ minus the modelled slant column at the nearest latitude:
        \begin{equation*}
          corr[lat,track] = SC[lat,track] - VC_{GEOS-Chem}[lat] \times AMF
        \end{equation*}
        \item These corrections are binned by satellite detector (track: 1-60), and latitude (0.36\degr; 500 latitudes from 90{\degr}S to 90{\degr}N).
        \item The median entry of each bin is determined and this forms the RSC[lat,track] (e.g., Figure \ref{Model:omiRecalc:RSC:fig_track_correction_interpolations}).
      \end{enumerate}
      \item $VCC$ are determined using $VCC = (SC - RSC[lat,track] )/AMF$ for each measured SC and using each AMF, with the RSC linearly interpolated to the latitude of the satellite pixel (see Section \ref{Model:omiRecalc:vcc}).
      \item The VCC (along with most of the pixel and GEOS-Chem data) are binned onto a \highhr ~grid along with other information (see Section \ref{Model:omiRecalc:binning}).
    \end{enumerate}
 
    
      
  
  \subsection{Creating new shape factors}
    \label{Model:omiRecalc:shape_factors}
    % Reminder of AMF -> leading to what is the shapefactor
    To visualise and analyse OMI HCHO columns, slant columns are transformed into vertical columns using the AMF.
    The shape factor ($S$) is one of the key components in creation of the AMF (see Section \ref{Model:AMF}, Equation \ref{Model:AMF:eqn_AMF_intwSdz}).
    The shape factor is calculated using GEOS-Chem midday output (see Section \ref{Model:GC:simulations:outputs}) which provides simulated HCHO concentration profiles ($\eta(z)$) and total columns ($\Omega$) at \lowhr ~horizontal resolution.
    Using Equation \ref{Model:AMF:eqn_shapefactor} to determine the shape factor is straightforwards $S(z) = \frac{\eta(z)}{\Omega}$.
    The associated OMI per-pixel scattering weights are not changed in this calculation (unlike in Section \ref{Model:omiRecalc:ppcode}).
    
    Model output is provided as mixing ratios ($C$ in ppbv), and is converted before being used in the shape factor calculation.
    The following equation converts model profile output from ppbv into number density ($\eta$) in molecules cm$^{-3}$:
    \begin{equation} \label{Model:omiRecalc:eqn_ppb_to_n}
      \eta_{HCHO} = C_{HCHO} \times \eta_a \times 10^{-9}
    \end{equation}
    where $\eta_{HCHO}$ is the number density of HCHO, $\eta_a$ is the number density of air (from model output), and $C_{HCHO}$ is the mixing ratio of HCHO.
    The modelled total vertical column $\Omega_{HCHO}$ is determined by:
    \begin{equation*}
      \Omega_{HCHO} = \Sigma_z \left( \eta_{HCHO} \times H(z) \right)
    \end{equation*}
    where $H(z)$ is the box height for altitude $z$.
    In effect this equation sums over the molecules per cm$^2$ in each altitude level.
    
    As a sanity check $S_\sigma$ is calculated (through Equation \ref{Model:AMF:eqn_ShapeFactorSigma}) to confirm that these shape factors are equivalent.
    Comparing the resulting AMFs created by Equations \ref{Model:AMF:eqn_AMFintwSdsigma} and \ref{Model:AMF:eqn_AMF_intwSdz} for each pixel provides confidence in the unit conversions (and other factors) applied.
    These AMF values are nearly identical in practice (correlation coefficient $> 0.99$).
    

  % How we get the a priori 
  \subsection{Creating new AMF using GEOS-Chem}
    \label{Model:omiRecalc:AMF}
    
    \subsubsection{Updating shape factors}
    
      From Equation \ref{Model:AMF:eqn_AMF_intwSdz} we have:
      $$ AMF = \int_0^\infty \omega(z) S(z) \mathrm{d}z $$
      Using the $\omega(z)$ from satellite data, along with our calculated $S_z$ interpolated linearly onto the same vertical grid as $\omega(z)$, the AMF can be determined through integration.
      The integration is performed using a simple rectangular method, which multiplies the integrand midpoints by the change in height, and then takes the sum for each vertical box.
      This assumes that the provided scattering weights and shape factors are linear between the 47 resolved values.
    
    \subsubsection{Recalculating the AMF using PP code}
      \label{Model:omiRecalc:ppcode}
      % pp code for AMF recalculation referred to as pp code
      
      %You'll only be showing the Australia ones, right? So I don't think you need this. Instead, I think you need to start this section by saying that a limitation of the method described in the previous subsection is that the scattering weights can also depend on the model atmosphere, which is not taken into account when only the shape factor is recalculated.
      
      The major limitation of vertical columns that implement $AMF_{GC}$ is that the scattering weights also depend on the model atmosphere, which is not updated when only the shape factor is recalculated.
      This is addressed using Fortran code written by Paul Palmer and Randal Martin, subsequently updated by Luke Surl.
      This Fortran code (PP code) is computationally expensive, and is run on a subset of the globe (50-10\degr~S, 160\degr~W-160\degr~E) covering Australia and the Pacific ocean.
      This allows both vertical column recalculation and reference sector correction..
      The instrument sensitivity (or scattering weights; $\omega$) and shape factors for each pixel are calculated within the PP code.
      
      The code uses a combination of GEOS-Chem a priori profile information and satellite measurement data to calculate the $AMF_{PP}$ by using LIDORT radiative transfer calculations to determine scattering.
      
      
      Code for recalculating AMF using satellite swaths and modelled aerosol optical depths and gas profiles can be found at \url{http://fizz.phys.dal.ca/~atmos/martin/?page_id=129}. 
      The coded method is detailed in \textcite{Palmer2001}, with modifications for clouds and use of the LIDORT radiative transfer model \parencite{Spurr2002} as described by \textcite{Martin2003}.
      Modifications performed by Luke Surl at University of Edinburgh enabled the PP code to utilise OMI satellite data.
      Additionally, required tropopause heights averaged within satellite overpass times are created by modifying the GEOS-Chem diagnostic output.
      The calculation of $\omega$(z) is simplified by using large ``look up tables'' of values based on many parameters such as cloud top heights and optical depths.
      These AMF look up tables can be found in the source code at \url{https://github.com/LukeSurl/amf581g}.
      The PP code uses HCHO concentration profiles averaged between 1300 and 1400 LT, including optical depths at 550~nm, and dust concentrations from GEOS-Chem, along with a subset of the OMI pixel information.
      The required information is taken from OMHCHO daily swath files, and sorted into csv files (one per day) which are then read by the PP code in conjunction with the GEOS-Chem outputs for each day.
      The PP code then produces a list of recalculated AMF that are read by python code and associated with the corresponding satellite pixel.
    
    
      
    \subsubsection{Saving the AMF with satellite pixels}
      
      % Downloading the data
      OMI satellite pixels are read and filtered for quality (see Section \ref{Model:omhcho:pixel_filtering}) and have their AMFs calculated as shown above. 
      The process is outlined in figure  \ref{Model:saving_omhchorp:fig_flow_omhchorp} for a single day.
      This filtering removes highly uncertain pixels, including row anomaly affected pixels.
      % reading pixels into a long list
      Additional information is added to each pixel, including the new AMFs.
      Each pixel and its relevant data are saved in a long list for subsequent processing.
      %The shape factors and scattering weights for each pixel lie along a z-axis that is vertically resolved to 47 layers.
    

  \subsection{Vertical columns from AMF}
    \label{Model:omiRecalc:vertical_columns}
    All that remains for recalculating the total vertical column using our new a priori shape factor is to apply the new AMF to the slant columns and grid them onto our chosen resolution.
    Each satellite pixel at this stage has an associated SC along with three $AMF$s: the original ($AMF_{OMI}$), one with recalculated shape factors (AMF$_{GC}$), and one completely recalculated using PP code (AMF$_{PP}$).
    These are used to create vertical columns ($\Omega$) through Equation \ref{Model:AMF:eqn_AMFFrac}: $\Omega = SC/AMF$.
    
    Figure \ref{Model:omiRecalc:vertical_columns:fig_AMFdistributions} shows a comparison between the three satellite $\Omega$ over Australia for 2005. 
    Fully recalculated $AMF_{PP}$ show a distinct bi-modal distribution compared against the approximately Gaussian distributions of the $AMF_{GC}$ and the original $AMF_{OMI}$.
    Additionally there is a steep decline in mean $AMF_{PP}$ after February that is not seen in the other AMF, and for the rest of the year the three AMF follow the same pattern with $AMF_{PP}$ around $10\%$ lower on average.
    While the explanation for this is beyond the scope of this thesis, it does show that a complete recalculation of both the shape factor and the scattering weights makes a large difference in the AMF calculation when compared to just recalculating the shape factor.
    
    % Figure from chapter_2_modelling AMF_distribution
    \mypic{Figures/OMI_link/AMF_distributions_2005.png}{%
      Top row: averaged OMI Satellite AMF for 2005, from the OMHCHO data set (left, $AMF_{OMI}$), recalculated using GEOS-Chem shape factors  (middle, $AMF_{GC}$), and recalculated using GEOS-Chem shape factors and scattering weights (right, $AMF_{PP}$).
      Middle row: AMF time series over 2005 for each recalculation.
      Bottom row: AMF frequency distributions over January and February.
      Oceanic pixels are filtered out, only land pixels are included in this figure.
    }{\label{Model:omiRecalc:vertical_columns:fig_AMFdistributions}}
    
  \subsection{Reference sector correction}
    \label{Model:omiRecalc:RSC}
    
    % Why we apply a reference sector correction
    On a medium to long time scale, OMI degradation needs to be accounted for.
    To remove effects from the deterioration of the satellite instrument, measurements over the remote pacific (our reference sector) are combined with GEOS-Chem simulations to create a reference sector correction (RSC).
    In this thesis we use OMI HCHO columns to estimate isoprene emissions, and correcting background HCHO does not affect this calculation.
    % hcho over the remote Pacific ocean is expected to be relatively invariable.
    The correction should reduce bias caused by satellite degradation, without impacting isoprene emissions estimation.
    The RSC corrects for several problems; however, it introduces some a priori model influence.
    One corrected problem is the potential influence of varying dead/hot pixel masks across the OMI 2-D detector array \parencite{DeSmedt2015}.
    This method also corrects for the errors introduced through correlations between BrO and HCHO absorption cross sections, which are especially significant at high latitudes \parencite{Abad2015}.
    
    % where we apply the RSC
    HCHO products from OMI and SCIAMACHY both use a median daily remote Pacific ocean radiance reference spectrum, over 15$^{\circ}$S-15$^{\circ}$N, 140$^{\circ}$-160$^{\circ}$W where it is assumed that the only significant source of HCHO is methane oxidation \parencite{DeSmedt2008,Barkley2013,Kurosu2014}.
    Here a new reference sector correction is created using modelled and measured HCHO columns over the remote Pacific to produce corrected vertical columns ($VCC$).
    This follows \textcite{Abad2016}, and defines the remote Pacific as the band between 140{\degr}W to 160{\degr}W.
    Each satellite slant column measurement is corrected by how much the satellite reference sector measurements at that latitude diverge from modelled amounts over the reference sector.
    
    
    % How my RSC is defined
    Modelled slant columns over the reference sector ($SC_{GEOS-Chem}$) are calculated by multiplying modelled vertical columns with the AMF calculated in prior sections using Equation \ref{Model:AMF:eqn_AMFFrac}: 
    \begin{equation*}
      SC_{GEOS-Chem} = \Omega_{GEOS-Chem} \times AMF
    \end{equation*}
    % Then RSC Longitudinally averaged, binned into 500 lats
    The longitudinal average is taken within the remote Pacific, as corrections are (assumed to be) longitudinally invariant.
    These modelled slant columns are averaged over the month and interpolated latitudinally to 500 equidistant bins.
    Figure \ref{Model:omiRecalc:RSC:fig_RSCeg} shows the simulated reference sector vertical columns as an example, calculated for January 1st 2005.
    In this figure the latitudinal resolution is increased from 2\degr ~to 0.36\degr, through linear interpolation, in order to form 500 vertical bins that are used in correcting the satellite data.
    This resolution is chosen to match that of \textcite{Abad2016}.
    
    
    % This picture was made RSC_tests -> Summary_RSC
    \mypic{Figures/HCHO/Summary_RSC_Effect8d_20050101.png}
    {%
      Example of remote Pacific RSC using 8-day average measurements and one month modelled data.
      $OMI VC$ shows the uncorrected vertical columns, while $Corrected$ shows the corrected vertical columns.
      OMI corrections shows the correction applied globally based on latitude and OMI track number(sensor).
      $\Omega_{GC}$ shows the GEOS-Chem modelled HCHO VC over the reference sector (region within black vertical lines), with $\Omega_{VCC}$ showing the corrected VC over the same area.
    }
    {\label{Model:omiRecalc:RSC:fig_RSCeg}}
    
    
    % Swath RSC corrections are done 'per track'
    For OMI swaths, each row of measured data contains 60 ``Across track'' (track) measurements.
    The track index $i$ relates the measurement to one of the 60 columns of data.
    Corrections (molecules cm$^{-2}$) for each measurement are calculated by taking the difference between the measured slant column and the a priori slant column as follows:
    \begin{equation} \label{Model:omiRecalc:eqn_RSC}
      Correction(i,j) = SC(i,j) - \Omega_{GEOS-Chem}(lat(j)) \times {AMF}(i,j)
    \end{equation}
    where $j$ represents a latitude index.
    The RSC is this list of $Correction$ values binned by latitude using medians, and used per pixel based on the linear interpolation to that pixels latitude.
    % Correction applied over each of the 60 tracks
    The RSC is independently calculated for each of the 60 tracks, at each latitude in the 500 0.36\degr ~bins.
    This provides a different RSC for each of the three AMFs.
    % Interpolated daily RSC to cover missing latitudes
    Due to incomplete latitudinal coverage, the correction for each track is interpolated linearly between measurements, with corrections outside of the highest measured latitudes being equal to the corrections at the highest measured latitudes.
    
    Figure \ref{Model:omiRecalc:RSC:fig_track_correction_interpolations} shows an example of the 60 track corrections for January 1st 2005
    The points are satellite measurements longitudinally averaged over the remote Pacific, coloured by track number.
    % Visualisation of RSC refering to plot and 8-day averaged corrections
    Another way to look at this correction is given in the OMI corrections panel of Figure \ref{Model:omiRecalc:RSC:fig_RSCeg}, which has tracks along the x axis and latitude on the y axis.
    This figure shows how corrections are distributed (over this 8-day sample) with more negative values towards the outside tracks, especially in the tropics.
    
    \mypic{Figures/OMI_link/RSC_track_corrections20050101.png}
      {Example of track correction interpolations for January 1st 2005, points represent the difference between satellite slant column measurements and modelled slant columns over the remote Pacific.}
      {\label{Model:omiRecalc:RSC:fig_track_correction_interpolations}}
    
  
  \subsection{Corrected vertical columns}
    \label{Model:omiRecalc:vcc}
    
    % How Vertical columns use reference sector corrections
    %Vertical columns in OMI use an oceanic background instead of a solar irradiance spectrum when calculating trace gas interference with DOAS calculations.
    Corrected vertical columns, or VCC, are created using the difference between slant columns ($SC$) and reference sector corrections (RSC, detailed in prior section) divided by the AMF. 
    \begin{equation}
      \label{Model:omiRecalc:vcc:eqn_vcc}
      VCC = \frac{ \left( SC - RSC \right) }{ AMF }
    \end{equation}
    This is equivalent to taking the difference between the slant column and its measured reference sector equivalent ($SC_0$), and then adding the modelled reference sector column ($\Omega_{GC,0}$):
    \begin{equation}
      \label{Model:omiRecalc:vcc:eqn_vcc_equiv}
      VCC = \frac{ \left( SC - SC_0 \right) }{ AMF } + \Omega_{GC,0}
    \end{equation}
    This method is used in several works, including \textcite[e.g.,][]{DeSmedt2008, DeSmedt2012, DeSmedt2015, Barkley2013, Bauwens2016}.
    Recently this correction was expanded (for OMI data) to include latitudinal and instrument track influence by \textcite{Abad2015}.
    
    % One correction per pixel, based on latitude and omi track
    One $correction$ (Equation \ref{Model:omiRecalc:eqn_RSC}) is associated with every good satellite measurement.
    This is used to create a reference sector corrected measurement (VCC) through the following equation:
    \begin{equation}
    VCC(i,j) = \frac{SC_{HCHO}(i,j) - correction(i,lat(j))}{AMF(i,j)}
    \end{equation}
    For each good satellite measurement the corrected vertical column is calculated two times, once using each new AMF.
    
    

  \subsection{Binning the results daily}
    \label{Model:omiRecalc:binning}
    Finally the pixels are binned into a gridded dataset named OMHCHORP, as shown in Figure \ref{Model:saving_omhchorp:fig_flow_omhchorp}.
    The resolution is chosen to match the native resolution of the GEOS meteorological fields (\highhr).
    A bin entry count is used to allow easy re-binning, and can be used to check for sparse data days due to filtering.
    Data averaged into this dataset:
    \begin{enumerate}
      \item OMI slant column %($\Os$)
      \item OMI air mass factor %(AMF$_{OMI}$)
      \item OMI vertical column %($\Omega_{V,OMI}$
      \item OMI corrected vertical column %($\Omega_{VC,OMI}$)
      \item GEOS-Chem recalculated air mass factor %(AMF$_{GC}$)
      \item GEOS-Chem recalculated vertical column %($\Omega_{V,GC}$)
      \item GEOS-Chem recalculated corrected vertical column %
      \item GEOS-Chem air mass factor recalculated using PP code (AMF$_{PP}$)
      \item GEOS-Chem corrected vertical column based on PP code
      \item satellite pixel counts (summed into bins)
      \item OMI vertical column fitting uncertainty
      % Fires are now stored separately
      %\item Smoke AAOD from OMAERUVd mapped into bins from 1x1\degr ~resolution (described in Section \ref{Model:filter:fire})
      %\item fire counts (summed into bins)
    \end{enumerate}
    
    This process requires processing time and storage space, and is performed on the National Computing Infrastructure (NCI) supercomputer.
    In order to reprocess one year of swath files ($\sim 162$~GB $ = 142 + 16 + 4$~GB OMHCHO, MOD14A1, and OMNO2d respectively per year) of daily data was downloaded and then transformed into $\sim 8$~GB (per year) of daily gridded data.
    This takes around 90 minutes per day; however, each day is completely independent and can be run in parallel once model output is available.
    Initially parallelism was built into the python code; however, simply running python code independently for each date was simpler and more scalable.
    As much as possible this work uses HDF-5 or NetCDF-4 formats, although GEOS-Chem output is in bitpunch format.
    The scripts to regrid and reprocess the swath data set are available from github at \url{https://github.com/jibbals/OMI_regridding}.
  
  \subsection{Difference between original and corrected OMI HCHO columns}
  \label{Model:omiRecalc:diff}
    
    Corrected vertical columns ($VCC$) of HCHO are created at \highhr ~horizontal resolution.
    Three $VCC$ are created based on OMI and the new AMFs (from section \ref{Model:omiRecalc:vertical_columns}) and RSCs.
    Figure \ref{Model:omiRecalc:diff:fig_vcc_comparison} shows how the recalculated columns compare to the original (OMI, left). 
    $VCC_{OMI}$ are the lowest overall for the depicted month, and recalculating just the shape factor increases the column amounts ($VCC_{GC}$); however, recalculating both the shape factor and the scattering weights leads to a broader spread of values.
    Both sets of recalculated columns ($GC$, $PP$) show more high concentration ($>1.5$ \moleccm) columns and lower background concentrations over the ocean.
    Together this may lead to a slightly steeper continental enhancement that may be expected since elsewhere OMI HCHO is seen to be up to 40\% underestimated \parencite[e.g.,][]{Zhu2016,DeSmedt2015,Barkley2013}.
    %The effect of not recalculating the $\omega_z$ can also be seen in Figure \ref{Model:omiRecalc:fig_VCC_pp_fires}, which shows the altered satellite vertical columns using each method.
    
    
    % Figure showing VCC omi, gc, and pp, and distribution
    % produced in tests.py -> check_products()
    \mypic{Figures/OMI_link/VCC_check_200501.png}{%
      Row 1: regridded corrected $\Omega_{HCHO}$ from OMHCHO on January 1, 2005: original (left), recalculated using new shape factors (middle), and additionally using updated scattering weights (right).
      Row 2: shows the monthly average for January 2005.
      Row 3: shows the distribution over the month for each of the three column amounts.
      Distribution bins are logarithmic, resulting in wider bins for higher column amounts.
    }{\label{Model:omiRecalc:diff:fig_vcc_comparison}}
    
    Potential TODO: Worth looking at regressions between the three AMF and VCC techniques? 
    Plot and discussion paragraph would go here.
    
    
    When reading $AMF_{PP}$, missing and near zero ($AMF_{PP}<0.0001$) values are removed, which leads to fewer overall pixels used to recalculate $VCC_{PP}$ than the original $VCC_{OMI}$. 
    Figure \ref{Model:omiRecalc:diff:fig_Npix_comparison} shows how many pixels go into the column recalculation over Australia each month.
    The filtered pixel counts refers to how many pixels are removed through applied filters, which are described in the following section (Section \ref{Model:filter}).
    A decrease in pixel counts is apparent starting in April and lasting until August, which is caused by the reduction in good pixels recorded at increased solar zenith angles that occur during winter at higher latitudes.
    There is a consistent bias of $1-10\%$ between the techniques, caused by the extra filtering applied to unreasonable AMFs calculated by the PP code.
    
    %Stricter filtering must be balanced against both coverage and the sensitivity of the AMF determination to recalculating vertical columns.
    
    % plot made in chapter_2_modelling.N_pixels_comparison()
    \mypicw{0.75\textwidth}{Figures/OMI_link/N_pix_comparison.png}{%
      Monthly pixel counts over Australia (non-oceanic) used in recalculations of vertical columns.
      $N_{GC}$ refers to the number of pixels used in AMF recalculation without running code from Paul Palmer, $N_{PP}$ refers to those that have been recalculated using their code.
      Filters applied are described in Section \ref{Model:filter}, 
      }{\label{Model:omiRecalc:diff:fig_Npix_comparison}}
    
    Figure \ref{Model:omiRecalc:diff:fig_VCC_distributions} shows global and Australian HCHO averaged total column maps for 2005, along with time series and distributions.
    The completely recalculated $VCC_{PP}$ has a higher number of high HCHO amounts compared to both the original and partially recalculated VCCs.
    Both recalculated VCCs appear to have lower average background amounts, seen over the ocean and parts of southern Australia, and also increased average amounts in the HCHO hot spots around the north and eastern coastal regions.
    This comparison shows how reprocessing with an updated model can have a systematic influence on the total column.

    
    % Figure from chapter_2_modelling AMF_distribution
    \mypic{Figures/OMI_link/VCC_distributions_2005.png}{%
      Top row: averaged OMI Satellite VCC for 2005, from the OMHCHO data set (left, $VCC_{OMI}$), recalculated using GEOS-Chem shape factors  (middle, $VCC_{GC}$), and recalculated using GEOS-Chem shape factors and scattering weights (right, $VCC_{PP}$).
      Middle row: VCC time series over 2005 for each recalculation.
      Bottom row: VCC frequency distributions over January and February.
      Oceanic pixels are filtered out, only land pixels are included in this figure.
    }{\label{Model:omiRecalc:diff:fig_VCC_distributions}}
    
    
\section{Filtering Data}
  \label{Model:filter}
  
  % Why filter data
  A major goal in this work is to infer biogenic isoprene emissions from HCHO columns.
  Isoprene is the dominant source of HCHO over continental land masses; however, other precursor NVMOCs can contribute to observed HCHO.
  The main interference in the biogenic isoprene signal in HCHO comes from fire smoke plumes and major anthropogenic source regions.
  Biomass burning can be a large local or transported (via smoke plumes) source of HCHO, glyoxal (CHOCHO), and other compounds that influence HCHO levels.
  Anthropogenic emissions from power generation, transport, and agriculture can influence these levels as well.
  So to infer biogenic isoprene emissions, pyrogenic and anthropogenic influences should be removed (where possible) from modelled and measured data.
  In GEOS-Chem we simply turn off pyrogenic and anthropogenic emissions; however, in the OMI HCHO satellite product we mask potentially affected pixels.
  
  
  In this work anthropogenic and pyrogenic influences on the OMHCHO satellite HCHO columns are removed by masking active fires, high AAOD, and high NO$_2$ levels measured by satellite.
  Active fires and suspected smoke plumes are masked, and together termed the pyrogenic filter.
  NO$_2$ measurements are used to mask potential anthropogenic influence.  
  %These masks negatively affect uncertainty,  as fewer measurements are available to be averaged.
  %This section describes the creation and effects of filters used on satellite data.
  A summary of yearly filtering over Australian land squares at \highhr ~resolution is provided in Table \ref{Model:filter:tab_summary_pixels_masked}.
  Figure \ref{Model:filter:fig_anthropyrofilters} shows an example year of anthropogenic and pyrogenic filtering, highlighting how many days and pixels are removed across Australia.
  The anthropogenic filter completely removes grid squares over Brisbane, Melbourne, and Sydney. 
  Other major cities in Australia either do not emit enough NO$_2$ or are too dispersed and do not breach the threshold to be filtered as anthropogenic.
  The anthropogenic filter removes from $\sim .25\% - \sim5\%$ of grid squares each day.
  Pyrogenic filtering removes from $\sim 5\% - \sim 20\%$ of the available grid square measurements per day.
  This filter tends to cover forested areas (as they are more prone to burning) as well as some hot spots that are likely due gas flaring or burning.
  
  
  % This table can be generated by test_filters.summary_pixels_filtered()
  \begin{table} \begin{threeparttable}
    \caption{Satellite pixels remaining after filtering by active fires, smoke, and anthropogenic masking. 
      In parenthesis are the portion of pixels filtered.}
    \begin{tabular}{ c c c c c c }
      \textbf{Year} & \textbf{Pixels} & \textbf{Active fires} & \textbf{Anthropogenic} & \textbf{Smoke} & \textbf{Total} 
      \\ \hline
      2005   &  3.9e+06    &  3.5e+06( 9.2\%)     &    3.8e+06( 1.3\%)    &    3.8e+06( 1.7\%)    &   3.4e+06(11.5\%)     \\
      2006   &  3.8e+06    &  3.4e+06(11.2\%)     &    3.7e+06( 2.4\%)    &    3.7e+06( 3.2\%)    &   3.2e+06(14.8\%)     \\
      2007   &  3.7e+06    &  3.4e+06( 9.9\%)     &    3.7e+06( 2.0\%)    &    3.7e+06( 2.6\%)    &   3.3e+06(13.0\%)     \\
      2008   &  3.5e+06    &  3.3e+06( 8.1\%)     &    3.5e+06( 1.4\%)    &    3.5e+06( 1.1\%)    &   3.2e+06(10.0\%)     \\
      2009   &  2.7e+06    &  2.4e+06( 9.1\%)     &    2.6e+06( 1.8\%)    &    2.6e+06( 1.7\%)    &   2.4e+06(11.6\%)     \\
      2010   &  2.0e+06    &  1.9e+06( 6.5\%)     &    2.0e+06( 1.4\%)    &    2.0e+06( 0.5\%)    &   1.9e+06( 8.1\%)     \\
      2011   &  1.9e+06    &  1.7e+06(12.5\%)     &    1.9e+06( 1.9\%)    &    1.9e+06( 2.7\%)    &   1.6e+06(15.2\%)     \\
      2012   &  2.5e+06    &  2.2e+06(14.3\%)     &    2.5e+06( 1.8\%)    &    2.5e+06( 3.3\%)    &   2.1e+06(17.4\%)     \\
    \end{tabular}
    \begin{tablenotes}
      \item Pixels: how many land pixels are read over Australia, after cloud fraction filtering.
      \item Total removed pixels accounts for overlap between filters.
    \end{tablenotes}
    \label{Model:filter:tab_summary_pixels_masked}
  \end{threeparttable} \end{table}
  

  
  % Figure from test_filters . show_mask_filtering - two images pasted side by side
  \mypic{Figures/OMI_link/Filters/AnthroPyroFilters_2005.png}{%
    Top row shows grid squares filtered out by anthropogenic (left) and pyrogenic (right) influence masks over 2005.
    Bottom row shows portion of Australian grid squares filtered out each day. 
    }{\label{Model:filter:fig_anthropyrofilters}}
  
  
  \subsection{Pyrogenic filter}
    \label{Model:filter:fire}
    
    % Products making up the pyro filter
    MODIS fire counts are used in conjunction with smoke AAOD enhancements from OMI to remove data points that may be affected by fires or fire smoke plumes.
    %The method used in this thesis follows that of \textcite{Marais2012} and \textcite{Barkley2013}, with active fires filtered using fire counts and smoke filtered using smoke AAOD.
    The MODIS fire counts come from a combination of measurements from the Terra and Aqua satellites (Terra overpasses at 10:30, 22:30 LT; Aqua at 13:30, 01:30 LT).
    satellite AAOD from product OMAERUVd (described in Section \ref{Model:datasets:OMAERUVd}) is analysed over Australia to determine a suitable filter threshold.
    AAOD is used instead of an alternative product AOD as it is less sensitive to the presence of clouds \parencite{Ahn2008}.
    
    % My method for fire filtering:
    OMHCHO total column HCHO $\Omega$ is processed into a \highhr ~horizontal daily grid.
    Pyrogenic filters are interpolated to the same horizontal resolution as $\Omega$ to simplify application. 
    The following steps are performed to create the pyrogenic influence mask:
    \begin{enumerate}
      \item MOD14A1 daily gridded Aqua/Terra combined fire counts (1x1~km$^2$) are binned into \highhr ~bins (matching the resolution of binned $\Omega$).
      \item A rolling mask is formed that removes $\Omega$ if one or more fires are detected in a grid square, or in the adjacent grid square, up to 2 days previously.
      This includes the current day, making 3 days of fires in total being filtered out on each day.
      \item AAOD at 500~nm is mapped from OMAERUVd (1x1\degr ~resolution) onto the \highhr ~resolution.
      \item An AAOD threshold of 0.03 is determined through visual analysis of AAOD distributions over several days, including days with and without influence from active fires, dust, and transported smoke plumes. 
      Grid squares with AAOD over this threshold are considered potentially affected by transported fire smoke.
    \end{enumerate}
    This method of masking fires can be compared to \textcite{Marais2012} and \textcite{Barkley2013}:
    \textcite{Marais2012} removed pixels colocated with non zero fire counts in any of the prior eight days, within grid squares with 1\degr$\times 1$\degr ~resolution, and \textcite{Barkley2013} used fires from the preceding and current day, within local or adjacent grid squares, at \highhr ~resolution.
    
    
    % Fire filtering affect split over recalculations
    Figure \ref{Model:omiRecalc:fig_VCC_pp_fires} shows the effects from filtering HCHO vertical columns with different fire filtering.
    Vertical column amounts averaged over January, 2005, and the pixel counts are shown side by side.
    The figure shows the effects of applying an increasingly strict fire filter.
    Each row has a stricter fire filter applied from top to bottom, with no fire filter on the first row up to filtering pixels from squares with fires up to 8 days prior.
    Increasing prior days used when creating the fire filter leads to a serious reduction in available pixels, with a wider range of areas dropping to near zero pixel counts.
    The overall decrease in vertical column HCHO is minor, dropping from $\sim{4}$ to $\sim{3.9}$ \moleccm ~between applying no filter and the strictest filter.
    In this month the highest the fires are most prominent across Queensland and northern NSW, and a large section of the north eastern coast becomes largely filtered out.
    Increasing the strength of the active fire filter increases the regional impacts on HCHO columns.
    In this work two prior days of active fires are masked.
    
    % Figure from chapter_2_modelling.plot_VCC_firefilter_vs_days
    \mypic{Figures/OMI_link/VCC_fires_200501.png}{% 
      Recalculated OMI vertical HCHO columns and pixel counts for January 2005.
      Column 1: $VCC_{PP}$ amounts after applying the active fire filter.
      The filter is applied at \highhr ~resolution, before output is averaged into \lowhr ~bins.
      Column 2: Pixel counts after applying the active fire filter (summed into \highhr ~bins).
      Row 1-4: increasing number of prior days that have active fires are included when masking fire influence.
      The first row shows the values without applying any active fire filter.
      The average and maximum VCC column amount (at \highhr ~resolution), and number of pixels, is inset as text in column 1 of each row.
    }{\label{Model:omiRecalc:fig_VCC_pp_fires}}
    
    
    % How other people have filtered fires
    %Influence from biomass burning can be removed %through measurements of acetonitrile and CO \parencite[e.g.,][]{Wolfe2016, Miller2017}, or else removal of 
    %by using satellite detected fire counts and aerosol absorption optical depth \parencite{Marais2014}.
    
    %In situ fire influence
    %\textcite{Wolfe2016} disregard HCHO measurements when acetonitrile > 210~pptv and CO > 300~ppbv, while acetonitrile > 200~pptv is used to determine fire influence in \textcite{Miller2017}.
    
    
    % Smoke AAOD filtering:
    Determining the AAOD due to smoke can be difficult since both smoke and dust absorb UV radiation \parencite{Ahn2008, Marais2012}.
    AAOD filtering is designed to remove pixels affected by smoke; however, it may occasionally remove pixels affected by dust.
    Dust in Australia is highly episodic and false positives in the smoke filter should not affect more than a few days per month, especially over regions with high tree coverage \parencite{Shao2007}.
    % Threshold for AAOD determination
    % repeats steps taken paragraph
    %The smoke filter is created by removing OMHCHO grid squares where the AAOD is above 0.03, after the AAOD is mapped from 1\degr$\times 1$\degr ~to the same \highhr ~resolution as our OMHCHO gridded product.
    The threshold is determined through analysing AAOD over Australia in 4 scenarios: normal conditions, active local fires, transported fire smoke, and large scale dust storms.
    An example of these scenarios and the AAOD distributions is shown in Figure \ref{Model:filter:fire:fig_typicalAAOD}. 
    This figure shows AAOD maps and distributions, along with satellite imagery on the same day in column 4 (from \url{https://worldview.earthdata.nasa.gov/}).
    
    
    % Figure from !?!?
    \mypic{Figures/OMI_link/typical_AAODs_final.png}{%
      AAOD from OMAERUVd (columns 1, 2, 3) over Australia for four different scenarios (rows 1-4).
      Black Saturday (row 2) demarks the occurrence of widespread bush fires across Victoria. %, which caused 173 deaths.
      The transported plume in row three can be seen in the overlaid AOD shown in the last column, Indonesia (northwards from Darwin) suffered from wide-spread forest fires around this time.
      Scenes in the final column are created using the EOS Worldview website \url{https://worldview.earthdata.nasa.gov/} from satellite products provided therein.
      AAOD $=$ 0.03 is demarked by a horizontal line in the density plots in column 3.
      Density plots show normalised AAOD frequencies (scale not shown).
    }{\label{Model:filter:fire:fig_typicalAAOD}}
    
    
    Figure \ref{Model:filter:fire:fig_portion_filtered_2005} shows the extent of pixels filtered out by the pyrogenic filter over 2005 and 2012.
    One clear hot spot is located over Port Kembla (south of Sydney), most likely due to the flame that burns over the blast furnace stack throughout the year (\url{https://www.bluescopeillawarra.com.au/community/skylineimages/}).
    Another hot spot can be seen in Western Australia over Kalgoorlie, where a large open cut gold mine is always open and blasting daily (\url{https://superpit.com.au/wp-content/uploads/2015/01/Blasting-Information-Sheet.pdf}).
    In Western Queensland over Mount Isa there is again a mining hotspot, with a blast furnace and several smoke stacks (\url{http://www.mountisamines.com.au/en/about-mim/Pages/home.aspx}).
    A large area in southern Queensland/northern NSW is also heavily filtered, potentially due to gas flaring in the Surat Basin, which has thousands of petrol and gas wells (\url{http://www.ga.gov.au/scientific-topics/energy/resources/petroleum-resources/gas}).
    The highest concentrations of removed pixels lie along the northern and eastern coastlines, and correspond with forested areas (see Figure \ref{Model:filter:fire:fig_forest_coverage}), which suggests that forest fires are being masked properly.
    Central Australia is largely unmasked, which could be due to a lack of sufficient vegetation to create fires and smoke visible by satellite.
    %Even though transported smoke plumes from South America and southern Africa would influence the measurements in the area.
    The proliferation of petrol or gas wells (see Figure \ref{Model:filter:fire:fig_petrol_and_gas}) may also lead to AAOD enhancement wherever activity stirs up dust, and could be mistaken as active fires wherever gas flaring occurs.
    In 2005, 388 gas wells existed in Queensland; however, more than 2000 wells (cumulative) were approved by 2013 \parencite{Carlisle2012}.
    The comparison between fire filtering in 2005 and 2012 does not indicate that the proliferation of gas wells in the intervening years had any strong effect on data filtering. 
    
    %Figure from chapter_2_modelling.pyrogenic_filter() + annotation in paint
    \mypic{Figures/OMI_link/Filters/Pyrogenic_filter_annotated_3.png}{%
      Top: percentage of pixels filtered out by fire and smoke masks in 2005 (left) and 2012 (right).
      Bottom: percentage filtered out each day from land squares in Australia for the two years shown.
    }{\label{Model:filter:fire:fig_portion_filtered_2005}}
    
    
    \mypicwh{\textwidth}{\textheight}{Figures/Gas_Petrol_AUS.jpg}{%
      Top: gas fields and pipelines (2018) for Australia (\url{http://www.ga.gov.au/scientific-topics/energy/resources/petroleum-resources/gas}).
      Bottom: petrol Well locations over Australia (as of 2018) (\url{http://dbforms.ga.gov.au/www/npm.well.search}).
    }{\label{Model:filter:fire:fig_petrol_and_gas}}
    
    \mypic{Figures/Australiasforests_2016.png}{%
      Forest coverage, coloured by predominant tree species.
    }{\label{Model:filter:fire:fig_forest_coverage}}
  
      
    %\textcite{Zhu2013_poster} has a similar analysis over south-eastern USA showing an exponential correlation of ${HCHO} = \exp(0.15\times{T}-9.07)$.
    
    
  \subsection{Anthropogenic filter}
    \label{Model:filter:NOx}
    
    % Why filter high NO? Anthro influence
    %Need to explicitly state that you are not suggesting that anthropogenic NOx leads to HCHO, but that it is likely to be co-located with anthropogenic NMVOCs that form HCHO and therefore would bias your inversion. In other words, you are using NOx as a proxy for possible anthropogenic NMVOC influence.
    Enhanced NO$_2$ concentrations indicate anthropogenic influence over Australia.
    We use NO$_2$ as a proxy for potential anthropogenic NMVOC emissions, as these could bias the inversion performed in Chapter \ref{BioIsop}.
    A filter is designed using the tropospheric NO$_2$ columns in the OMNO2d product.
    
    % Nox analysis from OMNO2d
    NO$_2$ columns near several major cities in south eastern Australia over 2005 are used to determine a suitable threshold for anthropogenic influence.
    The mean, standard deviation, and time series over Australia of tropospheric NO$_2$ measured by satellite is shown in Figure \ref{Model:filter:NOx:fig_omno2_timeseries}.
    Tropospheric NO$_2$ columns averaged within all of Australia and each region is shown in Figure \ref{Model:filter:NOx:fig_omno2_timeseries}.
    Anthropogenic influences are clearly visible near major cities in Australia, and some influence can be seen along nearly every coastline.
    
    %Figure from !??
    \mypic{Figures/OMI_link/OMNO2_timeseries_2005_final.png}{%
      Mean (top left) and standard deviation (top right) of OMNO2d daily 0.25\degr$ \times 0.25 $\degr ~tropospheric cloud filtered NO$_2$ columns. 
      Time series for Australia, and each region (by colour) shown in the bottom panel, with mean for that region shown on the right. 
      The grey shaded area depicts the 25th to 75th percentiles of Australia averaged NO$_2$ columns for each day in the time series, with a thicker black line showing the Australia-wide mean value.}
    {\label{Model:filter:NOx:fig_omno2_timeseries}}
    
    % Filters and Threshhold determination and reasoning
    The anthropogenic filter is created for each year from in two steps:
    \begin{enumerate}
      \item Daily grid squares with NO$_2$ greater than $1.5\times10^{15}$\moleccm  ~are flagged as anthropogenic.
      \item After taking the yearly average for each grid square, any tropospheric NO$_2$ columns greater than $2 \times 10^{15}$\moleccm are flagged for the whole year.
    \end{enumerate}
    These thresholds are chosen subjectively through trial and error so as to ensure the removal of definite anthropogenic influence while not removing too many data points over near uninhabited portions of Australia.
    These thresholds completely remove grid squares over major cities that are likely emitting NMVOCs year round, and also frequently remove grid squares down wind.
    The affects of applying this filter to the OMNO2d product itself can be seen in Figure \ref{Model:filter:NOx:fig_omno2_threshaffect}.
    Areas over Sydney and Melbourne tend to oscillate around the upper quartile of Australian NO$_2$ tropospheric columns once the filter is applied.
    Many coastal regions show some effect from the filtering, and Sydney, Melbourne, and east of Melbourne are completely removed.
    
    % Figure from chapter_2_modelling.no2_thresh()
    \mypic{Figures/OMI_link/OMNO2_thresheffect_2005.png}{%
      Top row: 2005 OMNO2d NO$_2$ column mean before (left) and after (right) applying the threshold filters as described in the text.
      Bottom row: time series for Australia, and each region (by colour) shown in the bottom panel, with mean for that region shown on the right.
      The time series before (left) and after (right) applying the anthropogenic filter are shown, the dashed and dotted horizontal lines show daily and yearly mean thresholds respectively.}
    {\label{Model:filter:NOx:fig_omno2_threshaffect}}
    
    % Threshhold analysis
    The same regions as in Figure \ref{Model:filter:NOx:fig_omno2_timeseries} are shown again in Figure \ref{Model:filter:NOx:fig_omno2_densities}, with NO$_2$ pixel densities for each region shown, along with the threshold of $2 \times 10^{15}$\moleccm.
    The reduction in NO$_2$ columns along with the portion of available data removed over Australia and each subset region is listed in Table \ref{Model:filter:NOx:tab_summary}.
    Roughly a quarter of the available data-points are removed over Sydney and Melbourne, decreasing the mean NO$_2$ amounts by $\sim{50}\%$, while not too much information is lost overall.
    
    % Figure from test_filters.no2_densities
    \mypic{Figures/OMI_link/OMNO2_densities_2005.png}{%
      2005 OMNO2d NO$_2$ column means (top left) and distributions (top right) for Australia, and each region shown in the area map (by colour).
      Vertical dashed lines show the threshold for anthropogenic influence, any columns above this value are filtered out.
      The vertical axis is normalised so that area under the curve adds up to unity, and as such is not important except as a visual measure of the relative width between the distributions.
    }{\label{Model:filter:NOx:fig_omno2_densities}}
    
    
    % Values for table from chapter_2_modelling.no2_threshaffect_print
    \begin{table}
      \caption{NO$_2$ averages (\moleccm $\times 10^{14}$) by region before and after filtering for anthropogenic emissions using 2005 data from the OMNO2d product.}
      \begin{tabular}{ c c c c }
        \hline
        \textbf{Region} & \textbf{NO$_2$} & \textbf{NO$_2$ after filtering} & \textbf{\% Data lost} 
        \\ \hline
        Aus       & 4.9  & 4.5  &   2.1\% \\
        Sydney    & 10.4 & 5.9  &  25.3\% \\
        Adelaide  & 6.2  & 5.4  &   4.0\% \\
        Middle    & 3.8  & 3.7  &   0.5\% \\
        Melbourne & 9.6  & 5.3  &  25.4\% \\
        \hline
      \end{tabular}
      \label{Model:filter:NOx:tab_summary}
    \end{table}
    
  \subsection{Smearing filter}
  %\subsection{Accounting for smearing}
    \label{Model:filter:smearing}
    
    Smearing is a measure of how much HCHO in a given grid box was produced from isoprene emitted in a different (upwind) grid box.
    Smearing affects emissions estimates as HCHO enhancements downwind of where precursor emissions occurred lead to misinterpretation of local emissions.
    In high NO$_x$ ($ > \sim 1 $~ppb) environments, isoprene has a lifetime on the order of 30 minutes, and HCHO can be used to map isoprene emissions with spatial resolution from 10-100~km \parencite{Palmer2003}.
    In low NO$_x$ conditions, isoprene has a longer lifetime (hours) and may form HCHO further from the source area \parencite{Fan2004,Liu2016a,Liu2017_hpald}.
    Over Australia, NO$_x$ levels are generally low and smearing is therefore expected to be important.
    Smearing limits the horizontal resolution of the linear top-down inversion process, as a finer resolution increases sensitivity to transport.
    %The relatively coarse horizontal resolution (\lowhr) used by GEOS-Chem is advantageous in this aspect.
    Horizontal transport \textit{smears} the HCHO signal so that its source location would need to be calculated using wind speeds and loss rates \parencite{Palmer2001,Palmer2003}.
    In this chapter smearing affected grid squares are filtered out prior to application of Equation \ref{BioIsop:method:slope:eqn_isop_to_hcho}.
    
    %\textcite{Marais2012} additionally use airborn isoprene, MVK $+$ MACR (isoprene oxidation products), and HCHO measurements to check smearing in Africa where there is a sharp gradient of isoprene emitting vegetation from north to south.
    
    \subsubsection{Calculation of smearing}
    \label{Model:filter:smearing:calculation}
    
      Smearing has been analysed in several publications \parencite[e.g.,][]{Martin2003, Palmer2003, Millet2006, Stavrakou2009, Marais2012, Barkley2013, Zhu2014, Wolfe2016, Surl2018} and is often calculated using the method used in this thesis, as first described by \textcite{Palmer2003}.
      This involves using two model runs, one of which has isoprene emissions scaled globally by a constant (generally from 0.5 to 2).
      %Another method \parencite[e.g.,][]{Stavrakou2009} involves the analysis of an adjoint CTM, however this is computationally expensive and is not pursued here.
      % slope and yields
      %In order to understand the smearing calculation the underlying equations and assumptions must first be understood.
      From Section \ref{BioIsop:method:slope}, Equation \ref{BioIsop:method:slope:eqn_isop_to_hcho} states that the modelled slope ($S$) is the yield of HCHO per C of emitted isoprene divided by the HCHO loss rate ($S = \frac{Y_{isop}}{k_{HCHO}}$).
      % smearing defined from two runs of geos chem
      Using two runs of GEOS-Chem with differing isoprene emissions but otherwise identical we have:
      \begin{eqnarray}
      \label{Model:filter:smearing:calculation:eqn_runs}
      \begin{split}
      Run_1 :&  \Omega_{HCHO} = S E_{isop} + \Omega_0 \\
      Run_2 :&  \Omega_{HCHO}' = S' E_{isop}' + \Omega_0' 
      \end{split}
      \end{eqnarray}
      There are several assumptions that need to be understood, as these are what is tested by the smearing calculation.
      The initial assumption is that the system is at steady state, with no transport of isoprene affecting HCHO columns.
      This is the basis for equation \ref{Model:filter:smearing:calculation:eqn_runs}.
      It is also assumed that background values ($\Omega_0$) are from oxidation of methane and other long-lived VOCs, so that $\Omega_0 = \Omega_0'$.
      Between these two runs we are only changing the $E$ term.
      Chemistry is unchanged so that the yield and loss rate should not change between the two runs: 
      \begin{equation}
      S = S' = \frac{Y_{isop}}{k_{HCHO}}
      \end{equation}
      Equations \ref{Model:filter:smearing:calculation:eqn_runs} may then be combined as follows:
      \begin{eqnarray}
      Run_1-Run_2 : \Omega_{HCHO} - \Omega_{HCHO}' = & S E_{isop} - S' E_{isop}' +\Omega_0 - \Omega_0' \notag \\
      \Omega_{HCHO} - \Omega_{HCHO}' = & S \left( E_{isop} - E_{isop}' \right) \notag \\
      \Delta \Omega_{HCHO} = & S \Delta E_{isop}  \notag \\
      \hat{S} \equiv & \frac{\Delta{\Omega_{HCHO}}}{\Delta E_{isop}} \approx \frac{Y_{isop}}{k_{HCHO}} \label{Model:filter:smearing:calculation:eqn_hats}
      \end{eqnarray}
      This allows the combination of outputs from the two runs to determine where $\hat{S}$ diverges from expected values for $S$.
      %The modelled slope multiplied by the column HCHO loss rate ($k_{HCHO} = 1/\tau$) should approximate the HCHO yield from isoprene \parencite{Palmer2003, Barkley2013}.
      
      Potential smearing is masked by checking a daily modelled value for $\hat{S} \approx Y_{isop}/k_{HCHO}$ against thresholds.
      By assuming that midday HCHO lifetime ($\tau = 1/k_{HCHO}$) typically falls within 1.5 to 4~hrs (as seen in the USA; \textcite[e.g.,][]{Palmer2006,Wolfe2016}) and isoprene-to-HCHO yield (HCHO per isoprene carbon emitted) lies within the range 0.2 to 0.4 (scenarios estimated in \textcite{Palmer2003}), one can set a simple bound on $\hat{S}$ of $[0.2 \times 1.5, 0.4 \times 4]$~hrs or 1080 to 5760 seconds.
      As NO$_x$ levels across Australia are relatively low, and lower NO$_x$ levels reduce the prompt yield \parencite{Palmer2003,Wolfe2016}, I reduce the threshold range by 20\% and round to the nearest hundred leading to a bounding range of 800 to 4600 for $\hat{S}$. 
      This range strikes a balance between unlikely modelled yields and how much data is lost to filtering.
      Table \ref{Model:filter:smearing:tab_smearing_ranges} compares the smearing filter for Australia used in this thesis to typical slopes used in previous work for other regions.
      %This smearing range captures isoprene to HCHO yields of around 0.16 to 0.32 C per C if HCHO lifetime is assumed to lie within 1.5 to 4 hours.
      %These ranges mean we only use data that is not outside the feasible bounds of yields or lifetimes in Australia.
      
      \begin{table}\begin{threeparttable}
        \caption{Smearing filters or typical slopes ($S$) from literature.}
        \begin{tabular}{ l | c  c  l  >{\centering\arraybackslash}p{3cm} } 
          \toprule
          Source & min. (s) & max. (s) & type$^a$ & Region \\
          \midrule
          \textcite{Palmer2003}      & 1270 & 2090 & Range & North America$^{b}$ \\
          \textcite{Marais2012}      &      & 4000 & Limits & Africa \\
          \textcite{Barkley2013}$^c$ & 1300 & 1800 & Limits & South America \\
          \textcite{Surl2018}        & 2200 & 4900 & Range & India \\
          This Thesis             & 800  & 4600 & Limits & Australia \\
          \bottomrule
        \end{tabular}
        \begin{tablenotes} 
          \item a: Slope \textit{range}s are observed or modelled $S$, while smearing \textit{limits} are the applied acceptable limits for $S$. 
          \item b: Slope range for summer only.
          \item c: Assumed HCHO lifetime of 2.5 hours implies yields from 0.14 to 0.2 per C, consistent with box modelling.
        \end{tablenotes}
        \label{Model:filter:smearing:tab_smearing_ranges}
      \end{threeparttable}\end{table}
      
      
      %\subsection{HCHO Products and yield}
      %\label{BioIsop:results:HCHOYield}
      
      %To determine which model grid boxes are affected by smearing, we follow \textcite{Marais2012}.
      GEOS-Chem is run with normal isoprene emissions and with isoprene emissions halved, then Equation \ref{Model:filter:smearing:calculation:eqn_hats} ( $\hat{S} = \frac{\Delta \Omega_{HCHO}}{\Delta E_{isop}} $) provides $\hat{S}$.
      Here $\Delta$ represents the difference (daily 1300-1400~LT) between default and scaled runs.
      If $\hat{S}$ sits outside the 800-4600 range then we remove that grid square day from both $S$ and subsequent a posteriori calculations.
      A relatively large change in $\Omega_{HCHO}$ compared to local emissions ($\hat{S}>4600$) suggests HCHO production is from non-local isoprene emissions.
      Alternatively, a relatively low value of $\hat{S}$ ($\hat{S}<800$) suggests emissions from the local grid square are being exported before they form HCHO.
      
      
      % Marais 2012 also look at average windspeed and best hcho-isop corelation when hcho is shifted by 0.5 degrees, don't think I can do that with 2.5 degree resolution
      % Marais compare smearing with model estimated yield "
      
    
      
      % Subsection on smearing length scale was here, now in spare notes
      
      
      %Nitrous oxide is N2O. You are after nitric oxide (NO) or nitrogen oxides (NO2)
    \subsubsection{NO$_x$ dependence}
    
      %There is an important non-linear relationship between VOCs, HO$_x$, and NO$_x$. 
      NO$_x$ concentration directly affects the fate of VOCs in the atmosphere, influencing HCHO production by isoprene.
      In low NO$_x$ environments, reported HCHO yields from isoprene are around 0.2 - 0.3 C per C (or 100-150 molar \%), while in high NO$_x$ environments this value becomes two to three times higher \parencite{Palmer2003, Wolfe2016}.
      %\textcite{Wolfe2016} also see background HCHO doubling in high NO$_x$ regions.
      %\textcite{Wolfe2016} determine that going from NO$_X = 0.1$ to $2.0$ ppbv triples the prompt yield of HCHO, from 0.3 to 0.9 ppbv ppbv$^{-1}$ due to isoprene, while the background HCHO doubles.
      Some values for HCHO yield from prior literature are shown in Table \ref{BioIsop:method:tab_VOCLiteratureYields}.
      %Increasing NO$_x$ from 0.1~ppbv to 2.0~ppbv can triple the prompt isoprene yield of HCHO,  %from 0.3 to 0.9 ppbv ppbv$^{-1}$
      %and double background HCHO \parencite{Wolfe2016}.
      
      %Conversions between HCHO per unit C yield and molar \% yield from species X are given by the equation $ Y_{molar \%} = 100 \times C_X \times Y_{HCHO \,C^{-1}}$, where $C_X$ is how many carbon are within species X (5 for isoprene).
      %For instance a 200\% molar yield of HCHO from isoprene implies 1 mole of C$_5$H$_8$ becomes 2 mole HCHO which is a 0.4 HCHO per unit C yield.
      %      
      % yields from Atkinsen2003
      %isoprene
      %0.63 0.10 Tuazon and Atkinson (1990a)
      %0.57 0.06 Miyoshi et al. (1994)
      % a-pinene
      %0.23 0.09 Noziere et al. (1999a)
      %0.19 0.05 Orlando et al. (2000)
      % b-pinene
      %0.54 0.05 Hatakeyama et al. (1991)
      %0.45 0.08 Orlando et al. (2000)
      
      % molar HCHO yield per unit carbon equal to HCHO molar percent yield(per carbon)? or some conversion?
      \begin{table} \begin{threeparttable}
        \caption{ Isoprene to HCHO yields and lifetime.}% against oxidation by OH. }
        \begin{tabular}{  l  l  l  l  }
          \toprule
          HCHO Yield    & Lifetime     & NO$_x$ background & Source   \\
          (molar \% )   &              &                   &          \\
          \midrule 
          315$\pm$50      &            & High          & a        \\ 
          285$\pm$30      &            & High          & a        \\ 
          225             & 35 min     & High          & b        \\ % Done
          450             &            & High          & c        \\
          235             &            & 1~ppbv        & d        \\
          150             &            & Low           & b        \\ % Done
          150             &            & Low           & c        \\
          150             &            & 0.1~ppbv      & d        \\
          %          $\alpha$-Pinene & 28$\pm$3        &        & Low                & c        \\ 
          %          & X$\pm$3         &        & X                  & d        \\ 
          %          & 230$\pm$90      &        & High        & a        \\ 
          %          & 190$\pm$50      &        & High        & a        \\ 
          %          & 19              & 1 hour &              & b        \\ % Done
          %          & 210             &        & 1~ppbv        & e        \\
          %          & 70              &        & 0.1~ppbv      & e        \\
          %          $\beta$-Pinene  & 65$\pm$6        &        & Low           & c      \\ 
          %          & X$\pm$3         &        & X             & d      \\ 
          %          & 540$\pm$50      &        & High          & a     \\ 
          %          & 450$\pm$80      &        & High          & a      \\ 
          %          & 45              & 40 min &              & b      \\ % Done
          %          Methane 	      & 100             & 1 year  &             & b     \\ 
          %          Ethane          & 180             & 10 days &             & b     \\ 
          %          Propane         & 60              & 2 days  &             & b     \\ 
          %          Methylbutanol   & .13(per C)    & 1 hour  &             & b     \\ 
          %          HCHO            & 100             & 2 hour  &             & b     \\ 
          %          Acetone         & .67(per C)      & 10 days &             & b     \\ 
          %          Methanol        & 100             & 2 days  &             & b     \\ %Done
          %\item c \parencite{Lee2006}: Calculated through change in concentration of parent and product linear least squares regression. Estimates assume 20$^\circ$~C conditions.
          \bottomrule
        \end{tabular}
        \begin{tablenotes} % \item makes new lines
          \item a \textcite{AtkinsonArey2003}: Table 2, Yield from Isoprene reaction with OH, two values are from two referenced papers therein.
          \item b \textcite{Palmer2003}: lifetimes assume [OH] is 1e15 mol cm$^{-3}$.
          \item c \textcite{Wolfe2016}: ``prompt yield'': change in HCHO per change in isop$_0$.
          $[isop]_0=[isop]\exp(k_1[\mathrm{OH}]t)$; where $k_1$ is first order loss rate.
          Effectively relates HCHO abundance with isoprene emission strength.
          \item d \textcite{Dufour2009}: One-day yields from oxidation modelled by CHIMERE, using MCM reference scheme.
          %\item f Calculated using PTR-MS and iWAS on SENEX campaign data.
        \end{tablenotes}
        \label{BioIsop:method:tab_VOCLiteratureYields}
      \end{threeparttable} \end{table}
      
      %% NO$_2$ from the OMNO2d product provides daily mid-day measurements that we compare to output from GEOS-Chem NO$_2$ (see Section \ref{Model:GC:NOx}).
      %\textcite{Travis2016} show how NO$_x$ can be used to examine model bias in ozone (potentially due to NO$_2$ bias) over the USA.
      The effect of NO$_2$ on smearing can be seen in Figure \ref{BiogIsop:Method:smearing:fig_smearing_vs_nox}.
      This plot shows how smearing over Australia compares to satellite NO$_2$, with smearing distributions binned by NO$_2$ both with and without filtering for smearing.
      At lower NO$_2$ levels smearing is more frequently below the lower threshold. 
      These low values decrease in frequency, and have less affect on the mean $\hat{S}$ at around $5 \times 10^{14} $~molec cm$^{-2}$ NO$_2$.
      %This is an issue in Australia since NO$_2$ levels sit at roughly this threshold.
      %Due to the coarse resolution of the model, many regions with low NO$_2$ also have low isoprene emission, and this filtering does not remove many data points from biogenic-dominated regions.
      
      %% FIGURE SHOWS SMEARING BINNED BY NOX
      % Figure made in test_filters.smearing_vs_nox() 
      \mypic{Figures/OMI_link/Filters/smearing_nox_200501.png}{%
        Top: OMNO2d tropospheric NO$_2$ columns averaged into \lowhr ~horizontal bins for Jan, 2005.
        Bottom: Scatter plot of NO$_2$ against smearing calculations from GEOS-Chem ($\hat{S}$), with points above and below the smearing threshold range of 800-4600~s coloured red and blue respectively. 
        Points are binned by NO$_2$ with and without the smearing filter applied (orange and magenta respectively).
        Overplotted is the mean and standard deviation (error bars) within each bin. 
        Due to the logarithmic Y scale we only show the positive direction of standard deviations for unfiltered data.
        %Bottom left: Daily NO$_2$ scattered against smearing with (magenta) and without (orange) applying the smearing filter. This plot is a zoomed out version of the right panel.
      }{\label{BiogIsop:Method:smearing:fig_smearing_vs_nox}}  
  
\section{Relationship between temperature and HCHO}
  
  % Used to be subsubsection looking at hcho vs temp vs fire...
  Biogenic HCHO concentrations are correlated with temperature, as isoprene emissions are strongly correlated with temperature \parencite{Palmer2006, Zhu2013_poster, Surl2018}.
  Fires emit HCHO precursors and increase HCHO concentrations independently of the relationship between temperature and HCHO, and should be revealed as outliers when comparing HCHO to temperature.
  Figures \ref{Model:analysis:HCHO:fig_hcho_vs_temp_SEA_200501} - \ref{Model:analysis:HCHO:fig_hcho_vs_temp_SWA_200501} show the relationship between modelled temperature, and satellite HCHO for January 2005 within subsets of Australia.
  A reduced major axis regression is used to determine the correlation between surface temperature (X axis) and HCHO (Y axis).
  Using the natural log of HCHO we can take the linear regression and then exponentiate each side in the equation $\ln{Y} = m{X}+b$ to get ${Y} = \exp{m{X}+b}$. 
  This gives us the exponential fit as shown, with the correlation coefficient between $\ln{HCHO}$ and temperature.
  The distributions of exponential correlation coefficient and $m$ terms is shown in the embedded plot, with one data point available for each grid square where the regression is performed.
  These figures show that the modelled temperature is not well correlated with corrected recalculated OMI vertical columns (r ranges from -0.24 to 0.47), but is with modelled columns (r ranges from 0.49 to 0.84).
  Correlations between modelled temperatures and HCHO are further improved when using the spatial average within each region.
  Furthermore the relationship is improved in individual grid squares over south Eastern Australia by removing non-biogenic emissions from the model (r increased from 0.58 to 0.75).
  This improvement is not seen in Northern Australia, nor south Western Australia.
  Overall this suggests that modelled correlations between temperature and HCHO are spatially dependent, and generally well reflected in satellite measurements over Australia.
  
  % Figure produced in GC_tests.HCHO_vs_temp
  \begin{figure}
    \includegraphics[width=\textwidth]{Figures/OMI_link/GC/HCHO_vs_temp_SEA_20050101-20050228.png}
    \caption{%
      Top row (left): surface temperature averaged over January and February 2005.
      Top row (right): correlation between spatially averaged GEOS-Chem temperatures and recalculated satellite vertical columns.
      Second row: GEOS-Chem surface temperatures correlated against GEOS-Chem HCHO, with different colours for each grid box, and black showing the spatially averaged correlation over time.
      Third row: as second row, except GEOS-Chem HCHO comes from the biogenic emissions only simulation.
      A reduced major axis regression is used within each gridbox using daily overpass time surface temperature and HCHO.
      The distribution of slopes (solid) and regression correlation coefficients (dashed) for the exponential regressions is shown in the inset panels in rows 2 and 3.
    }
    \label{Model:analysis:HCHO:fig_hcho_vs_temp_SEA_200501}
  \end{figure}
  
  % Figure from ???
  \begin{figure}
    \includegraphics[width=\textwidth]{Figures/OMI_link/GC/HCHO_vs_temp_NA_20050101-20050228.png}
    \caption{%
      As Figure \ref{Model:analysis:HCHO:fig_hcho_vs_temp_SEA_200501} but for northern Australia.
    }
    \label{Model:analysis:HCHO:fig_hcho_vs_temp_NA_200501}
  \end{figure}
  
  \begin{figure}
    \includegraphics[width=\textwidth]{Figures/OMI_link/GC/HCHO_vs_temp_SWA_20050101-20050228.png}
    \caption{%
      As Figure \ref{Model:analysis:HCHO:fig_hcho_vs_temp_SEA_200501} but for south-western Australia.
      TODO: Fix y axis in subplot 322
    }
    \label{Model:analysis:HCHO:fig_hcho_vs_temp_SWA_200501}
  \end{figure}
  
  One problem with detecting outliers in the temperature and enhanced HCHO relationship is that days when fires occur are likely to be hot.
  Another problem with correlating heat and HCHO is that increased temperature accelerates HCHO destruction \parencite{Zheng2015}.
  We test the fire mask (see prior Section \ref{Model:filter:fire}) by examining the relationship between modelled temperature and satellite HCHO with and without applying the filters for smoke and active fires.
  %Figure \ref{Model:filter:fire:fig_VCC_vs_GC_temperature} show the regressions between OMI HCHO total columns and temperature from GEOS-Chem output and CPC daily maximum temperatures.
  Figure \ref{Model:filter:fire:fig_HCHO_vs_temperature} show the regressions between midday surface HCHO and temperature, where two independent data sets are used for the temperature: one from GEOS-Chem output and the other from CPC daily maxima.
  Regressions over both the grid square containing Sydney and averaged over a wider area are reduced in quality when after applying the pyrogenic filter. 
  This could be due to the low resolution of available GEOS-Chem midday output, which greatly increases the strictness of the fire filter, which originally is created at \highhr ~horizontal resolution but here is widened out to \lowhr.
  While the modelled correlation between surface HCHO and temperature is quite strong ($r>0.8$) the correlation coefficient is reduced while the slope (temperature coefficient) is changed by more than 30\% when applying the pyrogenic filter at the model resolution.
  If we use recalculated satellite HCHO columns instead of modelled midday outputs, then higher horizontal resolution can be achieved; however, this is at the cost of vertical resolution.
  Figure \ref{Model:filter:fire:fig_VCC_vs_temperature} shows the relationship between total column HCHO and temperature, which is not as strong as that shown between surface HCHO and temperature.
  At this resolution the application of the pyrogenic filter is shown to slightly strengthen the correlation over the wider south eastern Australian region, with r increasing from $\sim{0.31}$ to $\sim{0.38}$.
  In this thesis the pyrogenic filter is applied at the higher horizontal resolution (\highhr), and this analysis suggests that the filter should strengthen the relationship between total column HCHO and its biogenic precursors over Australia.
  
  %Figure from chapter_2_modelling.HCHO_vs_temperature
  \mypic{Figures/OMI_link/Filters/HCHO_vs_temp_surface_20051201-20060228.png}{%
    Surface HCHO from GEOS-Chem overpass output (midday) on the Y axis vs surface temperatures at midday ($T_{GC}$) and vs maximum daily temperatures from the CPC data set ($T_{CPC}$).
    Top row: Sydney grid square scatter plot and regression with one data point for each day in the summer of 2005-2006.
    Bottom row: as top row except averaging over several grid boxes covering south eastern Australia (SEA: 37\degr S to 29\degr S, 146\degr E to 153.5\degr W).
    Grid squares with pyrogenic influence detected are removed (prior to any averaging) in the right column. 
  }{\label{Model:filter:fire:fig_HCHO_vs_temperature}}
  
  %Figure from chapter_2_modelling.HCHO_vs_temperature
  \mypic{Figures/OMI_link/Filters/HCHO_vs_temp_20051201-20060228.png}{%
    OMI recalculated vertical columns of HCHO on the Y axis vs surface temperatures at midday ($T_{GC}$) and vs maximum daily temperatures from the CPC data set ($T_{CPC}$).
    Top row: Sydney grid square scatter plot and regression with one data point for each day in the summer of 2005-2006.
    Bottom row: as top row except averaging over several grid boxes covering south eastern Australia (SEA: 37\degr S to 29\degr S, 146\degr E to 153.5\degr W).
    Grid squares with pyrogenic influence detected are removed (prior to any averaging) in the right column. 
  }{\label{Model:filter:fire:fig_VCC_vs_temperature}}
  
  
\section{Process schematic}
  \label{Model:saving_omhchorp}
  % final summary of omhchorp creation
  Figure \ref{Model:saving_omhchorp:fig_flow_omhchorp} shows an overview of how vertical columns (modelled and measured) along with filters and sundry data are created in this thesis.
  As described in this chapter, satellite and modelled data are combined combined to form recalculated corrected vertical columns (VCC), with masks created for potential anthropogenic, pyrogenic, and smearing influence.
  The end product is a gridded data file that contains new and old satellite and modelled vertical columns, along with sundry data to allow uncertainty calculation and other analyses.
  The original and recalculated AMF are also binned and stored.
  
  % made on www.draw.io 
  \mypic{Figures/Flow_Making_omhchorp.png}{%
    Flow diagram showing how OMHCHO level two swath data is read, processed, and gridded in this thesis.
  }{\label{Model:saving_omhchorp:fig_flow_omhchorp}}
     
\section{Data Access}
\label{Model:DataAccess}
\begin{description}
  \item[AIRS] Atmospheric Infra-red Sounder instrument aboard the Aqua satellite \parencite{AIRS3STD}.
  Satellite measurements of carbon monoxide downloaded from \url{https://search.earthdata.nasa.gov} with the product name AIRS3STD, DOI: 10.5067/AQUA/AIRS/DATA303.
  Data from this product is used in Chapter \ref{Ozone}, and briefly described in Section \ref{Model:datasets:AIRS}.
  
  \item[CPC] Climate Prediction Center global temperature data provided by the NOAA/OAR/ESRL PSD, Boulder, Colorado, USA, from their website at \url{https://www.esrl.noaa.gov/psd/}.
  Surface temperatures from this data set used in Section \ref{Model:omiRecalc:diff}, and described in Section \ref{Model:datasets:model:CPC}.
  
  \item[ERAI] European Centre for Medium-range Weather Forecasts (ECMWF) Interim Reanalysis (ERA-I) \parencite{Dee2011}. 
  Downloaded using the online portal: \url{https://apps.ecmwf.int/datasets/data/interim-full-daily/}.
  This data set is used in Chapter \ref{Ozone}
  
  \item[MUMBA] Measurements of Urban, Marine and Biogenic Air campaign \parencite{PatonWalsh2017}.
  Several trace gases from this campaign are examined here in Section \ref{BioIsop:results:measurements}, the data set is described in more detail in Section \ref{Model:datasets:MUMBA}.
  
  
  \item[OMHCHO] Satellite swaths of HCHO slant columns downloaded from \url{https://search.earthdata.nasa.gov}, with DOI 10.5067/Aura/OMI/DATA2015.
  More information can be found in Section \ref{Model:omhcho}.
  
  \item[OMNO2d] Daily satellite NO$_2$ product downloaded from \url{https://search.earthdata.nasa.gov/search}, DOI:10.5067/Aura/OMI/DATA3007. 
  For more information in refer to section \ref{Model:datasets:OMNO2d}.
  
  \item[OMAERUVd] Gridded satellite based AAOD measurements downloaded from the NASA earth data portal \url{https://search.earthdata.nasa.gov}.
  A summary can be found at \url{https://disc.gsfc.nasa.gov/datasets/OMAERUVd_V003/summary}, \fullcite{Omar2008}.
  This dataset is described in detail in Section \ref{Model:datasets:OMAERUVd}.
  
  %\item[SPEI] Monthly standardised precipitation evapotranspiration index (metric to determine drought stress) downloaded from \url{http://hdl.handle.net/10261/153475} with DOI:10.20350/digitalCSIC/8508.
  %See more information in section ?? THIS WORK IS FUTURE WORK
  
  \item[SPS] Measurements of trace gases relevant to air quality in Western Sydney, Australia, from May 2016 to September 2017 as part of the Western Air Shed and Particulate Study for Sydney \parencite{SPS2017}. 
  Available from PANGAEA, \url{https://doi.org/10.1594/PANGAEA.884317}.
  Some trace gas measurements are examined here in Section \ref{BioIsop:results:measurements}.
  Relevant measurements described in detail in Section \ref{Model:datasets:SPS}.
  
  \item[Wollongong FTIR] The instrument is part of the Network for the Detection of Atmospheric Composition Change (NDACC) and data can be retrieved from the NDACC database (\url{http://www.ndaccdemo.org/stations/wollongong-australia}).
  The current principle investigator producing and quality assuring the data set is \href{mailto:njones@uow.edu.au}{Dr. Nicholas Jones}.
  
  \item[Ozone sondes] Ozonesonde data was retrieved from the World Ozone and Ultraviolet Data Centre (WOUDC)  \url{http://woudc.org/data/explore.php}.
  
  
  
\end{description}