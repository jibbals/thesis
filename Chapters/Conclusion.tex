\chapter{Conclusions} % Chapter Title
\label{Conclusions}

TODO: Wrap up aims and answer each one

% The two largest contributors to tropospheric ozone concentrations are chemical production (driven by precursor emissions) and stratospheric transport.
% I aim to improve understanding of both of these sources using existing satellite and ground-based datasets along with GEOS-Chem modelled outputs.
\textbf{In this thesis I aim to improve understanding of natural contributions to ozone over Australia and the southern ocean.}
Chapter \fullref{BioIsop} shows how models are misrepresenting isoprene emissions by a large margin in Australia. 
Additionally how models respond to update emissions how these top-down estimates are sensitive to parameters such as HCHO yields is examined.



% % Aim for Chapter 2: Modelling examination
% Estimation of BVOC emissions in Australia can be improved through satellite measurements of one of the primary oxidation products HCHO.
% Satellites which overpass daily record reflected solar (and emitted terrestrial) radiation, and give us measurements over all of Australia.
% Combining satellite data with model outcomes provides a platform for the understanding of natural processes, which are uncertain over Australia.
% Satellite measurements use modelled a priori vertical profiles of HCHO to estimate total column amounts.
\textbf{I aim to recalculate satellite vertical columns of HCHO using updated model a priori information.}
% In this effort I aim to improving the understanding of the importance of relevant parameters (within GEOS-Chem) in calculating vertical columns of HCHO measured by satellite.
% This includes an examination of how well GEOS-Chem simulates several species such as NO$_X$, isoprene, and HCHO compared to both in-situ and remote measurement data that exists for Australia.
% Additionally I detail the construction and effects of satellite data filters.
% The work towawrds this aim is in Chapter \ref{Model}.
% %Soil moisture plays an important role in VOC emissions, as trees under stress may stop emitting various chemicals. 
% %This is especially true for Australia due to frequent droughts and wildfires.
% %The argument for improved understanding of land surface properties, specifically soil moisture, is an old one\parencite{Mintz1982, Rowntree1983, Chen2001}.

 
% % Aim for Chapter 3: Biogenic emissions estimation
% The technique of determining isoprene emissions from satellite detected HCHO is called satellite inversion.
\textbf{I aim to determine isoprene emissions in Australia using a top-down inversion of satellite HCHO, through an estimated yield from isoprene to HCHO}
% HCHO amounts and the yield of isoprene to HCHO over Australia is required to create top-down estimates.
% This process also requires careful examination of when the assumptions required within the inversion process are not valid.
% Due to the low availability of in-situ data over most of the Australian continent, a combination of modelled and satellite data could reduce the uncertainties of isoprene emissions from Australian landscapes.
% Improved emissions estimates will in turn improve the accuracy of CTMs, providing better predictions of atmospheric composition and its response to ongoing environmental change.
% The work towards fulfilling this aim is in Chapter \ref{BioIsop}.
The uncertainty and limitations of top-down estimates due to satellite and model uncertainty, along with temporal and horizontal resolution of available data is likewise studied
 
% % Aim for Chapter 4: STT Ozone
% \textbf{To improve understanding of ozone transported to the troposphere from the stratosphere in Australia and the southern ocean.}
% Stratospheric transport is the second largest driver of tropospheric ozone concentrations, and an improved understanding of transported ozone can be determined from ozonesonde measurements.
% Ozonesondes provide a glimpse of the vertical ozone profile up to $\sim 30$~km, and we use a Fourier filter to determine how often stratospheric transport is occurring at three sites: Melbourne, Macquarie Island, and Davis Station. 
% Combining transport event frequency analysis with modelled ozone distributions is used to derive a new method of detection and quantification of transported ozone in Chapter \ref{Ozone}.
% 
% 
% % Chapter 5: Conclusions?
% \textbf{I aim to describe relative importance of sources of tropospheric ozone in Australia, as well as seasonality.}
% I will describe how modelled ozone is affected by updated isoprene emissions, comparing changes in GEOS-Chem outputs.
% Trends of isoprene emissions and their relationship to tropospheric ozone trends could provide new insight into the future of tropospheric ozone in Australia.

\section{Ozone over Australia}
\label{Conclusions:ozone}
  
  Most affected by biogenics, secondarily by stratospheric transport.
  TODO: How these are changing over time

\section{Isoprene link to ozone}
\label{Conclusions:isoprene}

  TODO: how isoprene is linked to ozone in modelling

  TODO: how isoprene is linked to ozone in measurement datasets
  
  TODO: How isoprene oxidation cycle is linked to trop ozone
  
\section{Current trends}
\label{Conclusions:trends}

  TODO: Summary of how the VOC NOX Ozone dynamic is changing over time, including how they are linked
  
  TODO: how HCHO is changing over the period

  TODO: how isoprene emissions are changing over the period
  
  TODO: how NOx emissions are changing
  
  TODO: how ozone is changing over the period
  
\section{Potential future work}
\label{Conclusions:future}
  
  % Soil moisture improvement
  One of the important parameters in Australia is the soil moisture activity factor($\gamma_{SM}$), which can have large regional effects on the isoprene emissions \parencite{Sindelarova2014,Bauwens2016}.
  Generally if soil moisture is too low, isoprene emissions stop \parencite{Pegoraro2004,Niinemets2010}, however in many Australian regions the plants may be more adapted to lower moisture levels. (TODO: Find cites for this - talk from K Emerson at Stanley indicated this)
  GEOS-Chem runs MEGANv2.1, which has three possible states for isoprene emissions based on the soil moisture ($\theta$):
  \begin{align*}
  \gamma_\mathrm{SM} & = 1 && \theta > \theta_1 \\
  \gamma_\mathrm{SM} & = (\theta-\theta_w)/\Delta\theta_1  && \theta_w < \theta < \theta_1 \\
  \gamma_\mathrm{SM} & = 0 && \theta < \theta_w \\
  \end{align*}
  where $\theta_w$ is the wilting point, and $\theta_1$ determines when plants are near the wilting point.
  The wilting point is set by a land based database from \textcite{Chen2001}, while $\theta_1$ is set globally based on \textcite{Pegoraro2004}.
  Potentially importantly, these moisture states are disabled in GEOS-Chem V10.01, which is partly because accurate maps of soil moisture are not available.
  
  Improved isoprene emissions modelling requires this soil moisture problem to be handled.
  Simply enabling the parameter in its current form is not quite good enough for Australia, due to both the unknown soil moisture and the poorly understood plant responses in this country.
  TODO: cite the paper which could update MEGAN soil moisture parameterisation.
  Using satellite data to improve isoprene emission estimates such as done in this thesis must be used to analyse model improvements, since fully independent measurements are lacking.
  Emissions measurement would be greatly valuable however they remain expensive and difficult, especially over the large sparse environment which makes up the Australia outback.