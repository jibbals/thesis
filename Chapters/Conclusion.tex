\chapter{Summary and Concluding remarks} % Chapter Title
\label{Conclusions}

  % The two largest contributors to tropospheric ozone concentrations are chemical production (driven by precursor emissions) and stratospheric transport.
  % I aim to improve understanding of both of these sources using existing satellite and ground-based datasets along with GEOS-Chem modelled outputs.
  \textbf{In this thesis I aimed to improve understanding of natural contributions to ozone over Australia and the southern ocean.}
  Chapter \fullref{LR} examined processes leading to ozone creation in the troposphere, primarily isoprene emissions and subsequent chemistry.
  The secondary cause of ozone, stratosphere to troposphere transport (STT), is also outlined with associated processes and causes discussed.
  A summary of how the lack of information available over Australia affects modelling and forecasting is provided.
  Methodologies, tools and data-sets used throughout the thesis are detailed in Chapter \fullref{Model}.
  Chapter \fullref{BioIsop} showed how models are misrepresenting isoprene emissions by a large margin in Australia, along with the description and implementation of a relatively simple method of improvement.
  Chapter \fullref{Ozone} provided an estimate of stratospheric ozone influx to the troposphere, along with potential classifications and seasonality.

% Split by aim/chapter?
\section{Outcomes}
\label{Conclusions:aims}
  
  % % Aim for Chapter 2: Modelling examination
  \textbf{I aim to recalculate satellite vertical columns of HCHO using updated model a priori information.}
  HCHO from OMI on board the AURA satellite %(see Section \ref{Model:omhcho}) is read pixel by pixel and 
  is examined and recalculated in Chapter \ref{Model}, where influence of the air mass factor (AMF) is discussed in detail. %(Section \ref{Model:AMF}).
  This AMF was created using an older version of GEOS-Chem with out-of-date HCHO chemistry.
  Recalculation, binning, and analysis of the satellite HCHO vertical columns is performed using GEOS-Chem v10.01, outlined in Section \ref{Model:omiRecalc:outline}.
  This is performed using my own partial AMF recalculation code, along with full recalculation code set up in collaboration with Prof. Palmer and Dr. Surl.
  Subsequent use of the data makes further use of the recalculation as a basic sensitivity analysis of top-down emissions estimations to a priori satellite information.
   
  % % Aim for Chapter 3: Biogenic emissions estimation
  \textbf{I aim to determine biogenic isoprene emissions in Australia using a top-down inversion of satellite HCHO, through an estimated yield from GEOS-Chem.}
  In Chapter \ref{BioIsop} I determine the linear relationship between total column HCHO and biogenic isoprene emissions over Australia using the global CTM GEOS-Chem.
  Applying this relationship to satellite based HCHO measurements creates the desired top-down isoprene emissions estimate.
  This process is described in Section \ref{BioIsop:method}, and requires intensive filtering of both satellite data (to exclude non-biogenic HCHO sources) and model yield (to minimise spatial smearing).
  The uncertainty and limitations of top-down estimates due to satellite and model uncertainty, along with temporal and horizontal resolution of available data is examined in Section \ref{BioIsop:uncertainty}.
  Furthermore, this top-down estimation is used to scale a new simulation of HCHO, and O$_3$ over Australia, described in Section \ref{BioIsop:method:scaled}.
   
  % % Aim for Chapter 4: STT Ozone
  \textbf{I aim to improve understanding of ozone transported to the troposphere from the stratosphere in Australia and the southern ocean.}
  % Stratospheric transport is the second largest driver of tropospheric ozone concentrations, and an improved understanding of transported ozone can be determined from ozonesonde measurements.
  % Ozonesondes provide a glimpse of the vertical ozone profile up to $\sim 30$~km, and we use a Fourier filter to determine how often stratospheric transport is occurring at three sites: Melbourne, Macquarie Island, and Davis Station. 
  % Combining transport event frequency analysis with modelled ozone distributions is used to derive a new method of detection and quantification of transported ozone in Chapter \ref{Ozone}.
  In Chapter \ref{Ozone} the seasonal cycle of STT events is characterised, and their contribution to the SH extra-tropical tropospheric ozone budget is quantified using GEOS-Chem to estimate ozone flux extrapolated from three measurement sites.
  Causal climatology and event seasonality are examined in Section \ref{Ozone:eventclimatologies}.
  STT detection frequencies and modelled tropospheric ozone columns are used to estimate STT ozone flux near three sites in Section \ref{Ozone:STTevents}. 
  Findings are compared against relevant literature, and %(in Section \ref{Ozone:sensitivity}) 
  the uncertainties involved in STT event detection and ozone flux estimation are studied.
  
  % 
  % 
  % % Chapter 5: Conclusions?
  % \textbf{I aim to describe relative importance of sources of tropospheric ozone in Australia, as well as seasonality.}
  % I will describe how modelled ozone is affected by updated isoprene emissions, comparing changes in GEOS-Chem outputs.
  % Trends of isoprene emissions and their relationship to tropospheric ozone trends could provide new insight into the future of tropospheric ozone in Australia.

  In the following sections, summaries of results over Australia are often split into regions shown in Figure \ref{Conclusions:fig_subregions}.
  
  % figure from Analysis_E_isop.py.show_subregions()
  \mypic{Figures/OMI_link/subregions.png}{%
    Sub-regions used in subsequent figures.
    Australia-wide averages will be black or grey, while results from within the coloured rectangles will match the colours shown here.
  }{\label{Conclusions:fig_subregions}}

\section{Isoprene emissions}
\label{Conclusions:isoprene}

  % Compare overall isoprene apriori vs apostiori
  Being a major driver of continental boundary ozone production, a new isoprene emissions estimate is created using OMI satellite measurements of HCHO.
  This estimate is compared against the bottom-up estimate from GEOS-Chem (running MEGAN).
  % Taken from isoprene chapter:
  Generally months outside of May to August show the a postiori lower than the a priori, except in the south eastern portion of Australia where only February is notably reduced.
  MEGAN isoprene emissions appear to be overestimated in Austral summer, suggesting poorly understood emission factors for Australian forests.
  % TODO which regions are least/most overestimated?

\section{Ozone over Australia}
\label{Conclusions:ozone}
  
  Ozone production in the troposphere is a complex process involving various compounds.
  Tropospheric ozone is enhanced through VOC chemistry, stratospheric transport, and pollution.
  The first two of these processes are highly uncertain, with few studies performed on either topic within Australia.
  Of these ozone sources, VOC chemistry uncertainty is dominated by poor understanding of biogenic emissions (mainly) of isoprene.
  Emissions of isoprene are globally modelled at $\sim$465-500\tgcpyr  ~\parencite{Guenther2006, Messina2016}. 
  However these appear to be overestimated in Australia, as seen here with top-down estimates at $\sim 19$ \tgcpyr  ~compared to $\sim 43$ \tgcpyr  ~using GEOS-Chem (which implements MEGAN).
  
  TODO: ozone concentrations using emissions from MEGAN/TOP-Down, also how these compare to estimated STT:
  Tropospheric ozone production is estimated using GEOS-Chem before and after scaling isoprene emissions based on top-down satellite data.
  Figure \ref{Conclusions:ozone:fig_ozone_surf_summer} shows Australian summer surface level ozone with and without scaling isoprene emissions to match the multi-year top-down estimation.
  Tropospheric column ozone reduces very minimally ($<1\%$), however surface ozone drops $\sim 5\%$.
  This suggests that impacts from reduced isoprene emission are mostly important at the surface, and outweighed by other processes at higher altitudes.
  The results are tabulated for each of five regions, and all of Australia, and shown here in Table \ref{Conclusions:ozone:tab_emissions_vs_ozone}.
  Ozone production from isoprene emissions form an extra layer on top of anthropogenic ozone. 
  A 5\% mean difference in surface ozone levels will directly affect exposure levels and impacts air quality estimates. 
  % TODO read through ozone excedences in sydney? state of environment reports (2016) just for noting importance
  
  % UP TO HERE Minor comments Jenny
  Ozone transported from the stratosphere also may have some affect on surface ozone levels, although this is hard to detect with the methods used in this thesis.
  We only compare the affect of the two disparate influences on the tropospheric ozone column.
  STT analysis over Melbourne suggests up to $\sim 10\%$ of the tropospheric column is due to stratospheric influx, with the average increase caused by influx being $\sim 1$ to $3.5\%$.
  We see that a relatively large drop in isoprene emissions causes $\sim 1\%$ reduced ozone concentration over summer.
  Both of these sources become more important in summer, with more frequent STT and stronger isoprene emissions.
  However the impacts are separated by altitude with most STT occurring in the free troposphere, while isoprene emission impacts are largely seen at the surface.
  %Total uncertainty (shaded) is on the order of $100\%$ (see Sect. \ref{Ozone:fluxuncertainty}). 
  %We calculate the annual amount based on the sum of the monthly values.  
  %The regions over Davis, Macquarie Island, and Melbourne have estimated STT ozone contributions of $\sim 5.7 \times 10^{17}$, $\sim 5.7 \times 10^{17}$, and $\sim 8.7 \times 10^{17}$ molecules cm$^{-2}$ a$^{-1}$ respectively, or equivalently $\sim 2.0$, 2.1, and 3.3~Tg a$^{-1}$.
  
  \mypic{Figures/OMI_link/new_emiss/O3_surf_map_JanFeb05.png}{%
    Ozone maps before (left) and after (right) scaling isoprene emissions in GEOS-Chem for summer.
    The bottom panel shows the linear regression between the runs along with a black dashed line representing the 1-1 ratio.%
    }{\label{Conclusions:ozone:fig_ozone_surf_summer}}
  
  %TODO: UPDATE: CURRENTLY BASED ON JUST 2005 instead of 2005-2012 mya
  % TODO UPDATE TO USE SURFACE PPB AVERAGES
  %Ozone currently output from tests/test_new_emissions.hcho_ozone_series()
  \begin{table}\begin{threeparttable}
    \caption{Isoprene emissions from MEGAN and top-down estimation in Tg a$^{-1}$, along with ozone tropospheric column amounts in O$_3$ cm$^{-2} \times 10^{17}$ .}
    \begin{tabular}{ l c c c c c c } 
      \toprule
      Metric & AUS & SEA & NEA & NA & SWA & MID \\
      \midrule
      MEGAN & 43(2) & blah &  &  & & \\
      Ozone & 9.70 & 11.17 & 11.03 & 11.19 & 11.69 & 9.09 \\
      \midrule
      Top-Down & 19(2) & & & & & \\
      Ozone & 9.64 & 11.11 & 10.99 & 11.12 & 11.63 & 9.02 \\
      \bottomrule
    \end{tabular}
    \begin{tablenotes}
      \item Standard deviations shown in parenthesis.
    \end{tablenotes}
    \label{Conclusions:ozone:tab_emissions_vs_ozone}
  \end{threeparttable}\end{table}

  
  TODO: how isoprene is linked to ozone in modelling 
  Figure \ref{Conclusions:ozone:fig_new_emiss_series_O3} shows surface level (up to $\sim 150$~m altitude) ozone concentration over 2005 before and after scaling modelled isoprene emissions.
  Reducing isoprene emissions lowers surface ozone concentrations by TODO: XX to YY \% in summer, and XX to YY \% in winter.
  The direct correlation between reduced emissions and surface ozone (Figure TODO) is more or less clear monthly, and using multiple years enhances/reduces the relationship.
  
  % plot from tests/hcho_ozone_series()
  \mypic{Figures/OMI_link/new_emiss/O3_surface_20050101_20051231.png}{%
    Surface ozone concentrations (ppb) per region over 2005.}{\label{Conclusions:ozone:fig_new_emiss_series_O3}}

  
\section{Outputs and future work}
\label{Conclusions:future}
  
  % may need to include \usepackage{bibentry}
  Analysis of STT along with an estimate of ozone flux from the stratosphere over a portion of Australia and the southern ocean is published in \textcite{Greenslade2017}.
  Ozone production using GEOS-Chem with updated top-down isoprene emissions is examined in this thesis and may form part of a new publication in the near future with the aim of updating MEGAN isoprene emission factors in Australia.
  
  % Soil moisture improvement
  One of the important parameters in Australia is the soil moisture activity factor($\gamma_{SM}$), which can have large regional effects on the isoprene emissions \parencite{Sindelarova2014,Bauwens2016}.
  Generally if soil moisture is too low, isoprene emissions stop \parencite{Pegoraro2004,Niinemets2010}, however in many Australian regions the plants may be more adapted to lower moisture levels. (TODO: Find cites for this - talk from K Emerson at Stanley indicated this)
  GEOS-Chem runs MEGANv2.1, which has three possible states for isoprene emissions based on the soil moisture ($\theta$):
  \begin{align*}
  \gamma_\mathrm{SM} & = 1 && \theta > \theta_1 \\
  \gamma_\mathrm{SM} & = (\theta-\theta_w)/\Delta\theta_1  && \theta_w < \theta < \theta_1 \\
  \gamma_\mathrm{SM} & = 0 && \theta < \theta_w \\
  \end{align*}
  where $\theta_w$ is the wilting point, and $\theta_1$ determines when plants are near the wilting point.
  The wilting point is set by a land based database from \textcite{Chen2001}, while $\theta_1$ is set globally based on \textcite{Pegoraro2004}.
  These moisture states are disabled in GEOS-Chem V10.01.
  Improved isoprene emissions modelling requires this soil moisture problem to be handled.
  Simply enabling the parameter in its current form is not quite good enough for Australia, due to both the unknown soil moisture and the poorly understood plant responses in this country.
  TODO: cite the paper which could update MEGAN soil moisture parameterisation.
  
  Using satellite data to improve isoprene emission estimates such as done in this thesis must be used to analyse model improvements, since fully independent measurements are lacking.
  Measurements of isoprene emissions in Australia would provide a valuable opportunity to verify improvements in emission estimates.
  Emissions measurement would be greatly valuable however they remain expensive and difficult, especially over the large sparse environment which makes up the Australia outback.