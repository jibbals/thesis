% CHAPTER 3 (probably)
% Isoprene Emissions

\chapter{Isoprene Emissions in Australia} % Main chapter title
\label{ch_isop} %better reference name like ch_isop

\section{GEOS-Chem isoprene mechanisms}
\label{ch_isop:sec:GEOSChemMechanisms}
  \subsection{Outline}
    The isoprene reactions simulated by GEOS-Chem were originally based on (TODO: read+citet: Horowitz1998) (TODO: check wiki that this is true also).
    This involved blahblah(read that paper).
    The mechanism was subsequently updated by \citet{Mao2013}, who change the isoprene nitrates yields and add products based on current understanding as laid out in (TODO: read abstract of these and cite: Paulot2009a,Paulot2009b).
    The full current mechanism is described online at \url{http://wiki.seas.harvard.edu/geos-chem/index.php/New_isoprene_scheme} (TODO: check this).
    

%----------------------------------------------------------------------------------------
%	SECTION
%----------------------------------------------------------------------------------------
\section{Isoprene emissions estimation}
\label{ch_isop:sec:IsopreneEmissions}

  \subsection{HCHO Products and yield}
    Australian forests are strong emitters of both isoprene and monoterpenes, which go on to form various products including but not limited to secondary organic aerosols, oxygenated VOCs (OVOCs), ozone, OH, and HO$_2$.
    This production occurs over several steps, yields are often classed into at least two categories.
    First generation yield refers to the amount of HCHO produced per unit isoprene consumed by initial oxidation, total yield (sometimes molar yield) refers to time dependent yield of HCHO over multiple oxidation stages \citep{Wolfe2016}.
    \citet{Wolfe2016} define prompt yield as the change in formaldehyde measurement per unit change in initial isoprene emissions.
    Some argue that isoprene emissions are overestimated, due to the fact that they are based on relatively few measurements of isoprene emission factors \citep{Winters2009, FortemsCheiney2012} TODO: read and cite paper mentioned in Fortems.
    
    Isoprene production of HCHO depends on several factors, importantly NO$_X$ levels have a direct effect on the fate of VOCs in the atmosphere.
    At higher NO mixing ratios (at least a few hundred pptv), organic peroxy radicals (RO$_2$) react mostly with NO. 
    At low NO (less than 10's of pptv), reaction with HO$_2$, other RO$_2$, and isomerization dominate the fate of RO$_2$.
    In low NO$_X$ environments, reported HCHO yields from isoprene are from XtoY\%, while in high NO$_X$ environments this value is XtoY\% TODO: these values from table.
    For monoterpenes the yields are around X, Y\% for low, high NO$_X$ respectively.
    Emissions and yields for various species including some terpenes can be seen in table \ref{ch_HCHO:tab:VOCAusYields}.
    \citet{Wolfe2016} determine that going from NO$_X = 0.1$ to $2.0$ ppbv triples the prompt yield of HCHO, from 0.3 to 0.9 ppbv ppbv$^{-1}$ due to isoprene, while the background HCHO doubles.
    (TODO:and finish Wolfe2016 discussion paper for yields)
    TODO:go through atkinsonarey2003
    
    Many of the HCHO yields from terpenoids are estimated through chamber studies which examine the products molecular mass and charge after mixing the compound of choice into a known volume of air.
    These conditions generally don't match those of the real world, where ambient air will have a cocktail of these compounds as well as various reactants.
    
    A proton transfer reaction mass spectrometer (PTR-MS) can be used to determine gas phase evolution of terpene oxidation products.
    This is done through analysis of mass to charge ratios ($m/z$) which can be used identify chemical compounds.
    
    Looking at Australian emissions from running GEOS-Chem and using yields provided by XYZ (another table)
    TODO: Fill out this table
    \begin{table}
    \begin{tabular}{ | c  c  c  c  c | }
      \hline
      \textbf{Species} & \textbf{Emissions$^a$} & \textbf{Lifetime$^b$} & \textbf{HCHO Yield$^c$} & \textbf{HCHO production$^d$\%}
      \\               & (Tg C per month)       &                       & (per C reacted)         &         \\ \hline
      Isoprene           & Y                     & n minutes            & 0.x                & 10       \\
      $\alpha$-Pinene    & Y                     & n minutes            & 0.x                & 10        \\ %\hline
      $\beta$-Pinene     & Y                     & n minutes            & 0.x                & 10        \\
      HCHO               & Y                     & n minutes            & 1.0                & 10         \\ \hline
    \end{tabular}
    \caption{HCHO yields from various species averaged over Australia during Summer. \hspace{\textwidth} \\ 
    ${}^a$Calculated using GEOS-Chem emissions over Australia in January 2005. \hspace{\textwidth} \\ 
    ${}^b$ \hspace{\textwidth} \\ 
    ${}^c$ \hspace{\textwidth} \\ 
    ${}^c$Production determined by dividing emission*yield by the sum of all VOC emissions*yields. \hspace{\textwidth} \\ 
    }
    \label{ch_HCHO:tab:VOCAusYields}
    \end{table}
    
    % yields from Atkinsen2003
    %isoprene
    %0.63 0.10 Tuazon and Atkinson (1990a)
    %0.57 0.06 Miyoshi et al. (1994)
    % a-pinene
    %0.23 0.09 Noziere et al. (1999a)
    %0.19 0.05 Orlando et al. (2000)
    % b-pinene
    %0.54 0.05 Hatakeyama et al. (1991)
    %0.45 0.08 Orlando et al. (2000)
    
    Yields table looking at literature provided yields of HCHO.
    % molar HCHO yield per unit carbon equal to HCHO molar percent yield(per carbon)? or some conversion?
    % TODO: ask steve
    
    \begin{table}
    \begin{tabular}{ | c | c | c | c |  }
      \hline
      \textbf{Species}  & \textbf{HCHO Yield} & Notes & Source
      \\                    & (molar \% unless specified) & &       \\ \hline
      Isoprene          & 63$\pm$10     & High NO$_X$ & \citep{AtkinsonArey2003}      \\
                           & 57$\pm$6       & High NO$_X$ & \citep{AtkinsonArey2003}      \\
                           & .45(per C)       & High NO$_X$ & \citep{Palmer2003}               \\
                           & .30(per C)       & Low  NO$_X$ & \citep{Palmer2003}               \\
      $\alpha$-Pinene & 28$\pm$3      & 20$^\circ$~C, Low NO$_X$ & \citep{Lee2006}$^b$      \\ 
                            & X$\pm$3       & X$^\circ$~C, X NO$_X$ & \citep{Wolfe2016}      \\ 
                            & 23$\pm$9      & High NO$_X$ & \citep{AtkinsonArey2003}      \\ 
                            & 19$\pm$5      & High NO$_X$ & \citep{AtkinsonArey2003}      \\ 
                            & .019(per C)      & 1 hour lifetime & \citep{Palmer2003}               \\
      $\beta$-Pinene & 65$\pm$3       & 20$^\circ$~C, Low NO$_X$ & \citep{Lee2006}      \\ 
                            & X$\pm$3       & X$^\circ$~C, X NO$_X$ & \citep{Wolfe2016}      \\ 
                            & 54$\pm$5      & High NO$_X$ & \citep{AtkinsonArey2003}      \\ 
                            & 45$\pm$8      & High NO$_X$ & \citep{AtkinsonArey2003}      \\ 
                            & .045(per C)      & 40 minute lifetime & \citep{Palmer2003}               \\
       Methane          & 1(per C)      & 1 year lifetime   & \citep{Palmer2003}     \\ 
       Ethane            & .54(per C)   & 10 day lifetime   & \citep{Palmer2003}     \\ 
       Propane           & .20(per C)   & 2 day lifetime    & \citep{Palmer2003}     \\ 
       Methylbutanol    & .13(per C)    & 1 hour lifetime  &  \citep{Palmer2003}     \\ 
       HCHO             & 1(per C)      & 2 hour lifetime   &   \citep{Palmer2003}     \\ 
       Acetone          & .67(per C)    & 10 days lifetime  &  \citep{Palmer2003}     \\ 
       Methanol         & 1(per C)      & 2 days lifetime   &  \citep{Palmer2003}     \\ 
       \hline             
    \end{tabular}
    \caption{%
      HCHO yields from various species. \hspace{\textwidth} \\ 
      ${}^b$Calculated through change in concentration of parent and product linear least squares regression. \hspace{\textwidth} \\
      ${}^c$Calculated from satellite detected concentrations of HCHO. \hspace{\textwidth} \\
      ${}^d$Calculated using PTR-MS and iWAS on SENEX campaign data.
    }
    \label{ch_HCHO:tab:VOCLiteratureYields}
    \end{table}

  \subsection{Outline}
    Once we have vertical columns of biogenic HCHO we can infer the local (grid space) isoprene emissions using effective formaldehyde yield from isoprene (todo: cites with examples of this yield, link to yield table).
    This works if there is fast HCHO yield, so that the effect of chemical transport is minimal.
    This yield can be calculated using CTMs such as GEOS-Chem, which was used by \citet{Millet2006} who produced a molar HCHO yield of 2.3.
    The calculations used to determine isoprene emissions over Australia are fully described in \ref{ch_isop:sec:EmissionCalculation}.
    
    The background HCHO is assumed to be equal to HCHO measured in the remote pacific at the same time.
    
    Isoprene quickly forms HCHO in the atmosphere when in the presence of high levels of NO$_X$.
    However, over Australia NO$_X$ levels are generally not high enough and we must take extra care that we can account for the transport or 'smearing' caused by slower HCHO formation.
    Smearing sensitive grid boxes within the model can be detected by running the model with two uniformly differing isoprene emission levels, then finding the grid boxes where the changed HCHO column is greater than can be attributed to local emission difference.
    Using equation \ref{ch_isop:eqn:isop_yield} with two different isoprene emission levels:
    \begin{equation*}
      \hat{S} = \frac{\Delta~\Omega_{HCHO}}{\Delta~E_{ISOP}}
    \end{equation*}
    Consider halving the isoprene emitted globally and rerunning the model, if the local grid HCHO is reduced by much more than half (factoring yield) then you can infer sensitivity to non-local isoprene emissions.
    This can be dependent on local or regional weather patterns, as greater wind speeds will reduce the time any emitted compound stays within the local grid box.
    As such smearing sensitivity is both spatially and temporally diverse, shown in figure TODO: is a picture of the smearing sensitivity over Australia.
   
    Once the smearing sensitive grid squares are filtered out, application of equation \ref{ch_isop:eqn:isop_yield} gives us an estimate of isoprene emissions across the nation.
    
    Most recently a \citet{Bauwens2016} undertook a similar process to what I am doing, although with slightly different focus, using the IMAGESv2 global CTM instead of GEOS-Chem.
    They calculate emissions which create the closest match between model and satellite vertical columns, and compare these postiori data with the apriori (satellite data) and independent data sets.
    (TODO: simple outline of what they did and how my focus is different, this paper will also need to be summarised in the LitReview)
    
  \subsection{Calculation of Emissions}
    \label{ch_isop:sec:EmissionCalculation}
    As is done in \citet{Palmer2003, Millet2006, Bauwens2016}, we assume that HCHO columns are in a steady state, with no horizontal transport.
    In these circumstances the emissions of precursors are easy to calculate as long as we know the molar HCHO yields (Y$_i$) and effective chemical loss rates (k$_i$):
    \begin{equation}
      \Omega_{HCHO} = \frac{1}{k_{HCHO}}\Sigma_i k_i Y_i \Omega_i = \frac{1}{k_{HCHO}}\Sigma_i Y_i E_i
    \end{equation}
    
    Using our measurements of the biogenic HCHO column ($\Omega_{HCHO}$) we can infer the local (grid space) isoprene emissions (E$_{ISOP}$) using effective formaldehyde yield from isoprene (S) (todo: cites with examples of this yield, link to yield table).
    \begin{equation} \label{ch_isop:eqn:isop_yield}
      \Omega_{HCHO} = S \times E_{ISOP} + B
    \end{equation}
    Where \textit{B} is the background HCHO.
    This works if there is fast HCHO yield, so that the effect of chemical transport is minimal.
    The background HCHO is calculated using measurements in the remote pacific at the same time and latitude.
    
  \subsection{Calculation of smearing effect}
    TODO: Smearing scale length, $\hat{S}$ formula, and results of calculations in here.
    
    
  \subsection{Extrapolating the circadian cycle}
    Isoprene emissions occur with regular daily cycles caused by things like local temperature, sunlight, drought, and other environmental factors (TODO: find/cite eucalypt isoprene paper, daily cycle plot if can find).
    
    (TODO: following stuff, add some basic plots and error analysis eventually also)
    Using a model of the daily isoprene emissions fit to the offset determined by satellite HCHO based estimates, we produce a high temporal resolution isoprene emissions inventory.
    During days with more than one HCHO column measurement we can more confidently fit the cycle. 
    For example EOS AURA's OMI measurements from 2004 can be combined with MetOp-A's GOME2 after October 2006, with daily overpasses by OMI and GOME2 at 1345 and 0930 respectively.
    This allows a better retrieval of the daily amplitude of isoprene emissions.
    
  \subsection{Comparison with MEGAN}
    TODO: Direct comparison here, maps of differences for some metrics(monthly average,?). comparison of model run results using different inventory shown in section (reference here)

\section{Model comparison with and without satellite HCHO based inventory}

