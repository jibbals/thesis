\chapter{Introduction and Literature Review} % Chapter Title
\label{LR}

\section{The atmosphere}
\label{LR:Atmos}
  % Overarching description of atmosphere
  The atmosphere is made up of various gases held to the earths surface by gravity. 
  These gases undergo transport on all scales, from barbeque smoke being blown into your face to smoke plumes from forest fires travelling accross the world and depositing in the antarctic snow.
  They take part in various chemical reactions along the way, largely driven by solar input and interactions with eachother.
  Various chemicals are lofted into the atmosphere by soil, trees, factories, cars, seas and oceans, you name it.
  They are also deposited back to the surface both directly and in rain drops.
  
  % Air
  Mostly the atmosphere is made up of nitrogen (N$_2$: $\sim 78\%$), oxygen (O$_2$: $\sim 21\%$), and argon (Ar: $\sim 1\%$).
  Water (H$_2$O) ranges from $0.001$ to $1\%$ depending on evaporation and precipitation.
  Beyond these major constituents the atmosphere has a vast number of \textit{trace gases}, including carbon dioxide (CO$_2$: $\sim 0.4\%$), Ozone (O$_3$: $.000001$ to $0.001\%$), and methane (CH$_4$: $\sim 0.4\%$) \cite[][Ch. 2]{BrasseurJacob2017}.
  Even in small amounts these trace gases can be very important, to the health of both us and our environment.
  
  % TODO: Pressure
  Most of the atmosphere ($\sim 85\%$) is within 10~km of the earths surface.
  This is due to air pressure, which decreases logarithmically with altitude.
  

  % TODO: Structure
  
  \subsection{Structure}
  \label{LR:Atmos:Struct}
    
    % TODO: Boundary layer
    
    % TODO: Troposphere
    
    % TODO: Stratosphere
    
  % Chemistry
  \subsection{Chemistry}
  \label{LR:Atmos:Chem}
    ...
    Oxidation and Radicals
    ...
    Photolysis
    
    One important aspect of the air is ozone, which has impacts on RF, health, crop-yields, etc...
    TODO: take some of o3 paper intro and throw here.
\section{Ozone}
\label{LR:O3}
  %TODO What is ozone
  Ozone (O3) is a toxic greenhouse gas, which is abundant and useful up in the stratosphere.
  
  %TODO Structure of ozone layer chapman eqn etc.
  
  
  % TODO things that affect ozone concentrations 
  \subsection{Stratosphere to troposphere transport}

  
  \subsection{Chemical production}
    Production from precursors: NOx and VOCs
    
    
    Of these precursors VOCs are less well understood, especially in Australia where emissions are largely extrapolated and uncertain.

\section{VOCs}
\label{LR:VOCs}
  % What they are
  
  % Impacts (AQ, Oxidation)
  
  % Isoprene subsection lead in
  Of these VOCs, isoprene has major impacts and is relatively uncertain.
  
  \subsection{Isoprene}
  \label{LR:VOCs:Isop}
    
    
  \subsection{Emissions}
  \label{LR:VOCs:Emissions}
    
  
    % Lead in for HCHO section
    One of the major products of isoprene chemistry is HCHO.
  
\section{HCHO}
\label{LR:HCHO}
  % What is HCHO:
  Sources
  How measured (in-situ, satellite)
  
  \subsection{Satellite measurements}
  \label{LR:HCHO:Sat}
  
\section{Modelling}
\label{LR:Models}