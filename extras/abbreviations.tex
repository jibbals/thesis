\begin{abbreviations}{ll} % Include a list of abbreviations (a table of two columns)

%% ABCD
\textbf{ACCMIP}   & Atmospheric Chemistry and Climate Model Inter-comparison Project \\
\textbf{AAOD}     & Aerosol Absorption Optical Depth \\
\textbf{AMF}      & Air Mass Factor \\
\textbf{AOD}      & Aerosol Optical Depth \\
\textbf{BVOC}     & Biogenic Volatile Organic Compound \\
\textbf{CCD}     & Charged Coupled Device spectrometer \\
\textbf{CPC}      & Climate Prediction Center \\
\textbf{CTM}      & Chemical Transport Model \\
\textbf{DOAS}     & Differential Optical Absorption Spectroscopy \\

%% EFGHIJK
\textbf{ECMWF}    & European Centre for Medium-range Weather Forecasts \\
\textbf{EDGAR}   & Emission Database for Global Atmospheric Research \\
\textbf{ERA-I}    & ECMWF ReAnalysis (Interim) \\
\textbf{GC-FID}  & Gas Chromatographer Flame Ionisation Detector \\
\textbf{FTIR}     & Fourier transform Infra-Red spectrometer\\
\textbf{GEOS}      & Goddard Earth Observing System \\
\textbf{GMAO}      & Global Modeling and Assimilation Office \\
\textbf{GOME}     & Global Ozone Monitoring Experiment \\
\textbf{GPH}    & GeoPotential Height \\
\textbf{HEMCO}   & Harvard-NASA Emissions Component \\
\textbf{IQR}        & Inter-Quartile Range \\

%% LMNOPQ
\textbf{LAI}      & Leaf Area Index \\
\textbf{LT}     & Local Time \\
\textbf{OMHCHO}   & OMI satellite HCHO product \\
\textbf{OMHCHORP} & OMI satellite HCHO product re-processed \\
\textbf{OMI}      & Ozone Monitoring Instrument \\
\textbf{MEGAN}  & Model of Emissions of Gases and Aerosols from Nature \\
\textbf{MUMBA}    & Measurements of Urban, Marine, and Biogenic Air \\
\textbf{NCO}     & National Computing Infrastructure \\
\textbf{NDACC}   & Network for the Detection of Atmospheric Composition Change \\
\textbf{NH}      & Northern Hemisphere \\
\textbf{NMVOC}    & Non-Methane Volatile Organic Compound \\
\textbf{(S,P)OA}  & (Secondary, Primary) Organic Aerosols \\
\textbf{OMR}    & Ozone Mixing Ratio \\
\textbf{PAN}      & PeroxyAcetyl Nitrate \\
\textbf{PFT}      & Plant Functional Type \\
\textbf{PM}       & Particulate Matter \\
\textbf{PTR-MS}   & Proton-Transfer-Reaction Mass spectrometer \\
\textbf{PV}   & Potential Vorticity \\

%% RSTUVWXYZ
\textbf{RA}      & Row Anomaly \\
\textbf{RF}     & Radiative Forcing \\
\textbf{RMA}    & Reduced Major Axis \\
\textbf{RMSA}   & Root Mean Square Error \\
\textbf{RSC}    & Reference Sector Correction \\
\textbf{RTM}     & Radiative Transfer Model \\
\textbf{SAO}     & Smithsonian Astrophysical Observatory \\
\textbf{SH}       & Southern Hemisphere \\
\textbf{SHADOZ} & Southern Hemisphere Additional OZonesonde \\
\textbf{SPS(1,2)}  & Sydney Particulate Studies \\
\textbf{STT}      & Stratosphere to Troposphere Transport \\
\textbf{SZA}     & Solar Zenith Angle \\
\textbf{TOA}      & Top Of the Atmosphere \\
\textbf{TOMS}    & Total Ozone Mapping Spectrometer \\
\textbf{VOC}      & Volatile Organic Compounds \\
\textbf{UCX}     & Universal tropospheric-stratospheric Chemistry eXtension \\
\textbf{UV-Vis}   & Ultraviolet and visible   \\
\textbf{VCC}    & Vertical Column Corrected \\
\textbf{VMR}    & Vertical Mixing Ratio \\
\textbf{WOUDC}    & World Ozone and Ultraviolet Data Centre \\

\end{abbreviations}

%%----------------------------------------------------------------------------------------
%%	PHYSICAL CONSTANTS/OTHER DEFINITIONS
%%----------------------------------------------------------------------------------------

%\begin{constants}{lr@{${}={}$}l} % The list of physical constants is a three column table
%  
%  % The \SI{}{} command is provided by the siunitx package, see its documentation for instructions on how to use it
%  
%  Speed of Light & $c$ & \SI{2.99792458e8}{\meter\per\second} (exact)\\
%  %Constant Name & $Symbol$ & $Constant Value$ with units\\
%  
%\end{constants}

%%----------------------------------------------------------------------------------------
%%	SYMBOLS
%%----------------------------------------------------------------------------------------

\begin{symbols}{lll} % Include a list of Symbols (a three column table)
  
  %Symbol & Name & Unit \\
  $C_x$ & mixing ratio or mole fraction of gas x & molecules of x per molecule of air \\
  $n_x$ & Number density of gas x & molecules of x per unit volume of air \\
  
  
  \addlinespace % Gap to separate the Roman symbols from the Greek
  
  %Symbol & Name & Unit \\
  $\Omega$ & Vertical column & molecules per square centimetre \\
  
\end{symbols}
