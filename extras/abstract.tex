
%----------------------------------------------------------------------------------------
%	ABSTRACT PAGE
%----------------------------------------------------------------------------------------
\begin{abstract}
  \addchaptertocentry{\abstractname} % Add the abstract to the table of contents
  %The Thesis Abstract is written here (and usually kept to just this page). The page is kept centered vertically so can expand into the blank space above the title too\ldots

%% Background paragraph
%
Ozone in the troposphere is a toxic pollutant that causes respiratory and agricultural damage. 
The two main sources of tropospheric ozone are chemical production, and transport from the stratosphere.
Chemical production can occur following biogenic emissions of volatile organic compounds (VOCs) when they mix with polluted urban air. 
Most tropospheric ozone is formed through chemical reactions involving nitrogen oxides, the hydroxyl radical, and VOCs.
Most emitted VOCs are of biogenic origin, and the primary biogenic VOC emitted to the atmosphere from land is isoprene; however, estimates of isoprene emission rates are highly uncertain.
Transport of ozone from the stratosphere is also uncertain, and difficult to measure.
These uncertainties affect atmospheric chemistry models, reducing confidence in modelled atmospheric processes such as radiative forcing and air quality forecasting. 
%Isoprene, formaldehyde, and ozone in the troposphere are linked by oxidative chemistry and are all important to air quality, climate, and radiation budgets.
%% Thesis aims:
This thesis has three aims: to quantify formaldehyde amounts over Australia observed by satellite using a global chemical transport model (GEOS-Chem), to determine Australian isoprene emissions using modelled isoprene-to-formaldehyde yields along with satellite formaldehyde amounts, and to attribute ozone in the troposphere to the contributions from chemical production and stratospheric transport. 
Model output and observations are combined for each aim in this thesis.


%% Modelling chapter:
%
Formaldehyde measurement records from satellite instruments can be used to estimate isoprene emission rates.
Satellite formaldehyde measurements require modelled a priori vertical profiles of concentration.
Corrections are required to remove the influence of this a priori profile when comparing satellite products against models or other measurements. 
In this thesis, formaldehyde measurements were recalculated in two different ways (removing the effects from the old a priori), with both recalculation techniques and outcomes analysed.
The method used to recalculate vertical column amounts significantly affected the outcome.
Measurements affected by anthropogenic and pyrogenic sources were filtered out using other satellite products, and then satellite pixels were gridded into a daily dataset.
These data were formed to be used with modelled isoprene-to-formaldehyde yields in order to estimate isoprene emissions.

%% Isoprene chapter
%
Isoprene is predominantly emitted by trees and shrubs, and Eucalypts are potentially very high emitters. 
Isoprene oxidises in the atmosphere to form formaldehyde, which has a sufficiently short lifetime in the atmosphere to establish chemical equilibrium.
In this thesis, A linear model was created using modelled isoprene emissions and associated formaldehyde enhancements over Australia.
This model was applied to satellite formaldehyde to create a new estimate of Australian isoprene emissions.
%Results are compared against existing measurement campaigns for validation and analysis. 
Isoprene emissions from Australian forests were found to be substantially lower than previous estimates have predicted.
This leads to an overall lowering of surface background ozone concentrations of approximately 5\%.

%% Ozone chapter
%
The second most important source of tropospheric ozone is the stratosphere, which occasionally mixes into the troposphere bringing ozone-rich air masses down towards the Earth's surface. 
In this thesis, an analysis of the local weather patterns and ozone seasonality was performed, showing most stratosphere to troposphere transport occurs during low pressure frontal weather systems. 
This work provides a novel technique using a mathematical (Fourier band-pass) filter on ozonesonde profiles for estimation and quantification of tropospheric ozone transported from the stratosphere. 
An estimate encompassing three measurement stations over the Southern Ocean near Australia of about $7.2 \times 10^{17}$ \moleccm ~per year was derived.

Overall, this thesis improves knowledge of two important natural sources of Australian tropospheric ozone concentrations by examining and estimating isoprene emissions and by characterising stratospheric intrusions based on ozonesondes.  
%Overall this thesis has contributed to the understanding of two important natural sources of ozone over Australia by improving our understanding of isoprene emissions and by improving the characterisation of stratospheric intrusions.
%Through chemical modelling, impacts on tropospheric ozone from these sources has been quantified and analysed for several regions over Australia.
%A top-down estimate of isoprene emissions, a dominant biogenic VOC and one of the least characterised in Australia, has also been created and analysed.
  
\end{abstract}
