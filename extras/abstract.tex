
%----------------------------------------------------------------------------------------
%	ABSTRACT PAGE
%----------------------------------------------------------------------------------------
\begin{abstract}
  \addchaptertocentry{\abstractname} % Add the abstract to the table of contents
  %The Thesis Abstract is written here (and usually kept to just this page). The page is kept centered vertically so can expand into the blank space above the title too\ldots

Ozone in the troposphere is a toxic pollutant which causes respiratory and agricultural damage. 
The two main precursors are biogenic emissions of volatile organic compounds (BVOCs) and transport from the stratosphere. 
Most tropospheric ozone is formed through chemical reactions involving nitrogen oxides, the hydroxy radical, and volatile organic compounds (such as isoprene). Atmospheric chemistry transport models (CTMs) have uncertain BVOC emissions in Australia, affecting our ability to model and predict ozone production along with other important atmospheric processes. Isoprene, formaldehyde, and ozone in the troposphere are linked by oxidative chemistry and are all important to air quality, climate, and radiation budgets. My thesis has three aims: recalculating formaldehyde amounts over Australia seen by satellite using a global CTM (GEOS-Chem), determination of isoprene emissions using modelled formaldehyde yields along with satellite formaldehyde amounts, and attribution of ozone in the troposphere to chemical production and stratospheric transport. Model and observations are combined for each aim in this thesis.

To quantify isoprene emissions (the dominant BVOC), formaldehyde (one of the main products of isoprene oxidation) is used as a proxy. Formaldehyde observed by satellite over Australia is calculated based on modelled a priori vertical distributions using a CTM and a radiative transfer model. In order to compare satellite formaldehyde products against models or other measurements, corrections are required to remove the influence of the a priori profile. Impacts on formaldehyde levels from anthropogenic and pyrogenic sources are determined and filtered out to determine the biogenic footprint and minimise bias in isoprene emissions quantification.

Isoprene is predominantly emitted by trees and shrubs, and Eucalypts are potentially very high emitters. Subsequent oxidation reactions form formaldehyde which has a sufficiently long lifetime in the atmosphere to establish chemical equilibrium. Using a simple linear model and assuming minimal transport and a chemical steady state allows an estimate of the yield of formaldehyde from isoprene emissions. This yield is modelled over Australia and then applied to the recalculated satellite formaldehyde to create a new estimate of Australian isoprene emissions. This technique is used to improve isoprene emissions estimates without the need for extensive measurement campaigns. Results are compared against existing measurement campaigns for validation and analysis. Isoprene emissions from Australian forests appear substantially lower than previous estimates have predicted.

The second most abundant source of tropospheric ozone is the stratosphere, which occasionally mixes into the troposphere bringing ozone-rich air masses down towards the earth's surface. Analysing the local weather patterns and ozone seasonality, most transport is seen to occur during low pressure frontal weather systems. This work provides a novel technique using a Fourier filter on ozone profiles for estimation and quantification of tropospheric ozone transported from the stratosphere. An estimate encompassing three measurement stations over the Southern Ocean near Australia of about $7.2 \times 10^{17}$ \moleccm ~per year is derived.

Overall, this work should improve the knowledge of tropospheric ozone and its precursors for Australia.  
  
  
\end{abstract}
