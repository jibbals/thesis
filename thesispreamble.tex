\documentclass[
11pt, % The default document font size, options: 10pt, 11pt, 12pt
%oneside, % Two side (alternating margins) for binding by default, uncomment to switch to one side
english, % ngerman for German
singlespacing, % Single line spacing, alternatives: onehalfspacing or doublespacing
%draft, % Uncomment to enable draft mode (no pictures, no links, overfull hboxes indicated)
%nolistspacing, % If the document is onehalfspacing or doublespacing, uncomment this to set spacing in lists to single
%liststotoc, % Uncomment to add the list of figures/tables/etc to the table of contents
%toctotoc, % Uncomment to add the main table of contents to the table of contents
%parskip, % Uncomment to add space between paragraphs
%nohyperref, % Uncomment to not load the hyperref package
headsepline, % Uncomment to get a line under the header
%chapterinoneline, % Uncomment to place the chapter title next to the number on one line
%consistentlayout, % Uncomment to change the layout of the declaration, abstract and acknowledgements pages to match the default layout
]{MastersDoctoralThesis} % The class file specifying the document structure

\usepackage[utf8]{inputenc} % Required for inputting international characters
\usepackage[T1]{fontenc} % Output font encoding for international characters

\usepackage{mathpazo} % Use the Palatino font by default

% sort by name, year, title (nyt)
%\usepackage[sorting=nyt,backend=bibtex,style=numeric,natbib=true]{biblatex}
% Use the bibtex backend with the authoryear citation style (which resembles APA)
%\usepackage[sorting=nyt,backend=bibtex,style=authoryear,natbib=true]{biblatex}
\usepackage[sorting=nyt,backend=bibtex,style=authoryear]{biblatex}
% Try to get author, (year) with parenthesis
% \citet{} or \textcite{} gives author, (year)
% \citep{} or \parencite{} gives (author, year)
% don't need natbib=True if using second option... may be better?

\addbibresource{references.bib} % The filename of the bibliography

\usepackage[autostyle=true]{csquotes} % Required to generate language-dependent quotes in the bibliography

%-----------------------------------------------
% --------------- Added by Jesse ---------------
%-----------------------------------------------

% seting level of numbering (default for "report" is 3). With ''-1'' you have non number also for chapters
\setcounter{secnumdepth}{4}
% if you want all the levels in your table of contents 
\setcounter{tocdepth}{4} 

% allow full bibentry in text
\usepackage{bibentry}


% Dimensions from UOW thesis guidelines.
%\pdfpagewidth=\paperwidth 
%\pdfpageheight=\paperheight
% This acts as a failsafe to ensure things aren't stretched or moved when it's finally printed as a PDF.
%\setlength{\parindent}{4ex}	% Sets the length of the paragraph indent. Current setup has a an indent. Disable this if you activate the return line above.
%\usepackage{setspace}
%% Double or one and a half spacing.
\usepackage{graphicx}
%\DeclareGraphicsRule{.tif}{png}{.png}{`convert #1 `dirname #1`/`basename #1 .tif`.png}
%% Graphics. Remove me and you won't have any figures, and that would be very boring.
%%\usepackage[usenames,dvipsnames,svgnames,table]{xcolor}
\usepackage{xcolor}
%% Adds the ability to make coloured text and lines throughout the document. See documentation for xcolor.
%%-------------------- Tables, figures and captions
%\usepackage[font={small},labelfont={bf},margin=4ex]{caption}
%% Makes bold labeled and smaller font captions. Must be loaded before the longtable package to avoid conflicts! 
%\usepackage{longtable} 
\usepackage{threeparttable}
%% Long tables (more than one page). Different headers and footers for beginning and end pages, etc.
%\usepackage{tabularx}
%\usepackage{afterpage} 
%% Make a longtable start on the next clear page, but fills the previous one with text first (no random gaps in the text-from long tables anymore! Man, the day I discovered this...)
%\usepackage{booktabs} 
%% Nice looking tables and lines in tables
%\usepackage{multirow} 
%% Entries in tables over multiple rows
%\usepackage{lscape} 
%% Pages in landscape
%\usepackage{pdflscape} 
%% Landscape pages also rotated in the pdf
%\usepackage{wrapfig} 
%% Allows figures that don't take up the entire width of the page, wrapping the text around the figure
%%\usepackage[position=top,singlelinecheck=false,captionskip=4pt]{subfig} 
%% Multiple figures in an individual figure. Fig. 1 a, b, c, etc. each with, or without, it's own individual caption, and with a global caption for all sub figures.
%
%%-------------------- Special symbols and fonts
\usepackage{amssymb} 
\usepackage{amsmath}
%% Maths symbols
%
\usepackage{relsize}
\usepackage[unicode=true,colorlinks=true,linkcolor=black,citecolor=black,anchorcolor=black,filecolor=black,menucolor=black,runcolor=black,urlcolor=black,breaklinks=true,hidelinks]{hyperref}
%
% Package for chemical equation typesetting: \ce{...}
\usepackage[version=4]{mhchem}

%% --------------------------------------
%% --- my common symbols for stuff ------
%% --------------------------------------
% Different vertical columns : use $\Ogc$

\newcommand{\Oomi}{\Omega_{OMI}} % total column hcho from satellite (PP corrected)
\newcommand{\Ogc}{\Omega_{GC}}  % total column hcho from geos chem
\newcommand{\Ogca}{\Omega_{GC}^{\alpha}} % total column HCHO after scaling geos-chem emissions
\newcommand{\Os}{\Omega_{S}} % Slant column

\newcommand{\apri}{E_{GC}} % a priori
\newcommand{\apost}{E_{OMI}} % a postiori



%% --------------------------------------
%% --- my commands for easy units -------
%% --------------------------------------
% Tg per year
\newcommand{\tgpyr}{~Tg yr$^{-1}$}
\newcommand{\tgcpyr}{~Tg C yr$^{-1}$}
% Tg Carbon -> Tg isoprene factor = (5*12.011 + 8*10.008)/(5*12.001) = 1.13427691283
\newcommand{\degr}{$^{\circ}$}
\newcommand{\roo}{ROO$\dot{}$ } %% hydroxyperoxyl radical (isopoo, roo, ro2,...)
% Molecules per cm2 (per s)
\newcommand{\moleccm}{~molec cm$^{-2}$}
\newcommand{\moleccms}{~molec cm$^{-2}$ s$^{-1}$}

\newcommand{\lowhr}{$2^{\circ} \times 2.5^{\circ}$}
\newcommand{\highhr}{$0.25^{\circ} \times 0.3125^{\circ}$}

\newcommand{\fullref}[1]{% command to do numbered and named reference 
  \ref{#1} \nameref{#1}
}

\newcommand{\mypic}[3]{% Command for quick picture adding
\begin{figure}
  \includegraphics[width=\textwidth]{#1}
  \caption{#2}
  #3
\end{figure}
}

% Same but with width as an input
\newcommand{\mypicw}[4]{
  \begin{figure}
    \centering
    \captionsetup{width=#1}
    \includegraphics[width=#1]{#2}
    \caption{#3}
    #4
  \end{figure}
}


% Same but with width and height as inputs
\newcommand{\mypicwh}[5]{
  \begin{figure}
    \captionsetup{width=#1}
    \includegraphics[width=#1,totalheight=#2]{#3}
    \caption{#4}
    #5
  \end{figure}
}


% highlight stuff in math environment
\newcommand{\highlight}[1]{%
  \colorbox{red!50}{$\displaystyle#1$}}
% eg: \highlight{\beta_{0}} 

%% Package for tracking examiners changes
\usepackage[final]{changes}
% add [final] option to remove markup
\definechangesauthor[name=Hart, color=blue]{mh}
\definechangesauthor[name=Butler, color=cyan]{tb}
%%-----------------------------------------------
%%----------End of Added by Jesse ---------------
%%-----------------------------------------------

%%----------------------------------------------------------------------------------------
%%	MARGIN SETTINGS
%%----------------------------------------------------------------------------------------

%inner=4cm, outer=2cm, top=3cm, bottom=2cm % FROM UOW THESIS GUIDELINES
\geometry{
	paper=a4paper, % Change to letterpaper for US letter
	inner=4cm,  %2.5cm, % Inner margin
	outer=2cm,  %3.8cm, % Outer margin
	bindingoffset=.5cm, % Binding offset
	top=3cm,   %1.5cm, % Top margin
	bottom=2cm, %1.5cm, % Bottom margin
	%showframe, % Uncomment to show how the type block is set on the page
}