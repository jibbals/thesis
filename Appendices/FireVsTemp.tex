% Appendix Template

\chapter{Relationship between temperature and HCHO} % Main appendix title
\label{App:hcho_vs_temp}

  % Used to be subsubsection looking at hcho vs temp vs fire...
  Biogenic HCHO concentrations are correlated with temperature, as isoprene emissions are strongly correlated with temperature \parencite{Palmer2006, Zhu2013_poster, Surl2018}.
  Fires emit HCHO precursors and increase HCHO concentrations independently of the relationship between temperature and HCHO, and should be revealed as outliers when comparing HCHO to temperature.
  Figures \ref{App:hcho_vs_temp:fig_hcho_vs_temp_SEA_200501} - \ref{App:hcho_vs_temp:fig_hcho_vs_temp_SWA_200501} show the relationship between modelled temperature, and satellite HCHO for January 2005 within subsets of Australia.
  A reduced major axis regression is used to determine the correlation between surface temperature (X axis) and HCHO (Y axis).
  Using the natural log of HCHO we can take the linear regression and then exponentiate each side in the equation $\ln{Y} = m{X}+b$ to get ${Y} = \exp{m{X}+b}$. 
  This gives us the exponential fit as shown, with the correlation coefficient between $\ln{HCHO}$ and temperature.
  The distributions of exponential correlation coefficient and $m$ terms is shown in the embedded plot, with one data point available for each grid square where the regression is performed.
  These figures show that the modelled temperature is not well correlated with corrected recalculated OMI vertical columns (r ranges from -0.24 to 0.47), but is with modelled columns (r ranges from 0.49 to 0.84).
  Correlations between modelled temperatures and HCHO are further improved when using the spatial average within each region.
  Furthermore the relationship is improved in individual grid squares over south Eastern Australia by removing non-biogenic emissions from the model (r increased from 0.58 to 0.75).
  This improvement is not seen in Northern Australia, nor south Western Australia.
  Overall this suggests that modelled correlations between temperature and HCHO are spatially dependent, and generally well reflected in satellite measurements over Australia.
  
  % Figure produced in GC_tests.HCHO_vs_temp
  \begin{figure}
    \includegraphics[width=\textwidth]{Figures/OMI_link/GC/HCHO_vs_temp_SEA_20050101-20050228.png}
    \caption{%
      Top row (left): surface temperature averaged over January and February 2005.
      Top row (right): correlation between spatially averaged GEOS-Chem temperatures and recalculated satellite vertical columns.
      Second row: GEOS-Chem surface temperatures correlated against GEOS-Chem HCHO, with different colours for each grid box, and black showing the spatially averaged correlation over time.
      Third row: as second row, except GEOS-Chem HCHO comes from the biogenic emissions only simulation.
      A reduced major axis regression is used within each gridbox using daily overpass time surface temperature and HCHO.
      The distribution of slopes (solid) and regression correlation coefficients (dashed) for the exponential regressions is shown in the inset panels in rows 2 and 3.
    }
    \label{App:hcho_vs_temp:fig_hcho_vs_temp_SEA_200501}
  \end{figure}
  
  % Figure from ???
  \begin{figure}
    \includegraphics[width=\textwidth]{Figures/OMI_link/GC/HCHO_vs_temp_NA_20050101-20050228.png}
    \caption{%
      As Figure \ref{App:hcho_vs_temp:fig_hcho_vs_temp_SEA_200501} but for northern Australia.
    }
    \label{App:hcho_vs_temp:analysis:fig_hcho_vs_temp_NA_200501}
  \end{figure}
  
  \begin{figure}
    \includegraphics[width=\textwidth]{Figures/OMI_link/GC/HCHO_vs_temp_SWA_20050101-20050228.png}
    \caption{%
      As Figure \ref{App:hcho_vs_temp:fig_hcho_vs_temp_SEA_200501} but for south-western Australia.
      \deleted{TODO: Fix y axis in subplot 322}
      %%TODO: Fix y axis in subplot 322
    }
    \label{App:hcho_vs_temp:fig_hcho_vs_temp_SWA_200501}
  \end{figure}
  
  One problem with detecting outliers in the temperature and enhanced HCHO relationship is that days when fires occur are likely to be hot.
  Another problem with correlating heat and HCHO is that increased temperature accelerates HCHO destruction \parencite{Zheng2015}.
  We test the fire mask (see Section \ref{Model:filter:fire}) by examining the relationship between modelled temperature and satellite HCHO with and without applying the filters for smoke and active fires.
  %Figure \ref{Model:filter:fire:fig_VCC_vs_GC_temperature} show the regressions between OMI HCHO total columns and temperature from GEOS-Chem output and CPC daily maximum temperatures.
  Figure \ref{App:hcho_vs_temp:fig_HCHO_vs_temperature} show the regressions between midday surface HCHO and temperature, where two independent data sets are used for the temperature: one from GEOS-Chem output and the other from CPC daily maxima.
  Regressions over both the grid square containing Sydney and averaged over a wider area are reduced in quality when after applying the pyrogenic filter. 
  This could be due to the low resolution of available GEOS-Chem midday output, which greatly increases the strictness of the fire filter, which originally is created at \highhr ~horizontal resolution but here is widened out to \lowhr.
  While the modelled correlation between surface HCHO and temperature is quite strong ($r>0.8$) the correlation coefficient is reduced while the slope (temperature coefficient) is changed by more than 30\% when applying the pyrogenic filter at the model resolution.
  If we use recalculated satellite HCHO columns instead of modelled midday outputs, then higher horizontal resolution can be achieved; however, this is at the cost of vertical resolution.
  Figure \ref{App:hcho_vs_temp:fig_VCC_vs_temperature} shows the relationship between total column HCHO and temperature, which is not as strong as that shown between surface HCHO and temperature.
  At this resolution the application of the pyrogenic filter is shown to slightly strengthen the correlation over the wider south eastern Australian region, with r increasing from $\sim{0.31}$ to $\sim{0.38}$.
  In this thesis the pyrogenic filter is applied at the higher horizontal resolution (\highhr), and this analysis suggests that the filter should strengthen the relationship between total column HCHO and its biogenic precursors over Australia.
  
  %Figure from chapter_2_modelling.HCHO_vs_temperature
  \mypic{Figures/OMI_link/Filters/HCHO_vs_temp_surface_20051201-20060228.png}{%
    Surface HCHO from GEOS-Chem overpass output (midday) on the Y axis vs surface temperatures at midday ($T_{GC}$) and vs maximum daily temperatures from the CPC data set ($T_{CPC}$).
    Top row: Sydney grid square scatter plot and regression with one data point for each day in the summer of 2005-2006.
    Bottom row: as top row except averaging over several grid boxes covering south eastern Australia (SEA: 37\degr S to 29\degr S, 146\degr E to 153.5\degr W).
    Grid squares with pyrogenic influence detected are removed (prior to any averaging) in the right column. 
  }{\label{App:hcho_vs_temp:fig_HCHO_vs_temperature}}
  
  %Figure from chapter_2_modelling.HCHO_vs_temperature
  \mypic{Figures/OMI_link/Filters/HCHO_vs_temp_20051201-20060228.png}{%
    OMI recalculated vertical columns of HCHO on the Y axis vs surface temperatures at midday ($T_{GC}$) and vs maximum daily temperatures from the CPC data set ($T_{CPC}$).
    Top row: Sydney grid square scatter plot and regression with one data point for each day in the summer of 2005-2006.
    Bottom row: as top row except averaging over several grid boxes covering south eastern Australia (SEA: 37\degr S to 29\degr S, 146\degr E to 153.5\degr W).
    Grid squares with pyrogenic influence detected are removed (prior to any averaging) in the right column. 
  }{\label{App:hcho_vs_temp:fig_VCC_vs_temperature}}